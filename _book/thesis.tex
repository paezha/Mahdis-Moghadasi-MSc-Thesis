%%%%%%%%%%%%%%%%%%%%%%%%%%%%%%%%%%%%%%%%%
% McMaster Masters/Doctoral Thesis
% LaTeX Template
% Version 2.2 (11/23/15)
%
% This template has been downloaded from:
% http://www.LaTeXTemplates.com
% Then subsequently from http://www.overleaf.com
%
% Version 2.0 major modifications by:
% Vel (vel@latextemplates.com)
%
% Original authors:
% Steven Gunn  (http://users.ecs.soton.ac.uk/srg/softwaretools/document/templates/)
% Sunil Patel (http://www.sunilpatel.co.uk/thesis-template/)
%
% Modified to McMaster format by Benjamin Furman (contact: https://www.xenben/com; Most up
% to date template at https://github.com/benjaminfurman/McMaster_Thesis_Template,
% occasionally updated on Overleaf template page)
%
% Modified for macdown by Antonio Paez; most up to date version at https://github.com/paezha/macdown
%
% License:
% CC BY-NC-SA 3.0 (http://creativecommons.org/licenses/by-nc-sa/3.0/)
%
%%%%%%%%%%%%%%%%%%%%%%%%%%%%%%%%%%%%%%%%%

%----------------------------------------------------------------------------------------
% DOCUMENT CONFIGURATIONS
%----------------------------------------------------------------------------------------

\documentclass[
11pt, % The default document font size, options: 10pt, 11pt, 12pt
oneside, % Two side (alternating margins) for binding by default, uncomment to switch to one side
english, % other languages available
singlespacing, % Single line spacing, alternatives: onehalfspacing or doublespacing
%draft, % Uncomment to enable draft mode (no pictures, no links, overfull hboxes indicated)
%nolistspacing, % If the document is onehalfspacing or doublespacing, uncomment this to set spacing in lists to single
%liststotoc, % Uncomment to add the list of figures/tables/etc to the table of contents
%toctotoc, % Uncomment to add the main table of contents to the table of contents
]{macthesis} % The class file specifying the document structure

%----------------------------------------------------------------------------------------
% Import packages here
%----------------------------------------------------------------------------------------
\usepackage[utf8]{inputenc} % Required for inputting international characters
\usepackage[T1]{fontenc} % Output font encoding for international characters
\usepackage{lastpage} % count pages
\usepackage{lmodern} % could change font type by calling a different package
\usepackage{lscape} % for landscaping pages
% New commands for landscape orientation
\newcommand{\blandscape}{\begin{landscape}}
\newcommand{\elandscape}{\end{landscape}}
%
\usepackage{siunitx} % for scientific units (micro-liter, etc)
\setcounter{tocdepth}{2} % so that only section and sub sections appear in Table of Contents. Remove or set depth to 3 to include sub-sub-sections

%----------------------------------------------------------------------------------------
% Define a blank page
%----------------------------------------------------------------------------------------
\def\blankpage{%
      \clearpage%
      \thispagestyle{empty}%
      \addtocounter{page}{-1}%
      \null%
      \clearpage}

%----------------------------------------------------------------------------------------
% Define a tight list
%----------------------------------------------------------------------------------------
\def\tightlist{}

%----------------------------------------------------------------------------------------
%	Highlight Code Chunks
%----------------------------------------------------------------------------------------

%----------------------------------------------------------------------------------------
% Handling Citations
%----------------------------------------------------------------------------------------
\usepackage[backend=biber, giveninits=true, doi=false, natbib=true, url=false, eprint=false, style=authoryear, sorting=nyt, maxcitenames=2, maxbibnames=99, uniquename=false, uniquelist=false, dashed=false]{biblatex} % can change the maxbibnames to cut long author lists to specified length followed by et al., currently set to 99.
% package xurl wraps long url in the citations.
\usepackage{xurl}
\DeclareFieldFormat[article,inbook,incollection,inproceedings,patent,thesis,unpublished]{title}{#1\isdot} % removes quotes around title
\renewbibmacro*{volume+number+eid}{%
  \printfield{volume}%
%  \setunit*{\adddot}% DELETED
  \printfield{number}%
  \setunit{\space}%
  \printfield{eid}}
\DeclareFieldFormat[article]{number}{\mkbibparens{#1}}
%\renewcommand*{\newunitpunct}{\space} % remove period after date, but I like it.
\renewbibmacro{in:}{\ifentrytype{article}{}{\printtext{\bibstring{in}\intitlepunct}}} % this remove the "In: Journal Name" from articles in the bibliography, which happens with the ynt
\renewbibmacro*{note+pages}{%
    \printfield{note}%
    \setunit{,\space}% could add punctuation here for after volume
    \printfield{pages}%
    \newunit}
\DefineBibliographyStrings{english}{% clears the pp from pages
  page = {\ifbibliography{}{\adddot}},
  pages = {\ifbibliography{}{\adddot}},
}
\DeclareNameAlias{sortname}{last-first}
\renewcommand*{\nameyeardelim}{\addspace} % remove comma in text between name and date
\addbibresource{Bibliography.bib} % The filename of the bibliography
\usepackage[autostyle=true]{csquotes} % Required to generate language-dependent quotes in the bibliography

% you'll have to play with the citation styles to resemble the standard in your field, or just leave them as is here.
% or, if there is a bst file you like, just get rid of all this biblatex stuff and go back to bibtex.

% This code is to fix cslreferences in new pandoc see: https://github.com/mpark/wg21/issues/54
%%\newlength{\cslhangindent}
%\setlength{\cslhangindent}{1.5em}
%\newenvironment{CSLReferences}%
%  {}%
%  {\par}
%
% https://github.com/ismayc/thesisdown/issues/133
% From {rticles}
\newlength{\csllabelwidth}
\setlength{\csllabelwidth}{3em}
\newlength{\cslhangindent}
\setlength{\cslhangindent}{1.5em}
% for Pandoc 2.8 to 2.10.1
\newenvironment{cslreferences}%
  {}%
  {\par}
% For Pandoc 2.11+
% As noted by @mirh [2] is needed instead of [3] for 2.12
\newenvironment{CSLReferences}[2] % #1 hanging-ident, #2 entry spacing
 {% don't indent paragraphs
  \setlength{\parindent}{0pt}
  % turn on hanging indent if param 1 is 1
  \ifodd #1 \everypar{\setlength{\hangindent}{\cslhangindent}}\ignorespaces\fi
  % set entry spacing
  \ifnum #2 > 0
  \setlength{\parskip}{#2\baselineskip}
  \fi
 }%
 {}
\usepackage{calc} % for calculating minipage widths
\newcommand{\CSLBlock}[1]{#1\hfill\break}
\newcommand{\CSLLeftMargin}[1]{\parbox[t]{\csllabelwidth}{#1}}
\newcommand{\CSLRightInline}[1]{\parbox[t]{\linewidth - \csllabelwidth}{#1}}
\newcommand{\CSLIndent}[1]{\hspace{\cslhangindent}#1}

%----------------------------------------------------------------------------------------
% Collect all your header information from the chapters here, things like acronyms, custom commands, necessary packages, etc.
%----------------------------------------------------------------------------------------
\usepackage{parskip} %this will put spaces between paragraphs
\setlength{\parindent}{15pt} % this will create and indent on all but the first paragraph of each section.
% should maybe change to glossaries package
\usepackage{acro}
\DeclareAcronym{est}{
	short = EST,
	long  = expressed sequence tags
}

\DeclareAcronym{Xl}{
	short = \textit{X.~laevis},
	long  = \textit{Xenopus~laevis}
}
\DeclareAcronym{Xg}{
	short = \textit{X.~gilli},
	long  = \textit{Xenopus~gilli}
}

\usepackage{etoolbox}
\preto\chapter{\acresetall} % resets acronyms for each chapter

\usepackage{xspace} %helps spacing with custom commands.
\newcommand{\oddname}{{\sc SoME goOfY LonG ThiNg With an AwkWarD NAme}\xspace}


\usepackage{pgfplotstable} % a much better way to handle tables
\pgfplotsset{compat=1.12}

% \usepackage{float} % if you need to demand figure/table placement, then this will allow you to use [H], which demands a figure placement. Beware, making LaTeX do things it doesn't want may lead to oddities.


%%%%
% LINK COLORS
% You can control the link colors at the end of the McMasterThesis.cls file. There is also a true/false option there to turn off all link colors.
%%%%


%----------------------------------------------------------------------------------------
%	THESIS INFORMATION
%----------------------------------------------------------------------------------------

\title{Active Transportation Modes: Data Requirements and Historical Analysis of Impedance Functions}
%\thesistitle{Thesis Title} % Your thesis title, print it elsewhere with \ttitle
\author{Mahdis Moghadasi}
%\author{John \textsc{Smith}} % Your name, print it elsewhere with \authorname
\bdegree{}
\mdegree{}
%Previous degrees % print it elsewhere with \bdeg and \mdeg
\date{}
% The month and year that you submit your FINAL draft TO THE LIBRARY (May or December)
\university{McMaster University}
%\university{\href{http://www.mcmaster.ca/}{McMaster University}} % Your university's name and URL, print it elsewhere with \univname
%\division{}
\faculty{Faculty of Science} % Your faculty's name and URL, print it elsewhere with \facname
\department{School of Earth, Environment and Society} % Your department's name and URL, print it elsewhere with \deptname
\subject{Geography} % Your subject area, print it elsewhere with \subjectname
%\group{\href{http://researchgroup.university.com}{Research Group Name}} % Your research group's name and URL, print it elsewhere with \groupname
\supervisor{Antonio Paez}
%\supervisor{Dr. Jane \textsc{Smith}} % Your supervisor's name, print it elsewhere with \supname
\examiner{} % Your examiner's name, print it elsewhere with \examname
\degree{Master of Science}
%\degree{Doctor of Philosophy} % Your degree name, print it elsewhere with \degreename
\addresses{} % Your address, print it elsewhere with \addressname
\keywords{} % Keywords for your thesis, print it elsewhere with \keywordnames


% this sets up hyperlinks
\hypersetup{pdftitle=\ttitle} % Set the PDF's title to your title
\hypersetup{pdfauthor=\authorname} % Set the PDF's author to your name
\hypersetup{pdfkeywords=\keywordnames} % Set the PDF's keywords to your keywords

\begin{document}
\sloppy

\frontmatter % Use roman page numbering style (i, ii, iii, iv...) for the pre-content pages

\pagestyle{plain} % Default to the plain heading style until the thesis style is called for the body content

%----------------------------------------------------------------------------------------
%	Half Title (lay title)
%----------------------------------------------------------------------------------------
%\begin{halftitle} % could not get this environment working
%\vspace*{\fill}
\vspace{6cm}
\begin{center}
\ttitle
\end{center}
%\vspace*{\fill}
\pagenumbering{gobble} % leave this here, McMaster doesn't want this page numbered
%\end{halftitle}
\clearpage

%----------------------------------------------------------------------------------------
%	TITLE PAGE
%----------------------------------------------------------------------------------------
\pagenumbering{gobble}
\begin{center}

\vfill
\textsc{\Large \ttitle} \\

\vfill
{By \authorname\, \bdeg }


 \vfill
{\large \textit{A Thesis Submitted to the School of Graduate Studies in the Partial Fulfillment of the Requirements for the Degree \degreename}}\\

\vfill
{\large \univname\, \copyright\, Copyright by \authorname\, \today}\\[4cm] % replace \today with the submission date

\end{center}
\blankpage
\clearpage

%----------------------------------------------------------------------------------------
%	QUOTATION PAGE
%----------------------------------------------------------------------------------------


\blankpage
\clearpage

%%%%%%%%%%%%%%%%%%%%%%%%%%%
%%%%%%%%%%%%%%%%%%%%%%%%%%%
% optional page stuff
%----------------------------------------------------------------------------------------
% can do physical constraints and symbols pages, see the original thesis example on overleaf if you want to include them at https://www.overleaf.com/latex/templates/template-for-a-masters-slash-doctoral-thesis/mkzrzktcbzfl#.VlPeicorpE4
%----------------------------------------------------------------------------------------

%----------------------------------------------------------------------------------------
%	DEDICATION
%----------------------------------------------------------------------------------------


\blankpage
\clearpage


%----------------------------------------------------------------------------------------
%	Descriptive note numbered ii
%----------------------------------------------------------------------------------------
% Need to add below info
\newpage
\pagenumbering{roman} % leave to turn numbering back on
\setcounter{page}{2} % leave here to make this page numbered ii, a Grad School requirement

\noindent % stops indent on next line
\univname \\
\degreename\, (\the\year) \\
Hamilton, Ontario (\deptname) \\[1.5cm]
TITLE: \ttitle \\
AUTHOR: \authorname\,  %list previous degrees
(\univname)  \\
SUPERVISOR: \supname\, \\
NUMBER OF PAGES: \pageref{lastoffront}, \pageref{LastPage}  % put in iv and number

\clearpage

%----------------------------------------------------------------------------------------
%	Lay abstract number iii
%----------------------------------------------------------------------------------------
% not actually included in most theses, though requested by the GSA
% uncomment below lines if you want to include one
\section*{Lay Abstract}
  Active travel, encompassing walking and cycling, is pivotal in fostering sustainable and healthy urban transportation systems. This thesis comprises two interconnected studies comprehensively addressing the complexities and imperatives of active travel analysis within the Canadian context. The first study embarks on a journey through three decades of active travel behavior in Canada. Utilizing data from Canada's General Social Survey, this research calibrates empirical impedance functions to elucidate individuals' willingness to undertake trips to various destinations by walking and cycling. The study not only reveals evolving patterns in active mobility but also offers valuable insights for urban planners and policymakers seeking to elevate walking and cycling as essential components of urban transportation in Canada. The second study evaluates geographic accessibility measures and data specifics necessary for assessing these modes in the Canadian context, shifting the focus to data prerequisites and sources critical for active accessibility analysis. This comprehensive analysis compares metrics, origin-destination considerations, geographic scales, and methods for calculating travel time and distance for walking and cycling. Beyond its relevance in Canada, this examination yields broader insights into the challenges and considerations associated with active travel accessibility analysis worldwide. These integrated studies contribute significantly to the ongoing discourse on active travel and accessibility, emphasizing the need for a transition from mobility-centric to accessibility-focused urban transportation policies. This transition holds the key to creating more sustainable, vibrant, and healthier cities for the future.
\blankpage
\clearpage


%----------------------------------------------------------------------------------------
%	ABSTRACT PAGE number iv
%----------------------------------------------------------------------------------------

\section*{\Huge Abstract}
\addchaptertocentry{\abstractname}
% Type your abstract here.
Active travel, encompassing walking and cycling, has emerged as a vital component in pursuing sustainable, healthy, and efficient urban transportation systems. This thesis comprises two interrelated studies that collectively contribute to a comprehensive understanding of active travel behavior and its analysis within the Canadian context. At the first step, we focused on the data prerequisites and sources required for active accessibility analysis, systematically evaluates geographic accessibility measures and data specificity essential to assessing walking and cycling in the Canadian urban landscape. As accessibility increasingly takes center stage in urban transportation policies, it becomes imperative to understand the unique data requirements and methodological considerations for these non-motorized modes. This review thoroughly examines the accessibility measures used in recent literature and the various data sources and considerations associated with walking and cycling accessibility analysis. The comparative analysis encompasses metrics, origin-destination considerations, geographic scales, and travel time or distance calculations, shedding light on the complexities and nuances of active travel data analysis. In the subsequent phase of our research, we extensively analyzed active travel behavior trends in Canada, spanning three decades from 1986 to 2015. This study uses data from Canada's General Social Survey to examine individual preferences and behaviors related to walking and cycling. The focus is mainly on understanding how these behaviors vary according to different trip purposes and geographical contexts. Our approach involved the intricate calibration of empirical impedance functions to gain insights into the temporal willingness of individuals to travel to various destinations using active modes of transportation. This aspect of the study is crucial in quantifying the time individuals are willing to allocate to walking and cycling to reach their destinations. The investigation notably provides a historical perspective on the evolution of active mobility over the past 35 years. This historical analysis is pivotal in unraveling the changes in active travel behavior, influenced by factors such as urban development trends, societal attitudes towards health and environmental sustainability, and modifications in transportation infrastructure and policy frameworks. Our findings reveal significant and consistent disparities in trip durations between walking and cycling modes, accompanied by intriguing temporal variations over the years. Moreover, the study delves deeper into the analysis of specific destinations, uncovering evolving patterns and distance-decay effects in active travel behavior. These patterns indicate how the proximity of destinations influences the choice and duration of active travel.While primarily centered on the Canadian context, this study offers valuable insights that apply to the global discourse on sustainable and active transportation. It underscores the crucial role of active travel in shaping the future of urban transportation systems, advocating for a shift from traditional mobility-centric approaches to an accessibility-focused paradigm. This shift recognizes walking and cycling as alternative modes and fundamental elements of vibrant, sustainable urban environments. The comprehensive nature of this study goes beyond presenting empirical findings; it contributes significantly to the broader conversation on urban mobility, emphasizing the importance of these modes in promoting healthier, more sustainable, and livable cities. The research findings are a vital resource for urban planners, policymakers, and academic researchers worldwide, guiding decision-making processes to enhance urban mobility. By elucidating the data prerequisites and behavioral patterns associated with active travel, this thesis informs the development of strategies that foster urban settings conducive to active and sustainable transportation methods. These insights are instrumental in advancing our understanding of how to effectively integrate walking and cycling into the urban fabric, thereby supporting the evolution of urban areas towards greater sustainability and livability.
\blankpage
\clearpage

%----------------------------------------------------------------------------------------
%	ACKNOWLEDGEMENTS
%----------------------------------------------------------------------------------------

  \begin{acknowledgements}
  \addchaptertocentry{\acknowledgementname} % Add the acknowledgments to the table of contents
    I am deeply indebted to my esteemed thesis supervisor, Dr.~Antonio Paez, whose guidance, expertise, and invaluable mentorship have significantly influenced the completion of this thesis. Their insights and feedback have been instrumental in shaping this work. In addition, I extend my profound gratitude to my husband and my parents for their unwavering support and belief in my academic endeavors. Their encouragement has been the bedrock of my academic pursuit.
  \end{acknowledgements}
\blankpage
\clearpage

%----------------------------------------------------------------------------------------
%	LIST OF CONTENTS/FIGURES/TABLES PAGES
%----------------------------------------------------------------------------------------

\tableofcontents % Prints the main table of contents

\listoffigures % Prints the list of figures

\listoftables % Prints the list of tables

%----------------------------------------------------------------------------------------
%	ABBREVIATIONS
%----------------------------------------------------------------------------------------
% many theses don't use this section, as it will be declared at first use and again each chapter. Uncomment these four lines to activate if you want
%\clearpage
%\section*{\Huge Acronyms}
%\addchaptertocentry{Acronyms}
%\printacronyms[name] % name without an option stops the header

%----------------------------------------------------------------------------------------
%	DECLARATION PAGE
%----------------------------------------------------------------------------------------

\begin{declaration}
\addchaptertocentry{\authorshipname}

\noindent I, \authorname, declare that this thesis titled, \emph{\ttitle} and the work presented in it are my own. I confirm that:



\end{declaration}


%----------------------------------------------------------------------------------------
% The following bit is just here to make sure we end up on a new page and get the total number of roman numeral
\label{lastoffront}
\clearpage
% make sure this command is on the last of your frontmatter pages, i.e. only this command, a \clearpage then \mainmatter
% should be fine without modification
%----------------------------------------------------------------------------------------

%----------------------------------------------------------------------------------------
%	THESIS MAIN BODY
%----------------------------------------------------------------------------------------

\mainmatter % here the regular arabic numbering starts
\pagestyle{thesis}
\hypertarget{this-is-the-degree-you-are-aiming-for-with-this-thesis}{%
\chapter{This is the degree you are aiming for with this thesis}\label{this-is-the-degree-you-are-aiming-for-with-this-thesis}}

Placeholder

\hypertarget{introduction}{%
\chapter{Introduction}\label{introduction}}

Placeholder

\hypertarget{rmd-basics}{%
\chapter{Data Sources and Needs for Active Travel-Based Accessibility Analysis: A Canadian Perspective}\label{rmd-basics}}

\hypertarget{abstract}{%
\section{Abstract}\label{abstract}}

Active transportation is critical in developing robust and healthy urban transportation policies. As the focus of transportation needs analysis in cities shifts from mobility to accessibility, there is an increasing urgency to evaluate accessibility through cycling and walking. These modes possess distinct characteristics such as lower speeds, shorter trips, and potentially different purposes than motorized travel. Consequently, the data inputs necessary for assessing accessibility differ from those used for motorized travel accessibility studies. This review examines the data sources and requirements for implementing active accessibility analysis. It reviews the geographic accessibility measures and data specific to walking and cycling, as employed in recent literature. Various aspects are compared between walking and cycling accessibility measures, including metrics, origins, destinations considered, geographic scales, and travel time or distance calculations. By comparing approaches for walking and cycling, this report also highlights potential considerations, challenges, and questions that arise when envisioning the future of active travel accessibility-based analysis. While the discussion in this review primarily focuses on the Canadian context, the insights gleaned may have broader applications in other national contexts.

\hypertarget{introduction-1}{%
\section{Introduction}\label{introduction-1}}

For decades, transportation planning has been focused on providing mobility for the private car, initially introduced in North America as a solution to challenges arising from rapid urbanization and replicated in other regions (Angotti, 1996; Brown, Morris, \& Taylor, 2009a). However, it has become evident that mobility centered on the private car is inefficient, inequitable, and unsustainable and requires immediate attention. This includes environmental issues such as climate change (Chapman, 2007), as well as numerous other social (Boschmann \& Kwan, 2008; Lucas, 2012, 2019), health (Khreis et al., 2016; Milne, 2012), and equity concerns (Bocarejo S \& Oviedo H, 2012; Martens, Golub, \& Robinson, 2012; Pereira, Schwanen, \& Banister, 2017).Transportation initiatives have sought to foster mobility polycultures, which include a diverse range of transportation alternatives, in order to reduce car dependency (Lavery, Paez, \& Kanaroglou, 2013; Millera, 2011). Polycultures are resilient and adaptable systems, offering mobility substitutes like information technologies. Notably, these systems are more intricate than monocultures, necessitating a more comprehensive range of mobility technologies and enhanced coordination between modes and travelers. As a result, active travel modes like walking and cycling are pivotal in urban areas, as they contribute to the establishment of robust and health-oriented urban transportation polycultures (Lavery et al., 2013; Lira \& Paez, 2021; Millera, 2011).

Cycling and walking are highly effective modes of transportation for short- and mid-range travel within urban areas. However, urban planning and transportation policies have predominantly prioritized automobile travel, often neglecting other modes of transportation (Brezina, Leth, \& Lemmerer, 2020; Brown, Morris, \& Taylor, 2009b; Koglin, 2020; Ruffino \& Jarre, 2021; Wiersma et al., 2020). This approach has led to concerns regarding the negative externalities of the current car-centric transportation system. Simultaneously, there is increasing recognition and awareness of the numerous co-benefits offered by active mobility, including improvements in public health, enhanced efficiency, and increased social inclusion (Banister, 2005; Garling, Ettema, \& Friman, 2014; Gossling, Choi, Dekker, \& Metzler, 2019; Mueller et al., 2015). Traditionally, transportation planning has primarily revolved around facilitating motorized travel, strongly emphasizing accommodating cars. Both transportation and land use systems have been designed to achieve mobility, often measured through efficiency metrics that disregard the underlying purpose of travel, which is reaching destinations (S. L. Handy \& Niemeier, 1997).

The concept of generating mobility appears straightforward when planning for affordable motorized travel, particularly in an era where automobile users have been deliberately shielded from bearing the total cost of their travel, often without even being aware of it (Taylor, 2006). However, recognizing the inherent contradiction of pursuing mobility while striving to mitigate its negative consequences, the transportation literature has long advocated shifting from mobility-centered planning to accessibility-based planning (S. L. Handy \& Niemeier, 1997; Social Exclusion Unit, 2003). Transportation accessibility is commonly defined as the capacity of transportation-land use systems to provide access to various opportunities (Paez, Scott, \& Morency, 2012a), and this approach directly challenges the shortcomings of mobility-focused planning by emphasizing the ability to reach destinations effectively. While adopting accessibility-based planning in practice has yielded mixed results (Boisjoly \& El-Geneidy, 2017; Proffitt, Bartholomew, Ewing, \& Miller, 2019), there are compelling reasons to believe that the future lies in embracing accessibility as a fundamental principle of urban planning (S. Handy, 2020). By prioritizing accessibility, planners can better align transportation systems with the needs of individuals and communities, ultimately leading to more sustainable, equitable, and efficient urban environments.

The significance of accessibility-based planning becomes even more apparent when considering active modes of transportation. Why would someone prefer longer trips if they could reach their destinations in shorter ones? Unlike motorized travel, pedestrians and cyclists cannot be shielded from the costs associated with their journeys, as the physical effort required to reach destinations is inherently palpable (Hsu \& Tsai, 2014; Iseki \& Tingstrom, 2014; Paez et al., 2020). With the growing interest in active travel-based accessibility (ATB accessibility) worldwide (Arranz-Lopez, Soria-Lara, Witlox, \& Paez, 2019; Li, Huang, \& Axhausen, 2020; Ortega, MartiN, LOPez-Lambas, \& Soria-Lara, 2021; Rosas-Satizabal, Guzman, \& Oviedo, 2020), transportation scholars have built upon decades of accessibility research that predominantly focused on motorized travel.

In principle, accessibility analysis is versatile enough to be applied to ATB accessibility analysis. However, it is crucial to acknowledge the distinctions between motorized and active travel and how these differences can influence the implementation of accessibility measures with a focus on active modes (Iacono, Krizek, \& El-Geneidy, 2010). Active modes of transportation have captivated the attention of researchers due to their profound implications and unique advantages in terms of the environment, health, and social inclusion (Koszowski et al., 2019; Otero, Nieuwenhuijsen, \& Rojas-Rueda, 2018; Pucher, Buehler, Bassett, \& Dannenberg, 2010; Rojas-Rueda, De Nazelle, Tainio, \& Nieuwenhuijsen, 2011; Rojas-Rueda, Nazelle, Teixido, \& Nieuwenhuijsen, 2012; Tinessa, Pagliara, Biggiero, \& Veneri, 2021). They have been linked to various health benefits, such as improved longevity (Hakim et al., 1998), enhanced cognitive function (Weuve et al., 2004), and better quality of life (Leveille, Guralnik, Ferrucci, \& Langlois, 1999; Strawbridge, Cohen, Shema, \& Kaplan, 1996). Moreover, active modes are perceived as cleaner, more efficient, and more sustainable forms of transportation (Bhopal \& Unwin, 1995). Active travel also presents an excellent alternative for mobility, reducing transportation costs for families, promoting gender equality, fostering resilient infrastructure, and contributing to the aesthetic value of the environment (Koszowski et al., 2019). Furthermore, active travel improves accessibility for individuals who do not have access to other modes of transportation and contributes to the development of local and regional economies. However, compared to motorized travel, active travel entails slower speeds, occurs on more minor scales, poses more significant safety risks due to the higher likelihood of severe injuries in collisions, is used to reach potentially different destinations, and involves costs, such as physical exertion, which are typically disregarded in motorized travel analysis (Akgun, Dissanayake, Thorpe, \& Bell, 2018; Ng, Debnath, \& Heesch, 2017; Oehl, Brandenburg, \& Huemer, 2019; Pokorny, Pritchard, \& Pitera, 2018; Useche, Montoro, Sanmartin, \& Alonso, 2019).

The objective of the present study is to investigate active travel-based (ATB) accessibility, with a specific focus on data sources and requirements, utilizing Canada as a case study. This research is motivated by a recent Canadian project establishing data-driven standards for analyzing transportation equity. A thorough understanding of how the input data influences the analysis and outcomes is essential to propose methods that can be applied uniformly across regions. It is worth noting that other reviews on ATB accessibility measures have been conducted (Geurs \& Van Wee, 2004; Iacono et al., 2010; Maghelal \& Capp, 2011; Talen \& Koschinsky, 2013; D. S. Vale, Saraiva, \& Pereira, 2016). This paper's contribution lies in addressing a literature gap by focusing on the data requirements for various ATB accessibility measures, comparing measures that can be consistently implemented in diverse contexts, and identifying the data needs for their standardized implementation.

\hypertarget{background}{%
\section{Background}\label{background}}

Transportation planning has emerged as a distinct field primarily focused on mobility, which refers to the ease of movement. In this context, mobility indicators such as travel speed and time have been proposed, mainly focusing on motorized transportation (Banister, 2008). The concept of accessibility, on the other hand, has long been adopted in spatial and transportation research to evaluate the quality and extent of relationships between the spatial development of an area and its transportation system. Hansen's seminal work 1959 defined accessibility as ``the potential of opportunities for interaction,'' quantifying the number and variety of opportunities accessible from a specific location through the transportation system. This groundbreaking research led to a shift in perspective, emphasizing the inclusion of accessibility as a performance indicator in land use and transportation plans, offering an alternative to mobility-centered planning (Koenig, 1980; Morris, Dumble, \& Wigan, 1979; Wachs \& Kumagai, 1973). Moreover, researchers argued that improved accessibility reflects economic and social benefits within the network, particularly regarding the land value and quality of life (Koenig, 1980; Wachs \& Kumagai, 1973). In recent years, accessibility has become a critical component in land use and transportation planning, particularly regarding social equity, economic development, and environmental impacts (Banister, 2008; S. L. Handy, 2002; Lucas, 2012; Preston \& Raje, 2007). For instance, Geurs and van Wee (2004) deconstructed the concept of accessibility into four elements: land use, transportation, time, and individual factors. \textbf{Land use} encompasses the quality, quantity, and spatial distribution of destination places such as schools, jobs, hospitals, and recreational facilities and the demand for opportunities at origin places. \textbf{Transportation} refers to the transportation system itself, including the disutility experienced by individuals when traveling from an origin to a destination using a specific mode. \textbf{Time} accounts for time constraints regarding the availability of opportunities throughout the day and the time available for individuals to access such opportunities. Finally, the \textbf{individual} factor considers individuals' or groups' specific capabilities and needs, influenced by factors such as income, education level, travel mode availability, age, and household situation. Overall, this body of literature highlights the significance of accessibility as a comprehensive and integrated approach in transportation and land use planning, encompassing various dimensions and recognizing its multifaceted benefits.

Mobility-based approaches emphasize reducing travel time, whereas accessibility planning aims to ensure that all individuals have reasonable travel times to a diverse range of destinations. Consequently, accessibility planning prioritizes active and public transportation and incorporates land use policies that minimize distances between activities (Banister, 2008). Accessibility refers to the ease of reaching valuable destinations, and its calculation requires the implementation of various access measures depending on research objectives, transportation modes, data requirements, activities, land use patterns, and travel costs over connecting transport infrastructure. Consequently, different aspects of accessibility have led to the development of multiple measurement methods and indicators, including proximity, cumulative, gravity, utility-based, and space-time prism models as the prevailing approaches (for example, Geurs \& Van Wee, 2004; Kelobonye et al., 2019; B. H. Lee, Waddell, Wang, \& Pendyala, 2010; Neutens, 2015; Paez, Mercado, Farber, Morency, \& Roorda, 2010; Vandenbulcke, Steenberghen, \& Thomas, 2009). However, there are debates and disagreements concerning evaluating this concept (Castiglione, Hiatt, Chang, \& Charlton, 2006; Fan, Guthrie, \& Levinson, 2012; Wang \& Chen, 2015).

In general, accessibility can be measured at either the individual-based or locational level (place-based) (Miller, 2005). Place-based metrics focus on the physical separation of critical locations, such as an origin and potential destinations, considering the land use and transportation components. Cumulative-opportunity accessibility and gravity accessibility are examples of place-based metrics, reflecting the extent to which people can reach and utilize services, amenities, and opportunities within a specific geographic area, such as a neighborhood, city, or region. On the other hand, individual-based metrics consider individuals' space-time behavior and assess the degree to which a person can access and utilize services, amenities, and opportunities based on personal characteristics, including age, gender, income, mobility, and health status. Space-time accessibility and utility accessibility fall under the category of individual-based metrics. Individual-based accessibility recognizes individuals' unique needs and preferences, considering their ability to access various transportation modes, travel routes, and types of transportation. Sometimes, individual-based accessibility is integrated into location-based studies by stratifying the population based on age groups, socioeconomic characteristics or segmenting destinations (Fan et al., 2012; Harris, 2001; Legrain, Buliung, \& El-Geneidy, 2015, 2016; Paez et al., 2010; D. de S. Vale, 2009). These two approaches are interconnected, and individual-based measures can be viewed as a particular case of place-based measures, where the impedance function and cost remain constant across destinations.

Location-based accessibility offers a notable advantage in terms of ease of computation (Bhat, 2000). Researchers typically rely on average travel time between two points, population data, and aggregated activity data when calculating place-based accessibility. Consequently, cumulative-opportunity accessibility emerges as one of the most widely employed methods for assessing accessibility. Furthermore, space-time accessibility examines how individuals can engage in desired activities while encountering various constraints (Kwan, 1998; M. S. Lee \& McNally, 1998; Miller, 1991). This framework is based on Hagerstrand's (1970) time geography model, which identifies three factors influencing individual accessibility. The first factor is capability constraint, where people can only engage in activities when awake and face travel impediments at all times. The second constraint is coupling, which mandates specific tasks to be performed at designated times. For instance, individuals typically remain at their workplace during working hours (e.g., 8 AM to 5 PM). The final constraint is the authority constraint, which encompasses activity opening hours. As a result, individuals may be unable to access a park late at night due to its closure. Space-time accessibility, as measured by Ilag's (1970) study, captures the number of activities that can be accomplished under various constraints (Ilagcrstrand, 1970). According to the research, space-time accessibility (i.e., person-based accessibility) is considered a superior measure to cumulative-opportunity accessibility (i.e., place-based accessibility) since it accounts for an individual's actual travel behavior. It is attributed to space-time accessibility incorporating temporal variations in transportation systems, activities, and individuals' unique travel patterns.

In contrast, place-based accessibility, which does not fully address temporal constraints, may yield inaccuracies when assessing accessibility. On the other hand, space-time accessibility considers temporal variations in activities(Boisjoly \& El-Geneidy, 2017; Fransen et al., 2015). Place-based accessibility assumes that all activities are carried out during hours chosen at random by researchers. However, activities are only available during their regular business hours. It is necessary to incorporate factors relevant to active transportation to formulate a comprehensive accessibility formula for active transportation modes. Presented here is one potential formula as an illustration equation \ref{eq:accessibility}:

\begin{equation}
A^{pk}_i = \sum _{j=1} ^ {J} g(O^ k_{j}) f(c^p_{ij})
\label{eq:accessibility}
\end{equation}

With functions for the opportunities that allow this equation \ref{eq:opportunities-function}:
\begin{equation}
g(O_j) = O_j
\label{eq:opportunities-function}
\end{equation}

\noindent as well as possibly agglomeration effects as in equation \ref{eq:agglomeration-effects}:

\begin{equation}
g(O_j) = O^\alpha_j
\label{eq:agglomeration-effects}
\end{equation}

This type of accessibility measure is based on Hansen's work in 1959, which quantifies accessibility by considering the number of reachable opportunities through an impedance function, also known as spatial decay. The impedance function decreases opportunities as travel costs increase, encompassing factors such as time, money, and other relevant variables. Various formulas, including negative power, exponential, lognormal, and log-logistic, can be employed to compute the impedance function (Geurs, Eck, et al., 2003; Hansen, 1959; Ingram, 1971; Reggiani, Bucci, \& Russo, 2011; D. S. Vale et al., 2016). In the formula, \(A_i\), \(O_j\), and \(C_{ij}\) represent matrices of accessibility opportunities and costs, respectively. Following the development of Levinson \& Wu (2020), the functions (g and f) discount opportunities and costs based on the diminishing value of opportunities concerning both number and cost. The matrix of weighted opportunities \(O_j\) can consider different activity types, times of day, and available modes. In contrast, the matrix of cost elements \(C_{ij}\) can incorporate total social costs or generalized internal costs, depending on the specific objectives (Levinson \& Wu, 2020).

The formulation of the decay function's cumulative opportunities (rectangular) can be expressed as equation \ref{eq:decay-functions-cumulative-opportunities}:
\begin{equation}
f(c^p_{ij}) = X
\label{eq:decay-functions-cumulative-opportunities}
\end{equation}

In addition, The widely used exponential decay formulation in gravity models is frequently given by equation \ref{eq:exponential-distance-decay}:

\begin{equation}
f(c^p_{ij}) = \exp(-\theta c^p_{ij})
\label{eq:exponential-distance-decay}
\end{equation}

Generally, proximity plays a significant role in determining preferences for opportunities, with closer options being preferred over more distant ones (Miller, 2004). Furthermore, studies have shown that fewer trips occur between distant locations than nearby ones, indicating that interaction is less likely in remote areas(Ingram, 1971). To calibrate travel impedance functions, researchers often utilize trip likelihood or trip rate data from travel surveys, allowing for a better understanding of interaction patterns(Geurs \& Ritsema van Eck, 2001; Ingram, 1971). Therefore, impedance functions effectively capture reduced interaction between different places, and trip rates can serve as valuable calibration parameters for the travel impedance function \(f(C_{ij})\).

Calculating individual-based accessibility in active modes (e.g., walking, cycling) can pose challenges due to various factors. The significant constraints involved in determining individual-based accessibility in active modes are as follows:

\begin{enumerate}
\def\labelenumi{\arabic{enumi}.}
\item
  Individual transportation preferences significantly influence accessibility. People have varying preferences for modes of transportation, with some preferring walking or cycling, while others opt for public transportation or driving. Different factors like distance, time, safety, and comfort contribute to these personal preferences.
\item
  Physical ability: Similar to place-based accessibility, individual-based accessibility in active modes is heavily influenced by personal physical ability. Individuals with disabilities or health issues may face difficulties or limitations when it comes to traveling on foot or by bike, thereby reducing their accessibility.
\item
  Time constraints: Time-saving considerations can impact accessibility as people may choose faster modes of transportation, such as driving or public transportation, which can diminish accessibility in active modes. It is essential to acknowledge that individual-based accessibility varies for each person due to factors like time of day, purpose, and individual constraints (Miller, 1991). Constraints arise from spatial and temporal characteristics that restrict an individual's activity choices (Chi, Porter, Cosby, \& Levinson, 2013; Miller, 1991; Miller \& Bridwell, 2009; Pred, 1977). Time constraints can be illustrated using a time-space diagram, which accounts for the available time and the transportation system's performance (Ilagcrstrand, 1970). The space-time prism, incorporating an individual's activity schedule, allows for measuring access to multiple activities and activity participation time based on behavioral rules (Miller, 1991). Accessibility is determined by the total number of opportunities within the space-time prism, encompassing all possible paths while considering time constraints (Tong, Zhou, \& Miller, 2015). Thus, although opportunities may be spatially accessible, the time available for individuals to reach and engage in these activities is limited. This perspective leads to a constraints-based or people-based accessibility measure (Y.-H. Wu \& Miller, 2001).
\item
  Weather conditions, similar to place-based accessibility, can affect individual-based accessibility in active modes. During extreme weather, individuals may be less inclined to walk or cycle, reducing their ability to choose these modes.
\item
  Infrastructure: The quality and availability of infrastructure, such as sidewalks, bike lanes, and pedestrian crossings, significantly influence individual-based accessibility in active modes. Poor infrastructure in certain areas can discourage people from choosing active modes of transportation, consequently reducing their accessibility options.
\end{enumerate}

Accessibility measures in active modes encompass various measures, including activity-based, distance-based, topological or infrastructure-based, utility-based, walkability, and bikeability. Activity-based measures, such as gravity-based and cumulative opportunities measures, employ the gravity model to assess accessibility by weighting opportunities based on a travel impedance function. These measures consider both the size of opportunities and the cost of traveling to them to evaluate the accessibility of a place. Distance-based measures focus on analyzing the proximity of facilities, including the distance to the nearest opportunity, the number of opportunities within a defined distance or time, the average distance to all opportunities, and the average distance to a specified number of closest opportunities. Infrastructure-based measures exclusively rely on street and transportation network features and are not sensitive to the spatial distribution of activities. Utility-based measures, also known as benefit measures, are derived from microeconomic random utility theory and describe accessibility as the outcome of rational decision-making among various destination transportation alternatives (Apparicio, Abdelmajid, Riva, \& Shearmur, 2008; Geurs \& Ritsema van Eck, 2001; Halden, Mcguigan, Nisbet, \& Mckinnon, 2000; Kwan, 1998). Walkability and bikeability measures assess the distribution of people, households, or jobs over a specific area or quantify the diversity of land uses, such as offices, housing, retail, entertainment, and services, within a given region (L. Frank, Engelke, \& Schmid, 2003; Leslie et al., 2007). Accurate accessibility measures for walking or cycling trips can aid transport planners in making informed decisions regarding infrastructure provision for non-motorized transportation (Devkota, Dudycha, \& Andrey, 2012; Iacono et al., 2010).

Moreover, calculating ATB accessibility in both approaches requires multiple data sets relating to travel behavior and land use. Unfortunately, this process has been hindered by the scarcity of suitable data (Iacono et al., 2010). Specifically, there is a lack of information regarding the spatial aspects of walking and cycling behavior, such as the origins and destinations of travel episodes, routes taken, and durations and distances involved. Consequently, researchers often rely on data obtained from local or national questionnaires and local maps to fulfill the required data requirements (Devkota et al., 2012; Iacono et al., 2010; Levine, 2010; Millward, Spinney, \& Scott, 2013; Yang \& Diez-Roux, 2012). Additionally, the available data tend to be highly location-specific or confined to a limited geographic area, thereby inadequately covering the broad scope of large-scale survey instruments, such as national transportation surveys (Achuthan, Titheridge, \& Mackett, 2007; Ulmer \& Hoel, 2003).

\hypertarget{methods-for-atb-accessibility-analysis}{%
\section{Methods for ATB accessibility analysis}\label{methods-for-atb-accessibility-analysis}}

Accessibility measures in active transportation modes can be categorized into two main types: location-based and individual-based. Location-based measures focus on assessing distances to opportunities from specific locations, while individual-based measures consider people's temporal and spatial constraints. In their study, Vale et al.~(2016) classified location-based accessibility measures into four primary groups. Firstly, activity-based measures encompass gravity-based measures, also known as attraction-accessibility or potential measures, as well as cumulative opportunities measures, which are also referred to as isochrones or contour measures. These measures have been extensively employed in studies investigating non-motorized accessibility (Iacono et al., 2010; Li et al., 2020; Lowry, Callister, Gresham, \& Moore, 2012; Millward et al., 2013; Prins et al., 2014). Secondly, topology infrastructure-based measures involve conducting topological analyses of the network (Hull, Silva, \& Bertolini, 2012; Lundberg, 2012). Thirdly, distance-based measures encompass analyzing the proximity of facilities (Apparicio et al., 2008; Sadler, Gilliland, \& Arku, 2011). Lastly, the fourth category comprises utility-based measures, also known as benefit measures (El-Geneidy \& Levinson, 2011; Geurs \& Van Wee, 2004; Hunt \& Abraham, 2007; D. de S. Vale, 2009).

One significant reason for the prevalent usage of location-based measures, as opposed to individual-based measures, in active transportation modes, is their potential compatibility with regional travel forecasting models. These models rely on data from various sources to predict travel patterns and demands across different regions. Employing location-based measures, such as coded networks, makes it easier to extract travel times and distances between different areas (Iacono et al., 2010; Saghapour, Moridpour, \& Thompson, 2017). However, there are limitations associated with using location-based measures for active travel modes:
1) Active travel modes demonstrate less sensitivity to travel times and levels of network congestion when compared to motorized modes.
2) Route choices for walking and cycling often involve qualitative, experiential, or challenging-to-measure factors (Hunt \& Abraham, 2007; Iacono et al., 2010; Tilahun, Levinson, \& Krizek, 2007).
3) Measuring active transport accessibility entirely relies on travel diary data.
4) Existing methods for evaluating cycling accessibility have not adequately addressed the accessibility of cycling destinations in terms of service area (Harkey, Reinfurt, Knuiman, Stewart, et al., 1998; Harkey, Reinfurt, \& Knuiman, 1998; Landis et al., 2003; Landis, Vattikuti, \& Brannick, 1997).
Some studies have examined the level of services provided by the bicycle network, such as the Bicycle Compatibility Index (BCI) or Bicycle Level of Service (BLOS). However, these measures primarily assess the performance of the bicycle network based on various geometric factors, including the width of bicycle routes, pavement conditions, route types, and connectivity. Nevertheless, other methods focus on bikeability in terms of the accessibility of different destinations for bicycles as a mode of transportation. These methods evaluate the potential for cycling using travel behavior data (Espada \& Luk, 2011; Milakis, Cervero, Van Wee, \& Maat, 2015; Rybarczyk \& Gallagher, 2014; Wahlgren \& Schantz, 2012).

\hypertarget{activity-based-measures}{%
\subsection{Activity-based measures}\label{activity-based-measures}}

Activity-based measures encompass two types: gravity-based measures (also known as Hansen-type measures) (Hansen, 1959)) and cumulative opportunities measures. Gravity-based measures also referred to as attraction-accessibility or potential measures, consider the number of opportunities weighted by the associated travel cost. These measures utilize a travel impedance function that assigns higher value to opportunities closer to proximity. Scholars have emphasized the importance of selecting an appropriate impedance function, resulting in the application of various functions such as power, negative exponential, logistic, and Gaussian functions (Iacono et al., 2010; Lowry et al., 2012; D. S. Vale \& Pereira, 2017; Vasconcelos \& Farias, 2012). In a recent study by Vale and Pereira (2017), twenty pedestrian accessibility measures were examined, and it was found that the modified Gaussian and exponential functions demonstrated the highest robustness for modeling walking accessibility. On the other hand, cumulative opportunity measures, also known as isochrones or contour measures, quantify the number of opportunities within a specified catchment area (D. S. Vale et al., 2016).

A gravity-based measure is derived from the denominator of the gravity model and is weighted by opportunities based on an impedance function. This function can be represented as equation \ref{eq:gravity-measures} (D. S. Vale et al., 2016):

\begin{equation}
A_i = \sum_{j = 1}^{J} W_{j} f(c_{ij})
\label{eq:gravity-measures}
\end{equation}

The equation \ref{eq:gravity-measures} provides the measure of accessibility from the origin location \(i\), where \(W_j\) represents the relevant opportunities found at \(j\), and \(f(c_{ij})\) denotes the cost of moving between \(i\) and \(j\). The function \(f()\) defines a kernel centered around location \(i\). Activity-based measures are valuable in complementary opportunities, such as jobs, services, people, and parks. In these scenarios, access to more opportunities and proximity (as modeled in gravity-based models) is advantageous.

Cumulative opportunities, also referred to as isochrones or contour measures, count the number of opportunities within a specified catchment area. In contrast to the utility measure, which assesses accessibility based on the ``net utility'' gained from travel, the cumulative measure quantifies accessibility by considering the total number of reachable opportunities and their associated travel costs. In contrast with gravity-based measures, in a cumulative measure (equation \ref{eq:gravity-measures}), the distance decay function is binary and calculated using equation \ref{eq:binary}:

\begin{equation}
f(c_{ij} \le \gamma _i) = X
\label{eq:binary}
\end{equation}

In the equation \ref{eq:binary}, \(\gamma\) represents the threshold value. According to this formula, all opportunities located within the \(\gamma\) threshold are considered accessible. Additionally, \(f(c)\) is an indicator function that yields a value of 1 if the logical statement in the function's argument is valid (i.e.~if the cost of reaching \(j\) from \(i\) does not exceed the bandwidth parameter value), and 0 otherwise. Alternative distance-decay functions (such as inverse distance or negative exponential) can generate smoother map patterns, as discussed by (OKelly \& Horner, 2003). However, they require more parameters and introduce distance- or cost-discounted schemes that are more challenging to interpret. Prior research has indicated that cumulative opportunity measures exhibit a high degree of correlation, irrespective of the distance-decay function employed (Kwan, 1998). Consequently, our preference lies with simplicity and interpretability when selecting our accessibility indicator.

\hypertarget{distance-based-measures}{%
\subsection{Distance-based measures}\label{distance-based-measures}}

Distance-based measures consider accessibility in terms of proximity, whether travel distance, time, or a generalized cost measure between locations. A distance measure analyzes the closest facilities using the following four criteria:
1. Distance to the nearest opportunity
2. The number of opportunities within a specified distance or time
3. The average distance to all opportunities
4. The average distance to a defined number of the closest opportunities (Apparicio et al., 2008). These measures are applicable when destinations such as hospitals, transit stops, and convenience stores can be seen as substitutes for each other. This assumption is based on the idea that individuals generally prefer accessing the nearest facility.

In distance-based accessibility measures, distance is considered as the travel impedance. Typically, four types of distances are employed: 1) Euclidean distance, which is commonly used for walkability assessments, especially in health studies (Apparicio et al., 2008), 2) Manhattan distance, 3) shortest network distance (Hochmair, 2015; Lundberg, 2012), and 4) shortest network time (Paez et al., 2012a; Pearce, Witten, \& Bartie, 2006). Furthermore, there are two approaches to measuring distance. The first method calculates the distance to the closest facility of each type, determining the distance from each zone centroid to the closest or the first n closest facilities (e.g., medical centers). The second method calculates the distance to all nearby facilities, based on the concept of floating catchment areas, which considers the closest facility irrespective of distance. It measures the distance from each zone center to the closest or the first closest facilities of different types (e.g., medical centers, shopping centers, and so on).

Distance to nearest location is calculated based on equation \ref{eq:distance-based-accessibility}:

\begin{equation}
A^{ip} = \min_{j \in L^p} (d_{ij})
\label{eq:distance-based-accessibility}
\end{equation}

In this equation \ref{eq:distance-based-accessibility}, \(A^{ip}\) is the accessibility of zone \(i\) to the location of type p, \(L^p\) is a set of locations of type p, and \((d_{ij})\) is distance (or travel time for a given mode) from i to location j in set \(L_p\). This measure is consistent with a straightforward location model in which the nearest location is always chosen with a probability of 1.0. (equation \ref{eq:distance-based-measure})

\begin{equation}
P^{ip}_j = X
\label{eq:distance-based-measure}
\end{equation}

In equation \ref{eq:distance-based-measure}, \(P_j^{ip}\) represents the probability of selecting location \(j\) for purpose \(p\), given that one is situated in zone \(i\). However, this measure suffers from two limitations. Firstly, it needs to account for the size or attractiveness of locations. Secondly, it does not investigate the cumulative impact of having access to multiple locations. Consequently, it is not advisable to employ this method as a standalone measure when calculating accessibility.

\hypertarget{topological-or-infrastructure-based-measures}{%
\subsection{Topological or infrastructure-based measures}\label{topological-or-infrastructure-based-measures}}

Topological-based measures consider accessibility in terms of the street network rather than access from origins to destinations. Topology measures may evaluate network connectivity, the quality of infrastructure within a catchment area, or some combination of connectivity and infrastructure quality. Indeed, this measure emphasizes infrastructure evaluation. Such approaches are applicable in the context of planning -- for example, in identifying priorities for development or identifying potential impacts of redevelopment.

Topological-based measures examine accessibility in terms of the street network rather than considering access from origins to destinations. These measures evaluate network connectivity, infrastructure quality within a defined catchment area, or a combination of connectivity and infrastructure quality. This approach primarily focuses on assessing infrastructure. Such methods find application in planning contexts, such as prioritizing development initiatives or identifying potential impacts of redevelopment.

There are three types of topological measures: The first group evaluates the level of service (LOS) within a floating catchment area (FCA)(Sisson, Lee, Burns, \& Tudor-Locke, 2006). The second type is similar to the first one. However, this one used a pre-defined spatial unit to evaluate LOS, and this is based on the segment instead of the point (Emery \& Crump, 2011; Horacek et al., 2012; Lowry et al., 2012). The third one is very different since traffic is not considered a relevant parameter (Hoedl, Titze, \& Oja, 2010; Jabbari, Fonseca, \& Ramos, 2021; Zielstra \& Hochmair, 2011). These measures are based on the evaluation of network segments' infrastructure characteristics and include variables such as the sidewalk or bike path availability, quality, and length, among others.

\hypertarget{utility-based-measures}{%
\subsection{Utility-based measures}\label{utility-based-measures}}

Utility-based measures assess accessibility by considering individual preferences and utilizing the log-sum of discrete choice models applied to destination choice analysis (M. Ben-Akiva \& Lerman, 2021). This approach, often referred to as the benefit measure, offers a more accurate representation of individual accessibility compared to location-based measures. Furthermore, individual-based accessibility provides more representative measurements than place-based accessibility and reflects the population's access. Aggregating accessibility through this method is a commonly used approach (Geurs et al., 2003; Wachs \& Kumagai, 1973). The utility-based measure can be calculated using two methods:

They are assuming that a decision-maker perceives the utility of a destination as \(U_j = V_j + \epsilon_j\), where \(\epsilon_j\) represents the individual's idiosyncratic deviation in terms of how they perceive the utility of alternative j relative to the population's average utility, \(V_j\). The person chooses the alternative that generates the maximum perceived utility, \(U_j\). Under commonly used assumptions, the probability that j is the alternative with the maximum utility and is thus chosen can be calculated using the multinomial logit (MNL) model (equation \ref{eq:utility-based-measure}) (Train, 2009):

\begin{equation}
P_j^{ip} = \frac{e^{V_j}}{\sum_{j' \in L^{ip}} e^{V_{j'}}} = \frac{e^{\beta z_j}}{\sum_{j' \in L^{ip}} e^{\beta z_{j'}}}
\label{eq:utility-based-measure}
\end{equation}

In equation \ref{eq:utility-based-measure}, \({V_j} = \beta{Z_j}\) represents the systematic utility of alternative \(j\), where \({Z_j}\) is a vector of explanatory variables, and \(\beta\) is a row vector of parameters.

The actual perceived maximum utility is unobservable. However, in the case of the MNL model, it has been demonstrated (M. E. Ben-Akiva, Lerman, Lerman, et al., 1985) that the expected maximum utility \((I^{ip})\) associated with this choice can be calculated using the following equation, denoted as equation \ref{eq:maximum-utility}:

\begin{equation}
I^{ip} = E\left[ \max_j (U_j) \right] = \ln\left(\sum_{j \in L^{ip}} e^{\beta z_j} \right)
\label{eq:maximum-utility}
\end{equation}

That is, it is the natural logarithm of the denominator of the logit choice model (sometimes referred to as the ``log sum'' term). Further, it can also be shown that this expected maximum utility is the consumer's surplus for this choice. Thus, it is a standard measure of economic benefit. Given this, Ben-Akiva and Lerman (1985) argue that it also provides a behaviorally and economically sound definition of accessibility: accessibility for a given activity is the expected utility derived from participation in this activity, which is the consumer surplus associated with it. That is equation \ref{eq:Accessibility-maximum-utility}:

\begin{equation}
A^{p}_i = \ln \left(\sum_{j \in L^{ip}} e^{\beta z_j} \right)
\label{eq:Accessibility-maximum-utility}
\end{equation}

In our analysis, Tables \ref{tab:table_1} and Tables \ref{tab:table_2} methodically categorize studies based on the various metrics employed to assess accessibility via active transportation modes, explicitly walking and cycling. This categorization highlights the varied methodological approaches adopted in the field. Our literature review indicates that a diverse array of measures has been applied in the context of cycling to analyze accessibility. These range from simple distance-based metrics to more complex models. Conversely, the majority of studies focusing on walking accessibility have predominantly utilized active-based measures. In addition, a limited number of studies have incorporated utility-based measures. These metrics generally evaluate the economic benefits and utilities derived from changes in the network, providing a different perspective on accessibility. However, a comparative analysis conducted by Vale et al.~(2016) illuminates a notable discrepancy in the research volume focused on cycling-specific accessibility relative to walking. Their comprehensive review, which encompasses 84 papers on walking and cycling accessibility published up to September 2013, reveals a significant skew towards walking-oriented studies. This imbalance underscores the need for more focused research on cycling accessibility to better understand and enhance this active transportation mode (D. S. Vale et al., 2016).

\begin{landscape}\begin{table}
\centering
\caption{\label{tab:unnamed-chunk-1}\label{tab:table_1}Cycling-Specific Accessibility Measures in Various Studies}
\centering
\resizebox{\ifdim\width>\linewidth\linewidth\else\width\fi}{!}{
\begin{tabular}[t]{c>{\centering\arraybackslash}p{5cm}>{\centering\arraybackslash}p{5cm}>{\centering\arraybackslash}p{5cm}>{\centering\arraybackslash}p{5cm}>{\centering\arraybackslash}p{5cm}cc}
\toprule
\multicolumn{1}{c}{\textbf{Metric Type}} & \multicolumn{1}{>{\centering\arraybackslash}p{5cm}}{\textbf{Study}} & \multicolumn{1}{>{\centering\arraybackslash}p{5cm}}{\textbf{Measure}} & \multicolumn{1}{>{\centering\arraybackslash}p{5cm}}{\textbf{Travel Time/Distance}} & \multicolumn{1}{>{\centering\arraybackslash}p{5cm}}{\textbf{Origins/Destinations}} & \multicolumn{1}{>{\centering\arraybackslash}p{5cm}}{\textbf{Geographic Scale}} & \multicolumn{1}{c}{\textbf{Travel Calculation}} & \multicolumn{1}{c}{\textbf{positive/normative implementations}}\\
\midrule
Activity-based & Murphy and Owen (2019) & Gap in Cumulative Access to Job Opportunities: A Comparative Analysis of LTS 1-4 Based on Weighted Accessibility by Worker Count & 20 mins (Tested 5 to 6 mins) & Census block centroids & Neighbourhoods -> city level & Network travel time 15 km/h & Normative\\
 & Faghih Imani, Miller et al. (2019) & Cumulative Access to Opportunities: Employment and Population Analysis Using Calculated Isochrones for LTS 1-4. & 30 mins & Dissemination area centroids & Dissemination areas & Network travel time & Normative\\
 & Wu, Lu et al. (2019) & Utilizing a Gravity-Based Approach: Evaluating Accessibility to Points of Interest at Metro Stations through a Lognormal Distance Decay Function (Validated by the Distribution of Bicycle-Metro Trip Data) & 2.5 Km & Metro stations (origins) POIs (destinations) & 2.5 km buffer & Euclidean distance & Normative\\
Distance-based & Houde, Apparicio et al. (2018) & Proximity to Cycling Infrastructure: 1. Network distance to the closest segment of the cycling network. 2. Network distance to exclusive bicycle pathways &  & Census tract centroids (origins) Bike paths (destinations) & Census tracts & Network distance & Positive\\
 & Pérez, Buck et al. (2017) & 1. Dedicated Cycling Network Accessibility: The distance from the centroid of each census tract to the closest segment of the bicycle network was evaluated to ascertain the accessibility level of the respective tracts. 2. Traffic Stress Assessment: The Level of Traffic Stress (LTS) network was employed to gauge accessibility within the district. &  & census tract centroid to nearest bike network & The District of Columbia & Euclidean distance & Positive\\
\addlinespace
Topology-based & Mekuria (2012) & Network Connectivity Assessed by LTS 1-4: This measures the percentage of trips and nodes connected within the network &  & Home-to-work O-D pairs from regional trip  Land parcel 'attraction strength' (size and land-use attraction) (destinations) & Census blocks & Network & \\
\bottomrule
\end{tabular}}
\end{table}
\end{landscape}

\begin{landscape}\begin{table}
\centering
\caption{\label{tab:unnamed-chunk-2}\label{tab:table_2}Walking-Specific Accessibility Measures in Various Studies}
\centering
\resizebox{\ifdim\width>\linewidth\linewidth\else\width\fi}{!}{
\begin{tabular}[t]{l>{\raggedright\arraybackslash}p{5cm}>{\raggedright\arraybackslash}p{5cm}>{\raggedright\arraybackslash}p{5cm}>{\raggedright\arraybackslash}p{5cm}>{\raggedright\arraybackslash}p{5cm}ll}
\toprule
\multicolumn{1}{c}{\textbf{Metric Type}} & \multicolumn{1}{>{\centering\arraybackslash}p{5cm}}{\textbf{Study}} & \multicolumn{1}{>{\centering\arraybackslash}p{5cm}}{\textbf{Measure}} & \multicolumn{1}{>{\centering\arraybackslash}p{5cm}}{\textbf{Travel Time/Distance}} & \multicolumn{1}{>{\centering\arraybackslash}p{5cm}}{\textbf{Origins/Destinations}} & \multicolumn{1}{>{\centering\arraybackslash}p{5cm}}{\textbf{Geographic Scale}} & \multicolumn{1}{c}{\textbf{Travel Calculation}} & \multicolumn{1}{c}{\textbf{positive/normative implementations}}\\
\midrule
Activity-based & Cheng, Caset et al. (2019) & Cumulative Accessibility Opportunities: Evaluation of chess/card rooms and urban parks catered to both senior and younger adults. The park areas were assessed and given weight according to their dimensions & Variable Distance Thresholds: These were determined by considering both geographical location and socio-economic factors & Locations derived from the travel survey (as origins) include parks and chess/card rooms (as destinations) & Traffic analysis zones & Network distance & Positive\\
 & eyes, Páez et al. (2014) & Cumulative Access Opportunities to Urban Parks: An assessment was conducted focusing on park spaces designed for children, with each park's weight determined by its size. The accessibility was evaluated in two primary ways: 1. Based on individual attributes extracted from the travel survey. 2. According to various scenario profiles, such as gender, age, income, and other demographic factors & Derived from a statistical model evaluating travel patterns & Locations sourced from the household travel survey are juxtaposed with rasterized parks, having dimensions of 25x25 meters, which serve as the destinations & Dissemination areas (weighted average) & Euclidean distance & Normative\\
 & García-Palomares, Gutiérrez et al. (2013) & Quality of Access Indicators for Metro Stations: The evaluation assessed the extent of the population served by these stations. Furthermore, accessibility was analyzed by distinct age groups, employing a distance decay function tailored to each age category & Distance thresholds of 1500m have been computed for various age categories & Metro stations serve as the origins, while populations are assessed at the transport-zone level & 1500m metro station catchment areas & Network distance & Normative\\
 & Papa, Carpentieri et al. (2018) & Contour-Based Accessibility Evaluation: This study employed a method to determine bus service catchment areas. The analysis incorporated the number of residents served, categorized by age groups. Furthermore, catchment areas were computed both considering and disregarding the network slope & Subject to the regularity of bus services & Bus stops & Hexagonal cells 50m & Network distance Walking speeds dependent on age & Normative\\
 &  &  &  &  &  &  \vphantom{1} & \\
\addlinespace
 &  &  &  &  &  &  & \\
Utility-based & Blecic, Ivan, et al &  &  &  &  &  & \\
2013 &  &  &  &  &  &  & \\
\bottomrule
\end{tabular}}
\end{table}
\end{landscape}

\hypertarget{walkability-measures}{%
\subsection{Walkability measures}\label{walkability-measures}}

Walkability indices can be defined based on both the social and physical environment, serving as predictive indicators of active travel and physical activity in accessing facilities. These indices measure the usability of the built environment for individuals who walk to various destinations and for different purposes (e.g., from a clear origin to a specific set of destinations) (Blevcic et al., 2015; Dovey \& Pafka, 2020; Saelens \& Handy, 2008; D. S. Vale et al., 2016). It is essential to note the distinction between gravity- and distance-based accessibility measures and the walkability index. While walkability considers area characteristics surrounding origins and destinations in its calculation, it still needs to consider route characteristics.

There are four main categories of walkability measures: Frank's Walkability Index, Walk Score, Objective Walkability Index (OWI), and Graz's Walkability Index. In Frank's index, the walkability score is determined by summing normalized scores across factors identified based on the concept of walkability. Residential density, land-use mix, retail floor area ratio, and intersection density are used to measure walkability. Grasser, Van Dyck, et al.~(2013) further enhanced Frank's index for assessing European cities by incorporating population density, household density, entropy index for land-use mix, and three-way intersection density, resulting in the Graz walkability index (Grasser, Van Dyck, Titze, \& Stronegger, 2013). Additionally, other theory-based methods, such as OWI, have been proposed. Weiss, Maantay, et al.~(2010) developed OWI, which includes street connectivity, land-use mix, pedestrian safety, neighborhood aesthetics, neighborhood safety, and neighborhood infrastructure(Weiss, Maantay, \& Fahs, 2010). In 2011, Duncan, Aldstadt et al.~(2011) introduced a Walk Score to measure the walkability of neighborhoods. Walk Score identifies eight types of walking attractors: errands, culture, grocery, park, dining and drinking, school, and shopping. Although Walk Score can be assessed for any location worldwide, validation is necessary for locations outside the US, Canada, Australia, and New Zealand due to incomplete geolocated data (Duncan, Aldstadt, Whalen, Melly, \& Gortmaker, 2011; Walk Score, 2020).

Most studies assessing walkability have utilized Frank's index and Walk Score (L. D. Frank et al., 2006, 2010; L. D. Frank, Schmid, Sallis, Chapman, \& Saelens, 2005; D. S. Vale et al., 2016). The cr
itical difference between these approaches lies in the methodology employed. Walk Score utilizes a gravity-based approach, where opportunities are weighted using a distance decay function, while the Walkability Index relies on a cumulative opportunities measure. Table \ref{tab:table_3} summarizes this categorization.

\begin{landscape}\begingroup\fontsize{8}{10}\selectfont

\begin{longtable}[t]{>{\raggedright\arraybackslash}p{3cm}>{\raggedright\arraybackslash}p{3cm}>{\raggedright\arraybackslash}p{3cm}>{\raggedright\arraybackslash}p{3cm}>{\raggedright\arraybackslash}p{3cm}>{\raggedright\arraybackslash}p{3cm}}
\caption{\label{tab:unnamed-chunk-3}\label{tab:table_3}Studies examining walkability}\\
\toprule
\multicolumn{1}{>{\centering\arraybackslash}p{3cm}}{\textbf{Methods}} & \multicolumn{1}{>{\centering\arraybackslash}p{3cm}}{\textbf{Author}} & \multicolumn{1}{>{\centering\arraybackslash}p{3cm}}{\textbf{Variables}} & \multicolumn{1}{>{\centering\arraybackslash}p{3cm}}{\textbf{Data}} & \multicolumn{1}{>{\centering\arraybackslash}p{3cm}}{\textbf{Descriptions}} & \multicolumn{1}{>{\centering\arraybackslash}p{3cm}}{\textbf{positive/normative implementations}}\\
\midrule
Frank's Walkability Index & Frank et al. (2005a,b), Frank et al., (2010) & Net residential/ Population density & Travel-related data have been gleaned from the Census Journey to Work, and this encompasses information pertinent to both specified regions. &  & Normative\\
 &  & Retail floor area ratio & Household Travel Survey Data census-based demographic data &  & \\
 &  & Intersection density &  &  \vphantom{1} & \\
 &  & Land use mix &  &  & \\
 & Manaugh and El-Geneidy (2011) & Net residential/ Population density & Information pertaining to retail entities, encompassing both shopping centers and educational institutions, was meticulously extracted from the comprehensive Dun and Bradstreet business database. Subsequently, this extracted information was synergistically amalgamated with a weighted intersection index to yield a more holistic and representative overview of the retail landscape in the studied regions & For each trip purpose, nine distinct models were developed, each employing a different walkability metric in every iteration (Walk Score, walk opportunities, the Walkability Index at four distinct scales, and three variations of sizes for the pedestrian shed connectivity measure) & Normative\\
\addlinespace
 &  & Retail floor area ratio & A comprehensive database, consisting of more than 100,000 postal code points, was employed, as procured from Walk Score, to ensure the meticulous analysis and accuracy of the assessed variables within the study & he walkability index was developed at four distinct scales, employing 400, 800, and 1200m network buffers, in addition to the application at the census tract level & \\
 &  & Intersection density & Data at the household level, along with characteristics pertaining to travel behavior, have been acquired from the 2003 Montréal Origin–Destination survey to facilitate a detailed and nuanced analysis of prevailing travel patterns and preferences & A simplistic gravity-based measure was employed to assign greater weight to proximate locations in comparison to those more distanced & \\
 &  & Land use mix & Census tract level demographic data derived from Statistics Canada &  & \\
 & Adhikari et al.,(2021) & residential density & Demographic characteristics of participants & The walkability index employs 1 km pedestrian (walksheds) mapping pedestrian-accessible routes surrounding each postal code centroid & Normative\\
 &  & The commercial floor-to-area ratio &  & Each walkshed is equivalent to roughly 10–15 minutes of walking duration, a time frame frequently employed to evaluate perceived closeness to amenities and services & \\
\addlinespace
 &  & Land-use mix &  &  & \\
 &  & Intersection density &  &  & \\
 & Azmi et al., (2013) & Mixed-use planning & Data was gathered by using the questionnaire survey & The selection of community facilities or services was contingent upon the availability of services provided within a radius of 400 meters, equivalent to approximately 5 minutes of walking & Normative\\
 &  & Density & The study inquires into the accessibility experienced by residents, specifically focusing on their ability to walk from their homes to community facilities or services located within walkable catchment areas, thereby assessing the proximity and availability of essential services and amenities in relation to residential locations & A total of thirteen community facilities and services were analyzed, encompassing a diverse range of essential and recreational amenities, including grocery stores or supermarkets, parks, and both indoor and outdoor recreational facilities & \\
 &  & Street connectivity & The respondents specified the estimated duration they presumed would be required to walk from their residences to the closest destination, thus providing insights into perceived proximity and accessibility to nearby locations or amenities &  & \\
\addlinespace
 & Liao et al., (2020) & The variables within this section are categorized into four distinct components: 1. Density, 2. Facilities, 3. Green Space, and 4. Land Use Mix & As a source of walking frequency data, the Dutch national travel survey & dThe density variables within this study are meticulously categorized into three distinctive types: population density, population density, intersection density, and business property density & Normative\\
 &  &  & Data pertaining to the neighborhood was utilized as a source for acquiring socio-demographic and physical neighborhood variables, serving as control variables in the analysis. These encompassed factors such as gender, age, income status, employment status, household composition, and migratory background, each contributing to a multifaceted understanding of the demographic and social structure of the respective neighborhoods & The variables related to facilities are classified into two main categories: firstly, the distance to the closest facility, calculated as the average distance from the central point of the respective neighborhood; secondly, the number of accessible facilities within a 1 km radius originating from the neighborhood's center & \\
 &  &  & All variables related to walkability were extracted from the Esri-open postcode plane, coupled with data sourced from the Central Bureau of Statistics (CBS) & Variables associated with green space consider both the cumulative area and the diversity of open green spaces and recreational areas. These variables account for the various forms of green and recreational spaces available, offering insights into the environmental and leisure components within the studied area & \\
 &  &  &  & Land Use Mix Variables: Distinct lower-level land-use variables were employed, each expressed as a percentage of the total land area attributed to that specific land-use & \\
 & Arellana et al., (2021) & Intersection density & latest household Origin Destination survey & Calculated potential accessibility & Normative\\
\addlinespace
 &  & Land use entropy score & Land use data, the location of commercial zones, the population, and the characteristics of the walking trips from each zone (TAZ) & This metric assesses accessibility to a spectrum of opportunities including shopping, employment, education, and institutions & \\
 &  & Population density &  &  & \\
 &  & Commercial density &  &  & \\
 & Ruiz-Padillo et al., (2018) & Public Security & Census tracts were used that including size and number of households & Every census tract was categorized based on three distinct variables: the rate of motorization, the density of commercial and service establishments, and the average slope & Normative\\
 &  & Traffic Safety &  &  & \\
\addlinespace
 &  & Convenience and attractiveness: Street connectivity, Destination's proximity (number of shops and services), Mix of uses proximity (number of shops and services), attractiveness &  &  & \\
 &  & Characters of the roots:Pavement Quality, Pavement width, Slope &  &  & \\
The Walk Score calculates a numerical value by evaluating the walking distance to amenities across nine distinct amenity categories & Duncan et al., (2011) & Walking distance to amenities & The Google AJAX Search Application Programming Interface (API) furnishes the requisite data utilized in determining the Walk Score &  & Normative\\
 &  & Intersection density metrics &  &  & \\
 &  & Average block length &  &  & \\
\addlinespace
The Graz Walkability Index is devised, building upon the foundational structure of the American City and Frank's walkability indices & Grasser et al. (2017) & Net residential/ Population density/ household density & The data pertaining to outcomes were extracted from a survey that is cross-sectional and representative in nature & The current research documents the findings associated with both the 1000m circular buffer and the 1500m street network buffer & Normative\\
 &  & Intersection density & Walking (for at least 10 min) and cycling (in the warm season) &  & \\
 &  & Land use mix (entropy index for land-use mix) &  &  & \\
\bottomrule
\end{longtable}
\endgroup{}
\end{landscape}

\hypertarget{bikeability-measures}{%
\subsection{Bikeability measures}\label{bikeability-measures}}

Bikeability can be defined in various ways, encompassing the ability of individuals to bike, the suitability of the urban landscape for biking, and the likelihood of choosing bicycles as a mode of transportation or leisure (Krizek, Handy, \& Forsyth, 2009; Nielsen \& Skov-Petersen, 2018; Winters, Brauer, Setton, \& Teschke, 2013). Notably, several scientists have described the bikeability index. In 2012, Lowry, Callister, et al.~explained it as a measure of the comfort and convenience of an entire bikeway network for accessing important destinations, highlighting its exclusive focus on bicycle travel (Lowry et al., 2012). Furthermore, Winters, Brauer, et al.~(2013) presented the Bikeability Index, which incorporates the three fundamental measures along with factors such as the length of bicycle routes, slope, and separation from car traffic. Each variable is assigned a score from 1 to 10, which is then summed to produce the final bikeability score (Winters et al., 2013).

The assessment of bikeability in an environment involves considering the following characteristics:

\begin{itemize}
\item
  The suitability of the townscape or infrastructure, including bicycle tracks, crossings, and parking facilities, as described by Lowry, Callister et al.~(2012) as ``bicycle suitability''(Lowry et al., 2012).
\item
  NThe delineation of neighborhoods based on Euclidean distance rather than network distance (Greenberg \& Renne, 2005; Nielsen \& Skov-Petersen, 2018)
\item
  Explicit polygon features created around specific trajectories of individual respondents, such as buffers or ellipsoids, or based on the topology of the transport network (L. D. Frank et al., 2017; Madsen, Schipperijn, Christiansen, Nielsen, \& Troelsen, 2014).
\item
  The presence of connected infrastructures as a functional component of entire towns and urban fabrics (Lowry et al., 2012). According to Lowry, Callister et al.~(2012), this covers the term ``Bikeability.'' In the following, Table \ref{tab:table_4} prepares some of the studies that used the bikeability index.
\end{itemize}

\begin{landscape}\begingroup\fontsize{8}{10}\selectfont

\begin{longtable}[t]{>{\raggedright\arraybackslash}p{3cm}>{\raggedright\arraybackslash}p{3cm}>{\raggedright\arraybackslash}p{3cm}>{\raggedright\arraybackslash}p{3cm}>{\raggedright\arraybackslash}p{3cm}>{\raggedright\arraybackslash}p{3cm}}
\caption{\label{tab:unnamed-chunk-4}\label{tab:table_4}Studies examining Bikeability}\\
\toprule
\multicolumn{1}{>{\centering\arraybackslash}p{3cm}}{\textbf{Author}} & \multicolumn{1}{>{\centering\arraybackslash}p{3cm}}{\textbf{Study\_area}} & \multicolumn{1}{>{\centering\arraybackslash}p{3cm}}{\textbf{Measure}} & \multicolumn{1}{>{\centering\arraybackslash}p{3cm}}{\textbf{Data}} & \multicolumn{1}{>{\centering\arraybackslash}p{3cm}}{\textbf{Descriptions}} & \multicolumn{1}{>{\centering\arraybackslash}p{3cm}}{\textbf{positive/normative implementations}}\\
\midrule
Lowry, Callister et al. (2012) & Moscow, Idaho & A bikeability assessment devised based on the widely adopted accessibility equation known as Hansen's model & AThe term (bikeway) encompasses any road or pathway that permits bicycle travel, irrespective of whether it incorporates a dedicated bike lane & The evaluation of bikeability encompasses the examination of the comfort and safety aspects of the entire bikeway network, specifically in terms of accessing crucial destinations & Positive\\
 &  & The objective of this computation is to identify the optimal routes connecting zone i with each respective destination j & The case study community endeavors to establish an all-encompassing bikeway network that efficiently interconnects all segments while encompassing vital destinations within the region & Impedance functions were estimated by utilizing a negative exponential function as the underlying mathematical model & \\
 &  &  & The primary bikeway network under scrutiny in this study is composed solely of arterial roads, collector roads, and shared-use paths, with no inclusion of other road categories for assessment & Bikeability was evaluated for each commercial destination within the study area & \\
Nielsen and Skov-Petersen (2018) & Denmark & This research provides a detailed micro-level analysis of factors influencing bikeability, encompassing aspects such as population density and accessibility, infrastructure provisions, as well as terrain characteristics & The cycling-related data employed in this study was sourced from the Danish National Travel Survey & The survey captures cycling data encompassing its use as the primary mode of transportation, as well as its role as a stage mode, such as connecting to public transport, and recreational cycling without a specific destination in mind & Positive\\
 &  & The assessment of accessibility in this study was computed based on the shortest network distance between the starting points of trips & This study provides a thorough road and path classification, dividing them into seven distinct categories: roads without dedicated bicycle infrastructure, roads featuring designated bicycle lanes, roads with dedicated bicycle paths protected by curbs and/or separating strips, fully segregated paths catering to both bicycles and pedestrians, fully segregated paths exclusively for pedestrians, roads inaccessible to pedestrians and bicycles, and roads with no public access & The average slope of the terrain within the specified distances (measured using Euclidean distance) was taken into account in this analysis & \\
\addlinespace
 &  &  & For each trip origin, we tallied the counts of residents, jobs, retail employment, schools, high schools, and higher education institutions located within radial distances of 1 km, 2 km, 3 km, 4 km, and 5 km. These counts were subsequently integrated into the travel survey dataset. &  & \\
McNeil (2011) & Portland, Oregon & The assessment of bike accessibility, commonly referred to as bikeability, within a neighborhood is carried out using a 20-minute neighborhood criterion specifically tailored for bicycles & The 2009 National Household Transportation Survey & This research primarily concentrated on utilitarian trips originating from homes, excluding any travel to and from workplace & Normative\\
 &  & Utilizing a scoring methodology to evaluate bikeability, this study introduces an innovative index for quantifying the degree of bike-friendliness within an area & Geocoded data concerning parks, educational institutions, libraries, and public transit infrastructure, including light rail stations and bus routes, were sourced from Metro's Regional Land Information System & Business addresses for various establishments including childcare providers, grocery stores, clothing stores, general goods stores, beauty services (such as salons and barbers), banks, mail services (such as post offices and private mail providers), laundries and cleaners, gyms, general entertainment venues (like bowling alleys and performance venues), drinking establishments, movie theaters, restaurants, coffee shops, snack shops, and religious organizations were collected and geocoded for analysis & \\
 &  &  & Information regarding business addresses related to various destination types was acquired from Reference USA, a data clearinghouse &  & \\
Saghapour (2017) & Melbourne, Australia & The Cycling Accessibility Index (CAI) was developed to quantify cycling accessibility within metropolitan areas. It takes into account travel distance as a factor affecting impedance and integrates cycling catchments within local areas & The database comprised urban centers, prominent buildings, landmarks, public spaces, community facilities, and indigenous sites. These sites were considered destinations and categorized into four distinct groups based on their associated activities & The speed of 16 km/h, as adopted from the Austroads network operation planning framework, is utilized in this study & Positive\\
\addlinespace
 &  & Using gravity-based measures of accessibility & A database consisting of Mesh Blocks from the 2011 Census was utilized, containing information on the total population and number of dwellings in each block & The median desirable travel time or distance was utilized in this study & \\
 &  & Network models are employed to define suitable cycling catchments and construct an Origin-Destination (O-D) cost matrix for the evaluation of cycling accessibility & Point of Interests (POIs) were identified and considered in the analysis includes Education Centres,health and Care Facilities, retails and Recreation Centres, and Community Services & 4 km buffers were computed around education centers and health and care facilities & \\
 &  &  &  & Service area analysis and OD-cost matrix analysis were conducted separately for each set of destinations in this study & \\
 &  &  &  & In this study, a median desirable travel time of 20 minutes was established for accessing community services & \\
\bottomrule
\end{longtable}
\endgroup{}
\end{landscape}

\hypertarget{a-framework-for-assessing-data-sources-and-needs}{%
\section{A framework for assessing data sources and needs}\label{a-framework-for-assessing-data-sources-and-needs}}

Calculating accessibility in active transportation mode requires multiple data sources. A comprehensive review of the literature revealed that various types of data, such as travel data (trips), user data (socio-economic and personal data), origin-destination data, cycling and walking network data, spatial data (boundary, land use, postal code), and additional data (such as traffic data, weather data, slope, level of traffic stress, impedance value, and speed), are essential for determining the accessibility of active transportation.

A specific type of data management system known as a data warehouse plays a vital role in integrating valuable information from diverse and heterogeneous operational data sources for decision-making purposes. It employs application tools, architectures, information services, and communication infrastructures. The collected data is consolidated into a centralized repository, referred to as a data warehouse, which enables direct querying, analysis, and the creation of logical data marts focused on specific organizational aspects (Kimball \& Ross, 2011). To ensure the reliability and usability of the data, it undergoes processing steps such as cleaning, homogenization, and integration within the data warehouse, following the Extraction/Transform/Loading (ETL) process (Romero \& Abello, 2010). The first phase of data warehousing involves addressing common challenges like inconsistent data, incompatible data formats, varying data granularity, and other issues prevalent in distributed heterogeneous information services (Zhuge, Garcia-Molina, \& Wiener, 1996). The second step of creating the data warehouse necessitates different methodologies from those used in operational information systems. In this study, the design approach follows the multidimensional paradigm, where the fact/dimension dichotomy distinguishes the multidimensional representation of data. This representation allows for a comprehensive understanding and evaluation of data from multiple angles or dimensions of analysis. The multidimensional model is particularly suitable for non-expert users, such as knowledge workers, who are the end-users of the data warehouse. The third phase requires leveraging techniques such as aggregate navigation (Gupta, Harinarayan, \& Quass, 1995), complex query optimization (Chaudhuri \& Shim, 1994), advanced indexing strategies (Lomet \& Salzberg, 1990), and user-friendly visual interfaces for online analytical Processing (OLAP) (Chaudhuri \& Dayal, 1997; Colliat, 1996) and data mining (Fayyad, Piatetsky-Shapiro, \& Smyth, 1996). These capabilities enhance the navigation of aggregated data, optimize query performance, and provide user-friendly interfaces for data analysis and exploration.

This study employs a user-centered methodology to support the design of a multidimensional data warehouse and the elicitation of end-user requirements. The methodology consists of three key steps:

\begin{itemize}
\item
  The initial step involves a comprehensive analysis of the data sources to identify the multidimensional information they contain. This analysis focuses on understanding the data that can be analyzed from a multidimensional perspective without considering specific requirements at this stage.
\item
  The second step proposes leveraging the identified multidimensional knowledge to facilitate the elicitation of user needs. By incorporating this knowledge, the analytical capabilities of the data sources are fully utilized while ensuring a balance between the requirements and the available data sources.
\item
  Once the requirements have been established, the next step involves automatically generating the conceptual schema of the data warehouse and extracting the multidimensional knowledge from the data sources.
\end{itemize}

These three steps enable a systematic and efficient approach to designing the data warehouse and aligning it with the requirements and analytical capabilities of the available data sources. Table \ref{tab:table_5} shows required data based on each measure. Table 5, referenced in the study, is crucial in illustrating the specific data requirements associated with each identified measure. This table provides a detailed overview of the data elements necessary for each aspect of the multidimensional analysis. It categorizes the required data based on various dimensions and measures, offering a clear and organized representation of the data needs. This comprehensive tabulation is instrumental for users and designers alike, serving as a guide for understanding the data dependencies and requirements integral to the data warehouse design process. By providing this level of detail, Table \ref{tab:table_5} ensures that users can readily identify the data components relevant to their specific analytical needs, thereby enhancing the overall usability and effectiveness of the data warehouse. This table, referenced in the study, is crucial in illustrating the specific data requirements associated with each identified measure. This table provides a detailed overview of the data elements necessary for each aspect of the multidimensional analysis. It categorizes the required data based on various dimensions and measures, offering a clear and organized representation of the data needs. This comprehensive tabulation is instrumental for users and designers alike, serving as a guide for understanding the data dependencies and requirements integral to the data warehouse design process. By providing this level of detail, Table \ref{tab:table_5} ensures that users can readily identify the data components relevant to their specific analytical needs, thereby enhancing the overall usability and effectiveness of the data warehouse.

\begin{landscape}\begingroup\fontsize{10}{12}\selectfont

\begin{longtable}[t]{>{\raggedright\arraybackslash}p{2.5cm}>{\raggedright\arraybackslash}p{2.5cm}>{\raggedright\arraybackslash}p{2.5cm}>{\raggedright\arraybackslash}p{2.5cm}>{\raggedright\arraybackslash}p{2.5cm}>{\raggedright\arraybackslash}p{2.5cm}>{\raggedright\arraybackslash}p{2.5cm}}
\caption{\label{tab:unnamed-chunk-5}\label{tab:table_5}Required data according to each active transportation accessibility of measure}\\
\toprule
\multicolumn{1}{>{\centering\arraybackslash}p{2.5cm}}{\textbf{Data/Methods}} & \multicolumn{1}{>{\centering\arraybackslash}p{2.5cm}}{\textbf{Travel data}} & \multicolumn{1}{>{\centering\arraybackslash}p{2.5cm}}{\textbf{Users Data}} & \multicolumn{1}{>{\centering\arraybackslash}p{2.5cm}}{\textbf{Origin-destination Data}} & \multicolumn{1}{>{\centering\arraybackslash}p{2.5cm}}{\textbf{Cycling and walking network}} & \multicolumn{1}{>{\centering\arraybackslash}p{2.5cm}}{\textbf{Spatial data}} & \multicolumn{1}{>{\centering\arraybackslash}p{2.5cm}}{\textbf{Additional data}}\\
\midrule
Activity-based & Travel data is usually obtained from surveys and includes information about each trip such as duration, start point, end point, origin and destination. & Travel behavior characteristics such as age, gender, income, can be considered if the data are available & The origin and destinations spatial data or a database of POIs that is obtained from local map. Indeed, a database of POIs consists of the location of all of the facilities such as home, workplace, parks, schools, groceries, etc. & Walking and cycling network data are required for calculating time or distances (using network analysis or nearest distance) that can be obtained from both the OpenStreetMap and the local government data portals. & Spatial data of statistical areas such as blocks, mesh, zones, areas, etc. This data set includes some information such as population, number of dwellings, employment data, etc. This dataset is required for calculating accessibility in each area. & Impedance functions are required that are usually estimated based on a negative exponential function and it is mostly based on the travel time.\\
 &  &  &  &  &  & Slope can be considered for calculating accessibility.\\
Distance-based & ----- & ----- & Required to calculating the shortest distance to nearest facilities. & Required to calculating nearest distance or time to facilities using network analysis. & Spatial data of census tracts. & \\
Infrastructure-based & ----- & ----- & ----- & Walking and cycling network data are required & Spatial data of essential services or origins/destinations are required. For example, grocery stores, hospitals, schools, bikeshare systems, etc. & \\
Utility-based & Travel data is required & Travel behavior characteristics are required such as age, gender, etc. & ----- & Walking and cycling network data are required & Spatial data of essential services & \\
\addlinespace
Walkability & Travel data is usually obtained from surveys and includes information about each trip such as duration, start point, end point, origin and destination & Travel behavior characteristics are required such as age, gender, income, car availability and etc. & The origin and destinations database is required & Walking and cycling network data are required & Spatial data of net residential/ Population density, Retail floor area, Intersection density, and Land use mix are required & \\
Bikeability & Travel data is required and includes information about each trip such as duration, start point, end point, origin and destination & - & The origin and destinations database is required & Walking and cycling network data are required for calculating time or distances (using network analysis or nearest distance) & Spatial data of statistical areas such as blocks, mesh, zones, areas, etc. is required for calculating accessibility in each area. & Impedance functions are required that are usually estimated based on a negative exponential function and it is mostly based on the travel time\\
\bottomrule
\end{longtable}
\endgroup{}
\end{landscape}

\hypertarget{assessing-data-availability-and-requirements-for-calculating-active-mode-accessibility}{%
\section{Assessing Data Availability and Requirements for Calculating Active Mode Accessibility}\label{assessing-data-availability-and-requirements-for-calculating-active-mode-accessibility}}

In this section of the thesis, we embark on a comprehensive exploration of the various data sources available in Canada pertinent to studying active mode accessibility. The focus is on delineating and explicating the diverse array of data repositories, ranging from governmental databases to local urban mobility surveys, which collectively provide the necessary empirical foundation for calculating accessibility in active transportation modes such as walking and cycling. This examination is crucial for understanding the breadth and depth of data available in the Canadian context, thereby enabling a more accurate and informed analysis of active mode accessibility. The following subsections will provide detailed descriptions of each data source, highlighting their specific characteristics, the type of data they contain, and their relevance and applicability to the field of active transportation research. Figure 1.1 presents a structured overview of the classification of required data. The figure delineates two primary categories, trip data, and spatial data, which are essential for comprehensive accessibility analysis.Under Trip Data, we identify subcategories such as Travel Data, Users Data, and Origin-Destination Data. Travel Data encompasses information related to the specifics of trips, such as duration and mode of transport. Users Data provides socio-economic attributes of individuals who use active transportation. At the same time, Origin-Destination Data captures trip start and end points, which is crucial for understanding travel patterns. Spatial Data is subdivided into Land Use/Land Cover Data, Cycling and Walking Network Data, and Point of Interest Data. Land Use/Land Cover Data offers insights into areas' physical and environmental context, influencing active travel behavior. Cycling and Walking Network Data relates to the infrastructure supporting active modes of transportation, including the location of paths, bike lanes, and connectivity. Point of Interest Data details the locations that attract trips, such as workplaces, schools, and retail centers, which are vital for calculating accessibility to different destinations.

The figure further specifies key datasets used in the analysis, such as the General Social Survey (GSS) Data and Household Surveys under Trip Data, and a range of tools and databases like the Origin-Destination Survey of different regions, Open Street Map (OSM), Can-BICS, and various municipal open-source data sets under Spatial Data. This visual classification in Figure 1.1 aids in understanding the intricate web of data required for active mode accessibility calculation and highlights the comprehensive approach taken in this study to encompass the multifaceted nature of active travel behavior and its context within urban settings.

\begin{figure}
\centering
\includegraphics[width=0.9\textwidth,height=0.6\textheight]{figure/ch02-Fig-01.jpg}
\caption{Classification of required data}
\end{figure}

\textbf{General social survey (GSS)}

The General Social Survey (GSS) is a \textbf{\emph{national}} survey conducted by Statistics Canada that collects information on Canadians' social trends and attitudes. The Survey has been conducted since 1985 and is conducted every two years, making it one of the longest-running surveys of its kind in Canada. This Survey (GSS) in Canada covers various topics related to social trends and attitudes. Some major topics covered by the Survey include Social well-being, Health, Education, Work, Family, Social networks, Crime and justice, and Time use. The Survey is designed to provide a snapshot of social trends and attitudes in Canada. It is used by researchers, policy-makers, and the general public to gain a better understanding of the social issues facing Canadians.

The Time Use Survey is a component of the General Social Survey (GSS), which has been conducted since 1986. This Survey is conducted every five years (1986, 1992, 1998, 2005, and 2015) and continues until 2020. However, the information related to the 2020 census has not been released yet. So, The most recent Survey was released in 2015. It is designed to provide information on how Canadians allocate their time to daily activities such as paid work, household chores, leisure activities, caregiving activities, and travel trips. Time Use Survey collects data on how individuals travel from one location to another during their daily activities. It includes modes of transportation such as walking, cycling, driving, and taking public transit. The Survey collects information on the start and end time of each trip, the mode of transportation used, the purpose of the trip (e.g., work, shopping, leisure), and the distance traveled. In addition, This dataset contains travel time data for people in many of the Census Metropolitan Areas (CMAs) and non-CMA areas all over Canada. CMAs include St.~John's, Halifax, Saint John, Montreal, Quebec City, Toronto, Ottawa, Hamilton, Winnipeg, Regina, Saskatoon, Calgary, Edmonton, and Vancouver. The non-CMA areas of each of the ten provinces were also grouped to form ten more strata. The Survey also collects information on the characteristics of individuals and their households, such as age, sex, education, employment status, family composition, and income.

Based on the categories of data mentioned above, this database includes the following data:

\begin{itemize}
\item
  Travel data (trips) Travel data provides information about the trips, including the mode of travel, duration of travel, and trip origins and destinations. This dataset contains 301 bicycle and 4236 walking trips in 2015. Each trip contains pumID, start time, end time, duration, origin, and destination.
\item
  User data (Socio-economic and personal data)
\end{itemize}

Demographic variables of pedestrian and cycling users, including age, gender, and the number of households, can be obtained from this Survey.

\begin{itemize}
\tightlist
\item
  Origin- destination data
\end{itemize}

In the GSS database, different travel destinations and locations are considered. Each location is identified with a specific code, as follows: home or on the property, someone else's home or property, work or school, in the neighborhood, Outdoors, Grocery store, other stores or mall, Library, museum or theatres, Sports center, field or arena, Restaurant, bar or club, Place of worship, medical, dental or another health clinic, and Elsewhere.

\textbf{Household travel survey}

One of the existing data sources that can be utilized for data needs is the household travel survey, which has been conducted in various regions throughout Canada. This survey provides access to relevant data about travel, users, and origin-destination, which are essential for calculating accessibility measures accurately. The household travel survey is a valuable resource for transportation planners, policymakers, and researchers who seek to comprehensively understand travel behavior patterns and the factors that influence them. In a household travel survey, several variables are collected, such as household levels, person level for each person, and trip level for each trip made by each household member.

The following is an overview of the categories of data that are obtained through the household travel surveys:

\begin{table}
\centering
\caption{\label{tab:unnamed-chunk-6}\label{tab:table_6}A Comprehensive Analysis of Variables in Household Travel Surveys}
\centering
\fontsize{10}{12}\selectfont
\begin{tabular}[t]{>{\raggedright\arraybackslash}p{5cm}>{\raggedright\arraybackslash}p{5cm}>{\raggedright\arraybackslash}p{5cm}}
\toprule
\multicolumn{1}{>{\centering\arraybackslash}p{5cm}}{\textbf{Household levels}} & \multicolumn{1}{>{\centering\arraybackslash}p{5cm}}{\textbf{Person level}} & \multicolumn{1}{>{\centering\arraybackslash}p{5cm}}{\textbf{Trip level}}\\
\midrule
Home location & Gender & Origin location\\
Dwelling type & Age & Destination location\\
Household size (\# people) & Driver’s licence & Trip departure time (start and end time)\\
Number of vehicles & Transit pass & Purpose of travel (destination activity)\\
Number of bicycles & Student status & Mode(s) of travel (up to 5)\\
\addlinespace
Household income & School level & Number of vehicle occupants (if driver or passenger)\\
Home parking (\#off-street spots) & School location & Employment status\\
 &  & Workplace location\\
 &  & Parking at work and school(free or pay)\\
 &  & Other occupational status\\
\addlinespace
 &  & Location of residence\\
 &  & Frequency of cycling, walking, and other active mode use\\
\bottomrule
\end{tabular}
\end{table}

\textbf{Transportation Tomorrow Survey (TTS)}

The Transportation Tomorrow Survey (TTS) is a comprehensive survey (a series of population-based cross-sectional travel surveys) that gathers information about how and where people travel. The survey aims to gather detailed information about the travel patterns of households and individuals. Since 1986, the Greater Toronto Area (GTA) has been implementing the Transportation Tomorrow Survey (TTS) program every five years, which has proven to be a highly effective way of collecting travel behavior data. The Transportation Tomorrow Survey (TTS) is part of an ongoing data collection program by the Transportation Information Steering Committee (TISC). The survey data (2016, 2011, 2006, 2001, 1996, 1991, and 1986) are currently under the care of the Data Management Group. This group is responsible for maintaining the TTS databases and making appropriate travel information available for any urban transportation study in the area.

In 1986, the survey covered the Greater Toronto and Hamilton Area (GTHA), which included the Municipality of Metropolitan Toronto and the Regional Municipalities of Durham, York, Peel, Halton, and Hamilton. In 1991, the survey area was expanded to include the municipalities adjacent to the GTHA boundary, known as the `fringe area.' The 1996 survey included all of the GTHA, plus several additional regions, including Peterborough County, the City of Peterborough, Victoria County, the Town of Orangeville, Simcoe County, the City of Barrie, Wellington County, the City of Guelph, Waterloo Region, and Niagara Region. In 2001, the survey area changed again, with the addition of Simcoe County and the City of Orillia and the exclusion of the Regional Municipality of Waterloo and Northumberland County. The 2006 survey area included the Regional Municipality of Waterloo, the City of Brantford, and Dufferin County, with interviews conducted in Brant County during interviewer training. The 2011 and 2016 TTS survey areas, including Brant County, were the same as in 2006.

TTS database includes \textbf{\emph{Household attributes}} (such as Regional municipality of household, geocode of household, UTM X and Y coordinate of the household, Type of dwelling unit, Number of persons in the household, Day of week trip data, Number of vehicles, Number of persons possessing a driver's licence in the household, Number of full or part time workers in the household, Number of full or part time students in the household, Number of household trips on trip day, and Household's total income), \textbf{\emph{Person attributes}} (such as person number within the household, age, gender, possession of a driver's licence, possession of a transit pass, employment status of the person, person's occupation type, student status of person, school codes (Starting from 2001), regional/Local municipality of person's usual place of work, UTM X and Y coordinate of person's usual place of work, geocode person's usual place of work, number of trips made by the individual on trip day, and number of trips made by individual on trip day with primary mode being public transit), \textbf{\emph{Trip attributes}} (such as trip number for persons in household, start time of the trip, Primary mode of the trip (public transit, bicycle and walking), purpose of the trip (home-based work (i.e.~home-to-work or work-to-home), home-based-school, home-based-discretionary, non-Home-based), origin purpose of the trip, Regional municipality of trip origin, planning district of trip origin, 2001 and 2006 traffic zone of trip origin, UTM X and Y coordinate of trip origin, Method used to geocode trip origin, destination purpose of the trip (second and subsequent school trips, daycare (not in 1986), entertainment (1986 only), facilitate passenger, home, linked trip (1991 only), marketing/ Shopping (not in 1991), second and subsequent work trips, first school trip of the day, first work trip of the day and others), regional municipality of trip destination, planning district of trip destination, 2001 and 2006 traffic zone of trip destination, UTM X and Y coordinate of trip destination, method used to geocode trip destination, straight line trip length in kilometres, manhattan distance trip length in kilometres).

The information collected through the TTS provides valuable insights into travel patterns in a given region. It can help transportation planners and policymakers better understand the needs of travelers and make informed decisions about transportation infrastructure and services. By analyzing the data collected through the TTS, planners can identify trends and patterns in travel behavior, such as the most popular modes of transportation and the busiest travel times, and use this information to develop more effective transportation policies and programs.

\textbf{Autorité régionale de transport métropolitain survey(ARTM)}

The Montreal OD (Origin-Destination) survey is a large-scale cross-sectional household travel survey conducted every five years since 1970 in the Montreal metropolitan area (1970, 1974, 1978, 1982,1987, 1993, 1998, 2003, 2008, 2013, and 2018). The survey covers 5\% of the population residing in the Montreal metropolitan area and is conducted by the Autorité Régionale de Transport Métropolitain (ARTM).

Its purpose is to provide an accurate picture of all the trips made by region residents during an average weekday for all modes of transportation, such as walking, cycling, bus, metro, train, and car. They are descriptive surveys that provide a statistical portrait of the different characteristics of people's trips. For each documented trip, the surveys identify the origin, destination, purpose, departure time, and all the different modes of transport used. Other socio-demographic variables are also collected. It is primarily a survey conducted through telephone interviews that aims to provide a general overview of all trips made by region residents, regardless of the mode of transport used. The 2018 OD survey included a web questionnaire component for the first time. In addition, this survey covers an increasingly larger territory that spans the entire metropolitan region, from the major cities of Montreal, Laval, and Longueuil to the north and south crowns.

The survey collects detailed information on each trip made by all individuals in \textbf{\emph{each surveyed household}} (such as home location, size, vehicle ownership, and number of cars), \textbf{each person in the household} (age, gender, income, education level, driving license ownership, principal occupation, public transit monthly pass ownership), and \textbf{each trip} made by each person of 5 years and older ( such as departure time, origin and destination locations, trip purpose, mode sequence, and others).

\textbf{\href{https://vancouver.ca/streets-transportation/annual-transportation-survey.aspx}{Vancouver panel survey (VTS)}}

The Vancouver Panel Survey is a longitudinal survey of households living in Vancouver, British Columbia, Canada (Vancouver's nine transportation zones). The survey is conducted annually, and its primary goal is to provide a comprehensive picture of travel behavior and trends over time, as well as the impact of policies and interventions on these trends. The primary component of this survey is a travel diary in which individuals record the details of the trips they make on an assigned weekday. This trip data is then compared to the previous panel surveys conducted from 2018 to 2019 (2008,2011, 2013, and 2019), allowing analysis of transportation trends.

The survey collects detailed information on personal and household characteristics, including age, gender, income, education, and employment status. It also gathers information on travel behavior, including trip purpose (such as work, school, recreational/social/entertainment, home, work business trip, shopping, personal business, restaurant, drop off/pick up, and drive someone ), mode of transportation (Auto, transit, cycling, walking, and other), travel distance, start time for each of their recorded trips, travel time, trip rates, VKT, origin-destination patterns (The daily origin and destination (O-D) patterns for Vancouver residents based on geocoded trip-end coordinates), average trip distances (Trip lengths were estimated using the distance matrix from TransLink's Regional Transportation Model's shortest distance assignment) and trip frequency.

One of the unique features of the survey is the collection of data on active transportation, such as walking and cycling. This data evaluates the effectiveness of policies and interventions promoting sustainable transportation modes.

In addition to travel behavior and personal characteristics, the survey collects information on attitudes and perceptions related to transportation and the built environment. This information is used to inform policies and interventions to improve transportation options and promote sustainable and healthy communities.

\textbf{\href{http://www.ncr-trans-rcn.ca/surveys/o-d-survey}{Origin- Destination survey of National Capital Region}}

The Origin-Destination (OD) survey of the National Capital Region (NCR) in Canada is a comprehensive travel survey that aims to gather detailed information on the travel behavior of residents in the region. This survey was conducted in 1986, 1995, 2005, and 2011 and is currently being conducted again in 2022. The Origin-Destination (O-D) Survey examines the ``who, where, why, when, and how'' of trips made by residents of the National Capital Region (NCR), resulting in extensive, up-to-date information on current daily trip patterns of area residents in rural areas, growing suburbs, mature neighborhoods and downtown areas alike. The survey is a joint project of the TRANS Committee, comprised of the National Capital Commission, the City of Ottawa, the City of Gatineau, the Ontario and Quebec Ministries of Transportation, and the transit agencies of Ottawa and Gatineau.

Total trip numbers account for the age five and older population for the 2011 O-D survey and the age 11 and older for the 2005 O-D survey. In addition, The survey did not capture commercial trips or trips generated outside of the National Capital Region. The survey is designed to capture all trips made by residents in the NCR, including the trip origin and destination (includes different districts in this region such as Ottawa Inner Area, Ottawa West, Merivale, Ottawa East, Alta Vista), purpose (such as home, work, school, shopping, leisure, personal and others(include visiting friends and family, health-related trips, and other), and pick up or drop off), time of travel, distance of travel, and mode of transportation. The survey covers various transportation modes, such as walking, cycling, public transit, and driving.

The survey also collects personal and household information such as age, gender, income, education, household size, driver's license, household vehicle availability, and employment status. These variables are used to understand how travel behavior varies by demographic group and to identify potential disparities in access to transportation.

\textbf{\href{https://www.crd.bc.ca/project/regional-transportation/origin-destination-household-travel}{Origin- Destination survey of Capital Regional District (CRD)}}

The Capital Regional District (CRD) conducted a comprehensive trip diary (origin-destination, or O-D) survey. The survey profiles residents' travel behavior. The profile will aid the CRD in its Regional Growth Strategy, the Regional Transportation Plan, and other ongoing sustainable planning initiatives. The 2017 survey updates surveys conducted in 2011, 2006, and 2001.

The 2017 study area of the survey consisted of all 13 incorporated municipalities in the CRD, the Juan de Fuca Electoral Area, and Salt Spring Island. Most of the reporting described below covers the 13 incorporated municipalities and the Juan de Fuca Electoral Area: this area corresponds to the area covered by the Regional Growth Strategy. It is defined as the ``Regional Planning Area'' (RPA). Households from Salt Spring Island were included in order to build a better picture of travel between these regions and the RPA and of the travel patterns of Salt Spring Island residents. The Southern Gulf Islands and the CVRD were omitted.

Demographic variables

The demographics of the RPA's residents are essential indicators of travel. The key factors are population (people make trips), households (members of households coordinate their trips, including household size, age, employment, and type of dwelling ), and the vehicle available to each household.

Travel variables

The survey collects information on trip origins and destinations, travel modes used, trip purposes (such as Work / work-related, Post-secondary school, School, Personal business, Recreation / social, Dining / Restaurant, Shopping, Pick-up/drop-off passengers, home, and others), and the time of day that trips are taken.

\textbf{Origin- Destination survey of Nanaimo City}

The City of Nanaimo recently initiated the Nanaimo Transportation Master Plan (NTMP) process. In preparation for the development of the NTMP, the City undertook a Pre-Plan Consultation Process in 2011, which included open houses and surveys to identify key themes and priorities to be considered during the development of the NTMP.

The Origin-Destination (OD) survey of Nanaimo City collects data on the daily travel patterns of residents and visitors. The survey covers a wide range of travel-related information, including the purpose of the trip (Commute, Exercise, Recreation, Shopping/Errands, and others), mode of transportation, trip distance, trip duration, and demographics of the travelers.

The survey provides detailed information on the different modes of transportation used by travelers, including driving, walking, cycling, and public transit. It also documents the frequency and duration of each mode of transportation, and the distance traveled.

In addition to travel-related data, the survey also collects information on demographic factors such as age, gender, education level, income, and employment status. This data provides insights into how different groups of people travel within the City and helps identify any disparities in transportation access and mobility.

\textbf{\href{https://www.edmonton.ca/transportation/traffic_reports/travel-pattern-analysis}{Edmonton and Region Household Travel Survey(ERHTS)}}

The Edmonton and Region Household Travel Survey (ERHTS) is a comprehensive travel survey conducted in the Edmonton Metropolitan Region, Alberta, Canada, in 1994, 2005, and 2015. The ERHTS was conducted using a combination of online and paper-based surveys, with participants randomly selected from over 46,000 residential addresses in the Edmonton region. The report describes the weekday travel patterns of the Edmonton Capital Region residents. It travels between the Region and the City of Edmonton. The survey collected data on travel behavior over a 24-hour weekday period, with participants being asked to report all trips taken, including their purpose, mode of transportation, and time of travel.

The survey aims to collect data on travel behavior patterns of households residing in the Region and demographic characteristics of the population. The travel variables collected in the survey include the number of trips made, trip duration, distance traveled (The length of trips taken is an indicator of the spatial characteristics of travel and the extent to which people are willing to or forced to travel to complete activities), mode of transportation (car driver, car passenger, walk, transit, school bus, bicycle, and other), travel purpose (such as Work, Post-Secondary, School, Shopping, Social / Recreation, Personal Business, Pick up/Drop off, and other), and travel time. Personal variables include age, gender, education level, employment status, and household income. In addition to travel and personal variables, the survey collected information on household characteristics, including the number of household members, vehicles owned, and housing type.

The survey found that the most common mode of transportation for all trip purposes was driving, with 78\% of all trips made by private vehicles. Walking was the second most common mode of transportation, accounting for 10\% of all trips. Public transportation was used for only 6\% of all trips, and cycling for 3\%. The survey also found that the average trip distance was 8.7 km, and the average travel time was 29 minutes.

\textbf{\href{https://www.calgary.ca/planning/transportation/surveys.html?redirect=/travelsurveys}{Calgary and Region Travel and Activity Survey (CARTAS)}}

The Calgary and Region Travel and Activity Survey (CARTAS) was conducted in 2012. Household travel surveys have been conducted approximately every ten years since 1964 and provide critical information to decision-makers on how travel behavior and influences change over time. The CARTAS study area includes The City of Calgary, the Municipal District of Foothills, Rockyview County, Wheatland County, and all the towns and villages within those boundaries, including Airdrie, Chestermere, Cochrane, High River, Okotoks, Nanton, and Strathmore. The survey's primary purpose is to collect information to update the Regional Transportation Model (RTM). However, these surveys offer a unique insight into the characteristics of travel in Calgary and the surrounding region.

The travel data collected in the survey include information on the frequency of travel, the length of trips, and the types of transportation used, such as walking, cycling, public transit, and private vehicles. The survey also collected information on the number of people traveling together, the trip's purpose, and the time of day the trip was made.

The personal variables collected in the survey include demographic information, such as age, gender, income, education level, and employment status. The survey also collected information on household characteristics, such as the number of people living in the household, the number of vehicles owned by the household, and the availability of alternative modes of transportation, such as bicycles and public transit.

The survey data were used to analyze travel patterns and trends, assess transportation infrastructure needs, and evaluate the effectiveness of transportation policies and programs. The data have been made available to the public. They can be used by researchers, policymakers, and other stakeholders to inform decision-making and planning related to transportation and urban development in Calgary and the surrounding areas.

\textbf{City of Saskatoon Household Travel Survey}

The 2013 Saskatoon Travel Survey is a comprehensive study of travel behavior in Saskatoon, Canada. The survey gathered information about residents' travel habits, including mode choice, trip purposes, and travel times.

The survey collected data from households within the city limits of Saskatoon through telephone interviews, asking questions about each individual's travel on a typical weekday and weekend day. In addition to travel data, personal characteristics such as age, gender, employment status, and household size were also collected.

The travel data collected in the survey includes trip purpose, mode of transportation, origin and destination, time of travel, and travel distance. The survey also collected data on the frequency of travel, such as the number of trips taken by each individual in a week and the types of destinations visited.

The survey also included a mode choice analysis to better understand the factors influencing people's travel choices, such as personal and trip characteristics.

The data collected from the survey has been used to inform transportation planning and policy in Saskatoon, as well as to understand deeply the travel behavior of residents in the city. The information has been used to identify areas where improvements to the transportation system could be made, such as increasing transit service or building new cycling infrastructure.

\textbf{Okanagan Travel Survey (OTS)}

The City of Kelowna, City of Vernon, Regional District of Central Okanagan, West Kelowna, Lake Country, Peachland, and Westbank First Nation, along with the BC Ministry of Transportation and Infrastructure, have collaborated to conduct the Okanagan Travel Survey (OTS). The smartTRIPS program, part of the Sustainable Transportation Partnership of the Central Okanagan (STPCO), supported the survey's execution. The OTS employs a methodology that conducts a household travel survey every five years in the Central Okanagan and City of Vernon region. This type of survey gathers information on the daily travel patterns of each five-year-old or older household member based on their travel on the previous day. This survey was conducted in 2007, 2013, and 2018. Like the objectives of the previous Okanagan Travel Surveys conducted in 2007 and 2013, the data collected through the 2018 OTS creates a repository of residents' travel habits. This data can serve as a foundation for developing policies and transportation plans in the Central Okanagan and The City of Vernon. Additionally, the 2018 OTS contributes to the larger aim of tracking regional travel trends and establishing a regional transportation demand model for the area.

The Okanagan Travel Survey conducted in 2018 recorded the journeys taken by individuals within a specific region during a regular working day that lasted 24 hours. The survey measured the number of personal trips made for different reasons from one location to another. These trips were taken at a specific time of day and made using any of the five available modes of transportation.

The 2018 survey was a household-based survey that collected demographic information on all household members and travel characteristics. Data is as follows:

\begin{itemize}
\item
  household data includes address, dwelling type, number of householders, number of vehicles available to householders, number of working bicycles available to householders, and Household Income.
\item
  Person-level data for each person in the household includes gender, age, driver's license, student status, school level, employment status, workplace location, and type of occupation.
\item
  Trip level data for each trip made by each household includes origin (Geocode origin XY coordinates), destination (Geocode destination XY coordinates), trip departure time, trip arrival time, trip purpose (or activity at destination location such as travel to Work, post-Secondary School, School, restaurant, recreation (gym, swimming, etc.), social outing / meet friends, shopping, personal business, pick up and drop off a passenger, home and others ), mode of travel (such as auto driver, auto passenger, public transit, school bus, bicycle, walked, motorcycle or moped/scooter), transit route(s), number of vehicle occupants, vehicle availability for trip (if not by automobile and household has vehicles), and additional information about trip (open-ended response
\end{itemize}

\textbf{Winnipeg Area Travel Survey (WATS)}

The Winnipeg Area Travel Survey (WATS)is a travel `origin-destination' survey conducted in autumn 2007. The survey covered the entire City of Winnipeg and the surrounding area within a 100-km radius of downtown Winnipeg. The 2007 WATS is the first all household-based trip survey ever done in Winnipeg. In combination with traffic counts, on-board transit ridership counts, and demographic and employment data from the Census of Canada and other sources, the 2007 survey provides both a reliable profile of current conditions and a means to measure trends in local travel.

Typical of origin-destination surveys, there are three categories: household, person, and trip data. As noted, trip data were collected only for household members 11 years and older. The 2007 survey was a travel survey that collected demographic information on all household members and travel characteristics. The dataset is as follows:

Household data consists of location, household size, number of vehicles, type of dwelling, and household income.

Person data includes age, gender, driver's license, occupation status (worker, student, retiree, etc.), usual place of work or school, long-term physical disability, labor force status, hours of work, and education.

Trip data includes origin, destinations, purpose of travel(such as work, work-related, school, shopping, social/recreational, restaurant, medical/dental visit, drive someone somewhere / go pick someone up, return home), mode(s) of travel (such as car driver, car passenger, Winnipeg Transit, intercity bus, other transit, private transportation service, school bus, water taxi3 / ferry, taxi, handi-Transit, bicycle, walk, and motorcycle / moped), departure time, arrival time, If transit: use of park and ride lot, line(s) used, transfer point(s).

\newpage

\textbf{The London Household Travel Survey}

In 2016, the London Household Travel Survey was conducted to offer an in-depth understanding of the travel patterns of individuals residing in the City of London and the nearby Census Metropolitan Area. Vital data concerning travel making, preferences, and attitudes was collected during the Survey to support the development of infrastructure and services for road users, public transport passengers, cyclists, and pedestrians. Earlier editions of this Survey were conducted in 1987, 2002, and 2009. This Survey collected data on various variables related to households, persons, and trips.

For households, the Survey collected data on the location of residence, the number of people living in the household, driver's license, age and gender, employment status, and income. It also gathered data on the type of dwelling, such as whether it was a single-family home or an apartment, and the availability of cars and bikes in the household.

For persons, the Survey collected data on their age, gender, employment status, education level, and income. It also asked about their travel behavior, such as how often they traveled, what modes of transportation they used, and their travel purposes.

For trip data, the Survey collected data on the location of origin and destination for each trip, the purpose of the trip (work, post-secondary school, school, shopping/recreational, other discretionary), the mode of transportation used (such as auto driver, auto passenger, transit, walk/cycle, and others), the time of day, the distance traveled, and the duration of the trip. It also gathered information on the origins and destinations of the trips, such as the home address and the location of the destination.

\textbf{Kingston Household Travel Survey}

The 2019 Kingston Household Travel Survey (KHTS) was undertaken with a random sample of households in the City of Kingston. Previous household travel surveys of Kingston residents were conducted in 2002 and 2008. The 2019 survey builds on the legacy of the previous surveys while expanding the depth of the data collected and providing more detailed reporting on travel patterns captured by the survey. The survey gathered information on household and demographic characteristics relevant to understanding travel patterns. It also captured detailed trip information for residents aged 5+ years, providing a snapshot of the 24-hour travel patterns throughout a typical fall weekday.

The survey's objectives were to gather information to assist the City of Kingston in transportation planning and promote sustainable modes of transportation, such as walking, cycling, and public transit---the survey aimed to identify travel patterns and behavior and opportunities to reduce the reliance on single-occupancy vehicles.

This survey collected information on various aspects of travel and socio-demographic characteristics, such as:

\begin{itemize}
\item
  Household data consists of age, gender, income, student number in a household, type of dwellings, access to automobile, access to a bicycle, transit Passes, and employment status.
\item
  Trip data includes frequency and purpose of trips (such as usual work, work-related, post-secondary school, attending K-12 school, shopping, personal business, restaurant, recreation, social, serving passengers, return home, and other), modes of transportation used (such as auto Driver, auto Passenger, Kingston transit, School Bus, Walk, Bicycle, and others), trip distances, and travel times.
\end{itemize}

\textbf{North Shore Transportation Survey}

The North Shore Transportation Survey (NSTS) 2019 is a biennial survey of North Shore residents that tracks key transportation metrics associated with residents' travel patterns. The Survey is an initiative of the City of North Vancouver (CNV), District of North Vancouver (DNV), and District of West Vancouver (DWV). This Survey collected data on various variables related to households, persons, and trips as follows:

\begin{itemize}
\item
  Participant Characteristics: describes the characteristics of North Shore residents and their households, as captured by the Survey, including age, gender, household, employment, health status, occupation, bike access, and vehicle access characteristics. These characteristics are captured to better understand travelers' needs, challenges, and patterns. The results are based on the survey sample with selected information from the 2016 census.
\item
  Daily trip characteristics: provides a snapshot of daily (24-hour) travel patterns from the trips reported by survey participants and includes location, trip demand, purpose (such as usual work, work-related, school, personal business, restaurant, recreation, social, serve passenger, return home and others ), mode share (Auto Driver, Auto Passenger, Transit, Walk, bicycle, and others), and distribution (include the trip origin and destinations).
\end{itemize}

The following table is an overview of data sources from various household travel surveys conducted across Canada, as explained before. The detailed breakdown by survey, year, and data type provides a comprehensive overview of the methodologies. It focuses on travel surveys across Canadian regions. It details the type of data collected in these surveys, categorized into three main sections: Household Data, Person Data, and Trip Data. The table lists multiple surveys, such as the General Social Survey (GSS), Transportation Tomorrow Survey (TTS), and Autorité régionale de transport métropolitain survey (ARTM), among others. For each survey, the table specifies the year or years of the survey and the kind of information collected. For instance, the GSS is conducted every five years, such as in 1986, 1992, 1998, 2005, 2010, and 2015 collected household data ( including dwelling type, household size, and household income), personal data (such as gender, age, student status, school level, and employment status), and trip data (including modes of travel, the purpose of travel, start time and end time). Each survey's data points vary, focusing on demographic information such as household size, income, and employment status. In contrast, others delve into trip specifics like start and end times, trip purposes, and travel modes.

\begin{landscape}\begingroup\fontsize{8}{10}\selectfont

\begin{longtable}[t]{>{\raggedright\arraybackslash}p{3cm}>{\raggedright\arraybackslash}p{3cm}>{\raggedright\arraybackslash}p{3cm}>{\raggedright\arraybackslash}p{3cm}>{\raggedright\arraybackslash}p{3cm}}
\caption{\label{tab:unnamed-chunk-7}\label{tab:table_7}Analysis of Data Sources in Various Surveys Across Canada}\\
\toprule
\multicolumn{1}{>{\centering\arraybackslash}p{3cm}}{\textbf{Survey}} & \multicolumn{1}{>{\centering\arraybackslash}p{3cm}}{\textbf{Year\_of\_Survey}} & \multicolumn{1}{>{\centering\arraybackslash}p{3cm}}{\textbf{Household\_Data}} & \multicolumn{1}{>{\centering\arraybackslash}p{3cm}}{\textbf{Personal\_Data}} & \multicolumn{1}{>{\centering\arraybackslash}p{3cm}}{\textbf{Trip\_Data}}\\
\midrule
General social survey (GSS) & 1986 & Dwelling type & Gender & Mode(s) of travel\\
 & 1992 & Household size & Age & Purpose of travel\\
 & 1998 & Household income & Student status & Start Time\\
 & 2005 &  & School level & End Time\\
 & 2015 &  & Employment status & \\
\addlinespace
Transportation Tomorrow Survey survey (TTS) & 1991 & Regional municipality of household & Person number within the household & Trip number for persons in household\\
 & 1996 & geocode of household & Age & Start Time\\
 & 2001 & Type of dwelling unit & Gender & End Time\\
 & 2006 & Number of persons in the household & Transit pass & Mode of the trip\\
 & 2011 & Number of vehicles & Employment status of the person & Purpose of the trip\\
\addlinespace
 & 2016 & driver’s licence in the household & Person’s occupation type & Geocode of origins and destinations\\
 &  & Household’s total income & Student status of person & \\
 &  &  & School codes & \\
Autorité régionale de transport métropolitain survey (ARTM) & 1970 & Home location & Age & Origins location\\
 & 1974 & Size of household & Gender & Destination location\\
\addlinespace
 & 1978 & Vehicle ownership & Income & Trip purpose\\
 & 1982 & number of cars in households & Education level & Mode of transportation\\
 & 1987 &  & Driving license ownership & Start Time\\
 & 1993 &  & Main occupation & End Time\\
 & 1998 &  & Public transit monthly pass ownership & \\
\addlinespace
 & 2003 &  &  & \\
 & 2008 &  &  & \\
 & 2013 &  &  & \\
 & 2018 &  &  & \\
Vancouver panel survey (VTS) & 2008 & Home location & Age & Trip purpose\\
\addlinespace
 & 2011 & Size of household & Gender & Mode of transportation\\
 & 2013 & Vehicle ownership & Income & Travel distance\\
 & 2019 &  & Education & Start Time\\
 &  &  & Employment & End Time\\
 &  &  &  & Geocoded origin and destination\\
\addlinespace
 &  &  &  & Trip frequency\\
Origin-Destination survey of National Capital Region & 1986 & Size of household & Age & Trip origion and destination\\
 & 1995 & household vehicle availability & Gender & purpose of trip\\
 & 2005 & vehicle ownership & Income & Time of travel\\
 & 2011 &  & Education & Mode of transportation\\
\addlinespace
 &  &  & driver’s licence & \\
 &  &  & Employment status & \\
Origin-Destination survey of Capital Regional District (CRD) & 2001 & Households size & Age & Trip origins and destinations\\
 & 2006 & Type of dwelling & Gender & travel modes\\
 & 2011 & vehicle available to each household & Employment & trip purposes\\
\addlinespace
 & 2016 &  &  & Start Time\\
 &  &  &  & End \vphantom{1} Time\\
Origin-Destination survey of Nanaimo City & 2011 & Household size & Age & Purpose of the trip\\
 &  & Type of Dwelling & Gender & Mode of transportation\\
 &  & Number of vehicle & Employment status & Trip distance\\
\addlinespace
 &  &  & Income & Trip duration\\
 &  &  & Education & \\
Edmonton and Region Household Travel Survey (ERHTS) & 1994 & Household income & Age & Number of trips\\
 & 2005 & Number of household members & Gender & Trip duration\\
 & 2015 & Type of dwelling & Employment status & Distance traveled\\
\addlinespace
 &  & Number of vehicle & Education & Mode of transportation\\
 &  &  &  & Travel purpose\\
 &  &  &  & Travel time\\
Calgary and Region Travel and Activity Survey (CARTAS) & 2012 & number of people in the household & Age & Frequency of travel\\
 &  & number of vehicles & Gender & Length of trips\\
\addlinespace
 &  & Household income & Income & Mode of transportation\\
 &  & Number of vehicle & Education & Trip purpose\\
 &  &  & Employment & Start Time\\
 &  &  &  & End Time\\
City of Saskatoon Household Travel Survey & 2013 & Household size & Age & Trip purpose\\
\addlinespace
 &  & Household income & Gender & Mode of transportation\\
 &  & Number of vehicle & Employment status & Origin and destination\\
 &  &  &  & Time of travel\\
 &  &  &  & Travel distance\\
Okanagan Travel Survey (OTS) & 2007 & Location & Age & Geocode origin and destination coordinates\\
\addlinespace
 & 2013 & Dwelling type & Gender & Trip departure time\\
 & 2018 & Number of householdes & Driver’s licence & Trip arrival time\\
 &  & Number of vehicle & Student status & Trip purpose\\
 &  & Number of bicycle & School level & Mode of travel\\
 &  & Household Income & Employment level & \\
\addlinespace
 &  &  & Workplace location & \\
 &  &  & Type of occupation & \\
Winnipeg Area Travel Survey (WATS) & 2007 & household size & Age & Origin and destinations\\
 &  & Number of vehicles & Gender & Purpose of travel\\
 &  & Type of dwelling & Driver’s licence & Mode(s) of travel\\
\addlinespace
 &  & Household income & Occupation status & Departure time\\
 &  &  & Usual place of work or school & Arrival time\\
 &  &  & Labor force status & \\
The London Household Travel Survey & 1987 & Location of residence & Age & Modes of transportation\\
 & 2002 & Number of people living in the household & Gender & Travel purposes\\
\addlinespace
 & 2009 & Type of dwelling & Driver’s licence & Start Time\\
 & 2016 & Availability of cars & Employment status & End Time\\
 &  & Availability of bicycle & Income & Location of origin and destination\\
 &  &  &  & Distance traveled\\
Kingston Household Travel Survey & 2002 & Type of dwellings & Age & Frequency of trip\\
\addlinespace
 & 2008 & student number in a household & Gender & Purpose of trips\\
 & 2019 & Access to car & Employment status & Modes of transportation\\
 &  & Access to bicycle & Income & Trip distances\\
 &  & Transit Passes &  & Travel times\\
North Shore Transportation Survey (NSTS) & 2019 & Household employment & Age & Location of origin and destination\\
\addlinespace
 &  & health status & Gender & Trip demand\\
 &  & bike access &  & Purpose of travel\\
 &  & vehicle access &  & Mode of travel\\
 &  &  &  & Origin and destination\\
 &  &  &  & Occupation\\
\bottomrule
\end{longtable}
\endgroup{}
\end{landscape}

\textbf{The Linkable Open Data Environment (LOD)}

The Linkable Open Data Environment (LODE) is an exploratory initiative that aims to enhance the use and harmonization of open microdata primarily from municipal, provincial, and federal sources. It was compiled by the Centre for Special Business Projects (CSBP) at Statistics Canada in 2020.

This database includes variables such as address, postal code, city, province, and latitude and longitude of each facility and includes a Canada-wide Open Database of educational facilities (this database covers facilities such as \textbf{\emph{early childhood education, kindergarten, elementary, secondary, and post-secondary institutions}}, and \textbf{\emph{specific vocational training centers}}. The database does not include virtual educational institutions.) healthcare facilities (including ambulatory healthcare services, hospitals, and nursing and residential care facilities), cultural and art facilities (such as arts or cultural centers, artists, festival sites, galleries, heritage or historic sites, library or archive, museum, theatre/performance and concert hall, and miscellaneous), and recreational and sports facilities (including trails(such as urban and rural trails or pathways for walking, hiking, or biking), sports fields, arenas (facilities where sports and recreational activities take place), athletic parks, beaches, casinos, community centers, gyms, marinas, parks and green spaces, playgrounds, pools, race tracks, ice rinks, skate parks, splash pads, stadiums, miscellaneous), and Businesses (this database contains addresses of business, name, type of business and locations).

\begin{itemize}
\tightlist
\item
  \href{https://www.statcan.gc.ca/en/lode/databases/odb}{The Open Database of Buildings}
\end{itemize}

The inputs for the ODB are datasets provided by municipal, regional, or provincial sources available to the general public through open government portals under various types of open data licenses. The current database version (version 2.0) contains approximately 4.4 million records and includes provinces and territories where open building footprints were found during the collection period. Each data provider attached a different set of variables to their building footprints within the original datasets. The variables included in the ODB are as follows: Latitude, Longitude, Area, Perimeter, Data provider, Census subdivision unique identifier, Census subdivision, name, and Unique building ID.

\begin{itemize}
\tightlist
\item
  \href{https://www.statcan.gc.ca/en/lode/databases/odef}{The Open Database of Educational Facilities}
\end{itemize}

The Open Database of Educational Facilities (ODEF) is a collection of open data containing the names, types, and locations of education facilities across Canada. It is made available under the Open Government Licence - Canada. The inputs for the ODEF are primarily datasets provided by municipal, regional, or provincial sources available to the general public through open government portals under various types of open data licenses or otherwise published on their web pages and released under an open license with their permission. The variables included in the ODEF are as follows: Facility Name, Facility Type, Authority Name, International Standard Classification of Education (ISCED), Level, Official Language Minority School Status, Address, Unit, Street Number, Street Name, Municipality Name, Province, Postal Code, Province Unique Identifier, Census Subdivision Name, Census Subdivision Unique Identifier, Census Metropolitan Area Name, Census Metropolitan Area Unique Identifier, Longitude, Latitude, Geocoding Source, Source ID, Unique ID.

\begin{itemize}
\tightlist
\item
  \href{https://www.statcan.gc.ca/en/lode/databases/odhf}{The Open Database of Healthcare Facilities}
\end{itemize}

The inputs for the ODHF are datasets whose sources include regional health authorities, provincial, territorial, and municipal governments, and public healthcare and professional healthcare bodies. These datasets were available under various types of open data licenses, e.g., in an open government portal or as publicly available data. In some instances, data were obtained directly from administrative sources. Details of the sources used are available in the ODHF metadata. This dataset includes ambulatory healthcare services, hospitals, and nursing and residential care facilities. The variables included in the ODHF are as follows: Index, Facility Name, Source Facility, Type, ODHF Facility Type, Provider, Unit, Street Number, Street Name, Postal Code, City, Province or Territory, Source-Format Street Address, Census Subdivision Name, Census Subdivision Unique Identifier, Province or Territory Unique Identifier, Latitude, Longitude.

\begin{itemize}
\tightlist
\item
  \href{https://www.statcan.gc.ca/en/lode/databases/odcaf}{The Open Database of Cultural and Art Facilities}
\end{itemize}

The Open Database of Cultural and Art Facilities (ODCAF) collects open data containing cultural and art facilities' names, types, and locations across Canada. This dataset includes arts or cultural centers, artists, festivals, galleries, heritage or historic sites, libraries or archives, museums, theatre/performance and concert halls, and miscellaneous.

\begin{itemize}
\tightlist
\item
  \href{https://www.statcan.gc.ca/en/lode/databases/odrsf}{The Open Database of Recreational and Sport Facilities}
\end{itemize}

The Open Database of Recreational and Sports Facilities (ODRSF) is a collection of open data containing the names, types, and locations of recreational and sports facilities across Canada. Recreational and sports facilities include trails(such as urban and rural trails or pathways for walking, hiking, or biking), sports fields, arenas (facilities where sports and recreational activities take place), athletic parks, beaches, casinos, community centers, gyms, marinas, parks and green spaces, playgrounds, pools, race tracks, ice rinks, skate parks, splash pads, stadiums, miscellaneous).

\begin{itemize}
\tightlist
\item
  The Open Database of Businesses
\end{itemize}

The Open Database of Businesses contains addresses of business names and locations. It also includes information on the type of business and legal nature of business when supplied by the data providers.

\begin{itemize}
\tightlist
\item
  The Open Database of Infrastructures
\end{itemize}

The Open Database of Infrastructure contains the name and location of significant transport and physical infrastructures such as airports, railway stations, bridges, elevated roads, etc. Infrastructures are classified by type. Records are compiled from open sources and publicly available data (with permission from the data owners).

\newpage

\textbf{Can-BICS}

\begin{itemize}
\tightlist
\item
  Cycling and walking networks
\end{itemize}

Can-BICS is a classification system of five broad bicycle facilities assigned to three categories: high, medium, and low comfort, based on the facility's contribution to user safety and comfort while cycling. 1) High comfort includes low-stress routes that are comfortable for most people, including those of all ages and abilities, with a record for best safety, for example, cycle tracks, local street bikeways, and bike paths. 2) medium comfort is low- or medium-stress routes that are comfortable for some people but whose safety requires careful design, such as multi-use paths (A two-way paved path shared by cyclists, pedestrians, and other users). 3) Low-comfort bikeways are high-stress routes comfortable for a few people, with little or no additional safety, compared to no bicycle facility, such as painted bike lanes designated by bicycle and diamond pavement markings and signs exclusively for cyclists. Moreover, 4) non-conforming bicycle facilities do not meet minimum Can-BICS standards, such as non-conforming - trails (these are multi-use trails with unpaved surfaces), non-conforming -- major roads (shared lanes on significant roads provide connectivity), and non-conforming - other.

\textbf{OpenStreetMap (OSM)}

Another source for obtaining cycling and walking networks is OpenStreetMap (OSM). This dataset is a collaborative global map used for active transportation research. OSM considered cycle lanes, tracks, and sidewalks. A cycle lane lies within the roadway (on-road), whereas a cycle track is separate from the road (off-road). Tracks are typically separated from the road by, e.g., curbs, parking lots, grass verges, trees, etc., as well, trails line that indicates the paths or routes suitable for walking, hiking, bicycling, and other outdoor activities from 2015 to 2019 can be obtained from scholars Geoportal.

\textbf{Municipal open data}

Municipal open data is a standard source of bicycling infrastructure data, and city governments are making this spatial data for bicycling infrastructure. Some cities in Canada, such as Toronto, Montreal, Vancouver, etc., have this dataset. However, open data of different cities use different definitions for bicycling infrastructure, and they may have different levels of timeliness, completeness, and documentation (Schoner \& Levinson, 2014). For example, for bicycle facilities, the City of Toronto Open Data portal consists of a high-resolution geospatial data set with attributes accumulated from several sources of cycle tracks or bike lanes, road classification (local, collector, minor arterial, etc.), number of lanes, directions, stop signs and signalized intersections. (City of Toronto, 2017).

\textbf{\href{https://open.toronto.ca/catalogue/?search=point\&sort=score\%20desc}{City of Toronto}}

\begin{itemize}
\tightlist
\item
  Cycling and pedestrian Networks
\end{itemize}

The Toronto bikeways dataset illustrates the city's existing cycling network, including shared and dedicated bikeways. The Toronto bikeways dataset contains the following types of bikeways: cycle tracks, bicycle lanes (including buffered bike lanes and contra-flow bike lanes), neighborhood routes with sharrows (including wayfinding sharrows), multi-use trails (including off-road and in-boulevard), as well as signed cycling routes.

The Pedestrian Network (pednet) was created by the DAV team at the City of Toronto, and it is based on the sidewalk inventory from Transportation Services, Toronto road centrelines, and manual collection from aerial imagery. Pednet is integrated with centerline intersections, traffic signals, pedestrian crosswalks and crossovers, traffic signal data from Transportation Services, as well as other City of Toronto datasets.

\begin{itemize}
\tightlist
\item
  Points of Interest (POI) :
\end{itemize}

Cultural Hotspot: The Cultural Hotspot covers various locations and diverse neighborhoods, each with distinct characteristics. This dataset provides information about attractions that interest locals and tourists, including public art installations, murals, historically and architecturally significant buildings, parks, restaurants, and other noteworthy establishments.

School Locations - All Types: The present dataset encompasses a geographical file specifically designed for the City of Toronto, encompassing the entirety of the points denoting the locations of schools. Furthermore, the dataset includes information regarding private schools.

Green Spaces: This dataset includes parks and other open spaces within the City of Toronto.

\textbf{Land use and Land cover}

Property Boundaries: This data is a GIS file that outlines the geographical area of all properties in the City of Toronto.

Park and Recreation Facility: This dataset provides information on the recreation facilities administered by Toronto's Parks, Forestry, and Recreation Division.

Green Spaces includes parks and other open spaces within the City of Toronto.

\textbf{\href{https://open.hamilton.ca/}{City of Hamilton}}

\textbf{\emph{Cycling and pedestrian Networks}}

\begin{itemize}
\tightlist
\item
  Recreational Pathways
\end{itemize}

A pathway is defined as land dedicated to pathway use, the pathway is mapped, signage exists, and has a recreational purpose. Pathways may support a range of non-motorized recreational uses such as walking, hiking, and bicycling. It includes the length (kilometers) of all paved and unpaved maintained recreational pathways that fall under municipal responsibility or control and if the municipality incurs costs to maintain those pathways. This data excludes non-developed footpaths, sidewalks, and pathways that link to roadways and does not include unpaved trails maintained by others, such as the Bruce Trail and similar trail systems.

\begin{itemize}
\tightlist
\item
  Trails This dataset includes the location of recreational trails owned or maintained by the City of Hamilton, including walking/hiking trails and off-street bikeways. Types include:
\end{itemize}

Multi-Use Path Paved Multi-Use Recreational Trail Unpaved (Stone) Multi-Use Recreational Trail Stairs (with or without Bicycle troughs) Trail/Cycling Networks (Trans Canada, Waterfront, others)

\begin{itemize}
\item
  Road Sidewalk This dataset includes the edge of the sidewalks in Hamilton, and it captured from the 2019 aerial photography
\item
  Bikeways The dataset encompasses the currently available on-street bikeways in Hamilton, which incorporates proposed enhancements and infrastructural linkages to neighboring municipalities. The bikeways include Designated Bike Lanes, Bicycle Paths, Bike Routes (cautionary un-signed and signed on-street), and Paved Shoulders.
\end{itemize}

\textbf{\emph{Land use and Land cover}}

\begin{itemize}
\tightlist
\item
  Housing Placements
\end{itemize}

Housing Placements depicted are the initial occurrence that housing has been secured for unique individuals participating in a City-funded homelessness program: Intensive Case Management (ICM), Rapid Re-housing (RRH), and Transitional Living Program (TLP).

\textbf{\emph{Points of Interest (POI)}}

\begin{itemize}
\item
  Hospitals: includes the location of hospitals in Hamilton (2022).
\item
  libraries: This dataset showcases the locations of libraries in Hamilton. The latest version of this dataset was released in 2022.
\item
  Park Sports Fields: The dataset comprises a comprehensive inventory of sports fields in Hamilton. It encompasses many sporting facilities, including soccer fields, baseball diamonds, tennis courts, and other venues.
\item
  Places of Worship: Location of buildings used for religious congregations.
\item
  Recreation and Community Centers contain the location of recreation and community centers in Hamilton.
\item
  Tourism Points of Interest includes the location of points of interest in Hamilton.
\item
  Beaches: This data contains the location of the City of Hamilton's public recreational swimming beaches.
\item
  Educational Institutions: This dataset provides information regarding the geographic placement of all educational institutions and schools situated within the boundaries of Hamilton. The categorization encompasses various educational levels, including Elementary, Middle, Secondary, Post Secondary, Alternative Education, and Adult Learning establishments.
\end{itemize}

\textbf{\emph{Census Population, Age and Gender}}

In the 2021 Census, Statistics Canada introduced the concept of gender. Given that the non-binary population is small, data aggregation to a two-category gender variable was necessary to protect the responses' confidentiality. In these cases, individuals in the category ``non-binary persons'' are distributed into the other two gender categories and are denoted by the ``+'' symbol.

Data is derived from custom tabulations of Statistic Canada's Census obtained by the City of Hamilton as a consortium member of the Canadian Community Economic Development Network (CCEDNet) Community Data Program.

\textbf{\href{https://opendata.vancouver.ca/explore}{City of Vancouver}}

\textbf{\emph{Cycling and pedestrian Networks}}

\textbf{\emph{Bikeways}}

This dataset contains information about bikeways in the City of Vancouver. These bikeways follow street centrelines, so their placement in the street right of way is approximate. This dataset is maintained manually. This dataset includes data on shorter bikeway segments, which can differ from how the bikeways are displayed in the Vancouver Cycling Map.

\textbf{\emph{Land use and Land cover}}

\begin{itemize}
\tightlist
\item
  Property addresses
\end{itemize}

The property addresses dataset contains addresses used for parcel polygon display. Please note: These addresses are the primary addresses displayed in VanMap and do not represent a complete set of all addresses. Some addresses are duplicated because they appear more than once on VanMap.

\begin{itemize}
\tightlist
\item
  Property parcel polygons
\end{itemize}

The property parcel polygons are assessment-based land polygons.

\textbf{\emph{Points of Interest (POI)}}

\begin{itemize}
\item
  Parks: This data set provides information on the names and locations of City parks.
\item
  Cultural Spaces: This data set provides locations and attributes of cultural spaces in the City of Vancouver, University Endowment Lands, and Musqueam Community.
\item
  Schools: This dataset includes the name, address, location, and category of schools in Vancouver.
\item
  Libraries: This data set contains the name, address, latitude, and longitude of each of the City's public libraries.
\end{itemize}

\textbf{\href{https://catalogue-hrm.opendata.arcgis.com/search?tags=Halifax\%20Open\%20Data}{City of Halifax}}

\textbf{\emph{Cycling and pedestrian Networks}}

\begin{itemize}
\tightlist
\item
  Bike Infrastructure and Suggested Routes
\end{itemize}

Line representation of existing and suggested bike routes and infrastructure within the Halifax Regional Municipality. This dataset includes Bike Facility ID, Bike Facility Type, Bike Facility Name, Street Name, Bike Facility Implementation, Pilot Project or Permanent, One Side Only, Direction, Install Year, Physical Protection.

\begin{itemize}
\tightlist
\item
  Active Travelways
\end{itemize}

Active Travel Ways is a Linear representation of assets that includes trails, sidewalks, walkways, pathways, and Multi-Use Pathways.

\begin{itemize}
\tightlist
\item
  Pedestrian Oriented Commercial Streets
\end{itemize}

The location of pedestrian-oriented commercial streets defined in the Regional Centre Land Use By-law is depicted as a linear representation.

\begin{itemize}
\tightlist
\item
  Trails Single line representations of trails either owned, maintained, or of interest to HRM.
\end{itemize}

\textbf{\emph{Land use and Land cover}}

\begin{itemize}
\tightlist
\item
  Outdoor Recreation: Polygon representations of HRM-owned or maintained outdoor recreation areas such as active use areas, sports fields, sports courts, ball diamonds, and water-related areas.
\end{itemize}

\textbf{\emph{Points of Interest (POI)}}

\begin{itemize}
\item
  Bus Stops Point represents bus stop locations along roadways, terminals, and park-and-ride facilities.
\item
  Outdoor Recreation Equipment: Points representation of HRM-owned or maintained outdoor recreation equipment such as playground equipment, outdoor gym equipment, basketball hoops, and modular ramps.
\item
  Park Recreation Features: Represent various owned or maintained outdoor recreational uses.
\item
  Schools: This dataset includes the school locations in Halifax.
\end{itemize}

\textbf{\href{https://data.waterloo.ca/search?tags=base\%20data}{City Of Waterloo}}

\textbf{\emph{Cycling and pedestrian Networks}}

\begin{itemize}
\tightlist
\item
  Active Transportation
\end{itemize}

Active Transportation represents active transportation infrastructure, including cycling, sidewalks, walkways, and pathways (trails).

\begin{itemize}
\tightlist
\item
  Trail and Sidewalk Clearing
\end{itemize}

This dataset provides Sidewalks and Trails plowed or shoveled by the City.

\begin{itemize}
\tightlist
\item
  Walkability
\end{itemize}

This dataset combines sidewalks, trails, and links, forming a continuous network. To be used in conjunction with the road network. Walkability is a measure of how accommodating the built environment is to walking. This dataset is a work in progress and will be updated as new information becomes available.

\textbf{\emph{Points of Interest (POI)}}

\begin{itemize}
\tightlist
\item
  Places of Worship
\item
  Points of Interest: Point locations of various points of interest within the City of Waterloo.
\item
  Schools
\end{itemize}

\textbf{\href{https://data.waterloo.ca/search?tags=base\%20data}{City Of Kitchener}}

\textbf{\emph{Cycling and pedestrian Networks}}

\begin{itemize}
\item
  Trails - Community trails are multi-use pathways that exist throughout the city.
\item
  Sidewalk: The sidewalk inventory provides information on the location of sidewalks along the right of way in the city.
\item
  Cycling Infrastructure: The cycling data contains bicycle lanes, signed bicycle routes, and pathways.
\item
  Crosswalk: The crosswalk inventory provides information and the location of crosswalks along the right of way in the city
\end{itemize}

\textbf{\emph{Points of Interest (POI)}}

\begin{itemize}
\item
  Parks: This dataset includes the shapefile of parks in Kitchener.
\item
  Libraries: This dataset includes Kitchener Public Library (KPL) locations.
\item
  Playgrounds
\item
  Business Directory: This dataset contains a list of businesses operating in the City of Kitchener in 2017.
\item
  Landmarks include: education facilities, government services, parks, emergency services, entertainment, points of interest, and more
\item
  Points of Interest: Point locations of various points of interest within the City of Kitchener
\item
  Arenas and Outdoor Ice Rinks
\item
  Pools and Splash Pads
\item
  Places of Worship
\item
  Public Art and Industrial Artifacts
\item
  Sports facilities
\item
  Education Facility
\item
  Community Centres
\item
  Museums
\item
  Hospitals
\end{itemize}

\textbf{\href{https://data.waterloo.ca/search?tags=base\%20data}{City Of Burlington}}

\textbf{\emph{Cycling and pedestrian Networks}}

\begin{itemize}
\item
  Pedestrian Network: The pedestrian network contains the location and details about all pathways and sidewalks in the City of Burlington.
\item
  Cycling Network: Physical cycling facilities within the City of Burlington. Facilities include bike lanes, paved shoulders, shared use (sharrows), off-road, and adjacent road multiuse paths.
\end{itemize}

\textbf{\emph{Points of Interest (POI)}}

\begin{itemize}
\item
  City of Burlington Address points
\item
  Transit Bus Stops
\item
  Public art
\item
  Parks Points
\item
  Libraries
\item
  Facilities: This dataset includes arenas, pools, community centers, park facilities, etc. It also includes joint venture features.
\end{itemize}

\newpage

Table \ref{tab:table-8} presents a structured overview of the availability of different data types across several travel-related datasets in Canada. The table is organized into seven categories: Travel Data, Users Data, Origin-Destination Data, Cycling and Walking Data, Spatial Data, and Additional Data, which we explained in Table 1.5. Each category corresponds to a specific type of information that might be useful for active travel-based accessibility analysis.
The General Social Survey (GSS), Transportation Tomorrow Survey (TTS) of Toronto, and Autorité régionale de transport métropolitain (ARTM) survey of Montreal are comprehensive datasets that provide a wide range of information, covering most of the data categories identified. For instance, the GSS and TTS both include Travel, Users, and Origin-Destination Data, with TTS also offering insights into Cycling and Walking as well as Spatial Data, indicating its extensive coverage of travel behaviors and patterns.
On the other hand, localized datasets such as the Vancouver open data and Montreal open data, although not as comprehensive as the GSS or TTS, still offer valuable information in specific categories, including Origin-Destination and Cycling and Walking data, which are essential for urban planning and promoting active transportation modes.
This categorization and subsequent analysis highlight the richness and diversity of data available for transportation and urban planning in Canada. It demonstrates the potential for these datasets to inform policy-making and infrastructure development, particularly in enhancing active travel accessibility and understanding spatial dynamics within urban environments. The datasets that provide additional data, such as stress levels from the Canadian Business Counts (CBC) dataset, suggest opportunities for multidimensional analyses that can incorporate psychological factors into accessibility studies.

\begin{table}
\centering
\caption{\label{tab:Table8} \label{tab:table-8} Availability of datasets}
\centering
\resizebox{\ifdim\width>\linewidth\linewidth\else\width\fi}{!}{
\fontsize{8}{10}\selectfont
\begin{tabular}[t]{>{\raggedright\arraybackslash}p{3cm}>{\raggedright\arraybackslash}p{3cm}>{\raggedright\arraybackslash}p{3cm}>{\raggedright\arraybackslash}p{3cm}>{\raggedright\arraybackslash}p{3cm}ll}
\toprule
\multicolumn{1}{>{\centering\arraybackslash}p{3cm}}{\textbf{Dataset}} & \multicolumn{1}{>{\centering\arraybackslash}p{3cm}}{\textbf{Travel Data}} & \multicolumn{1}{>{\centering\arraybackslash}p{3cm}}{\textbf{Users Data}} & \multicolumn{1}{>{\centering\arraybackslash}p{3cm}}{\textbf{Origin-Destination Data}} & \multicolumn{1}{>{\centering\arraybackslash}p{3cm}}{\textbf{Cycling and Walking data}} & \multicolumn{1}{c}{\textbf{Spatial data}} & \multicolumn{1}{c}{\textbf{Additional data}}\\
\midrule
General social survey (GSS) & Y & Y & Y & - & - & -\\
Transportation Tomorrow Survey (TTS) of Toronto & Y & Y & Y & Y & Y & Y\\
Autorité régionale de transport métropolitain (ARTM) survey of Montreal & Y & Y & Y & Y & Y & Y\\
The Linkable Open Data Environment (LODE) & - & - & Y & - & - & -\\
Can-BICS & - & - & - & Y & - & Stress Level\\
\addlinespace
open street map & - & - & - & Y & Y & -\\
Toronto open data & - & - & - & Y & - & -\\
Vancouver open data & - & - & - & Y & - & -\\
Montreal open data & - & - & - & Y & - & -\\
\bottomrule
\end{tabular}}
\end{table}

\hypertarget{important-considerations-and-possible-challenges}{%
\section{Important considerations and possible challenges}\label{important-considerations-and-possible-challenges}}

\hypertarget{travel-timedistance-thresholds}{%
\subsection{Travel time/distance thresholds}\label{travel-timedistance-thresholds}}

Travel time and distance thresholds are essential for assessing accessibility for active transportation modes, such as walking and cycling. These metrics are pivotal as they delineate what is deemed accessible, significantly influencing the outcomes of accessibility studies. However, the challenge lies in determining these thresholds, which necessitates a delicate equilibrium between empirical evidence and normative presumptions.

A notable issue with conventional normative methods, as identified by Páez, Scott, and Morency (2012), is their underlying assumption that individuals invariably opt for the nearest available opportunity. This is compounded by their reliance on intuitively set distance thresholds, which may oversimplify the multifaceted nature of human mobility and the variance in individual preferences. This could potentially result in a skewed depiction of accessibility. In contrast, models of positive accessibility present a more nuanced approach by recognizing deprivation areas based on actual use patterns rather than normative benchmarks, thus eschewing the need for arbitrary thresholds. Nevertheless, this approach is contingent upon the availability of detailed data regarding trip distributions and distances, a requirement that might pose a significant constraint (Dai, 2011; Luo \& Wang, 2003; Paez, Scott, \& Morency, 2012b).

The task of selecting appropriate travel time or distance thresholds is critically acknowledged for its substantial impact on research outcomes in accessibility studies. This selection process demands a refined strategy that accommodates different thresholds for varied modes of transportation to accurately reflect their distinct attributes and the diverse abilities of different demographic groups. Despite this necessity, a notable inconsistency persists in applying these thresholds, even within the same mode, highlighting an imperative for a more unified methodological approach.

Research, such as the study conducted by Xing et al.~(2018), has delved into the sensitivity of accessibility outcomes to different time thresholds, exploring ranges between 15 to 35 minutes. These investigations generally reveal minimal variance within such intervals, suggesting that adopting a multimodal approach could mitigate overestimations and preserve uniformity across varied time thresholds. Yet, the definitive influence of threshold selection on accessibility metrics warrants further investigation. Furthermore, applying universal thresholds across all destination types and demographic groups fails to account for the heterogeneity in mobility needs and capacities. For instance, Saghapour (2017) underlines the necessity of differentiating thresholds based on destination types, advocating for distinct travel times for accessing retail and recreational centers versus community services. Similarly, imposing identical thresholds on all age groups neglects the unique mobility challenges seniors and children encounter, who may navigate slower or require more effort to traverse identical distances. While such distinctions have been recognized in the context of walking, their application to cycling remains inadequately addressed, suggesting a potential avenue for methodological refinement in accessibility studies (Saghapour et al., 2017 ; Xing, Liu, \& Liu, 2018). In the following, Table \ref{tab:table_9} shows the thresholds of bicycle and walking travel time and distance in different studies.

\begin{landscape}\begingroup\fontsize{12}{14}\selectfont

\begin{longtable}[t]{>{\raggedright\arraybackslash}p{9cm}>{\raggedright\arraybackslash}p{9cm}}
\caption{\label{tab:Table9}\label{tab:table_9}Thresholds of walking and cycling travel time and distance}\\
\toprule
\multicolumn{1}{>{\centering\arraybackslash}p{9cm}}{\textbf{Distance/Time thresholds}} & \multicolumn{1}{>{\centering\arraybackslash}p{9cm}}{\textbf{Description}}\\
\midrule
Neilson and Fowler (1972), \textbackslash{} O'Neill, Ramsey et al. (1992), \textbackslash{} Hsiao, Lu et al. (1997), \textbackslash{} Murray and Wu (2003), \textbackslash{} Zhao, Chow et al. (2003), \textbackslash{} Kimpel, Dueker et al. (2007), \textbackslash{} Gutiérrez and García-Palomares (2008) & The most common standard measure \textbackslash{} of walking distance to transit \textbackslash{} stops and stations has been 400 m (0.25 miles).\\
Lam and Morrall (1982) & In Calgary, Canada, observed a median \textbackslash{} walking distance to bus stops of 292 m, \textbackslash{} while the average was 327 m and the \textbackslash{} 75th percentile, 450 m.\\
O'Sullivan and Morrall (1996) & distinguished between walking to light-rail \textbackslash{} transit stations in the suburbs and in the \textbackslash{} central business district. They found an \textbackslash{} average distance of 649 m and a 75th \textbackslash{} percentile equal to 840 m in the former, \textbackslash{} while the average distance was 326 metres \textbackslash{} and the 75th percentile was 419 metres \textbackslash{} in the latter (Calgary, Canada)\\
Arasan, Rengaraju et al. (1996) & an average critical trip time is \textbackslash{} 20 min for walking.\\
Nicholls (2001), Smoyer?Tomic, \textbackslash{} Hewko et al. (2004) & used a distance of 0.8 km as a reasonable \textbackslash{} threshold for walking trips (the threshold is \textbackslash{} not specific to a population group)\\
\addlinespace
Zhao, Chow et al. (2003) & in southeast Florida, the number of riders \textbackslash{} walking over half a mile (800m) was negligible.\\
Van Herzele and Wiedemann (2003) & Maximum distance from home to: \textbackslash{} 1) Residential green (150 m); \textbackslash{} Neighborhood green (400 m); \textbackslash{} Quarter green (800 m); \textbackslash{} District green (1600 m); \textbackslash{} City green (3200 m); \textbackslash{} Urban forest (5000 m)\\
Tsou, Hung et al. (2005) & Defined varying distances that depended \textbackslash{} on the type of facility:\\
 & · the service range of municipal facilities \textbackslash{} such as town parks, universities, museums \textbackslash{} and dump sites cover the entire city.\\
 & · community facilities, including junior \textbackslash{} and senior high schools, transformer \textbackslash{} stations, etc., are typically in the 2 km range.\\
\addlinespace
Schlossberg, Agrawal et al. (2007) & · The service range of neighborhood \textbackslash{} facilities like playgrounds and elementary \textbackslash{} schools is typically in the 1 km range.\\
Alshalalfah and Shalaby (2007) & walking distances to rail transit stations \textbackslash{} in Portland, WA, and San Francisco, were \textbackslash{} a median distance of 0.47 miles (756 m)\\
Larsen and Gilliland (2008) & showed that among transit users, 60 \% \textbackslash{} live within 300 m from their stop and \textbackslash{} 80 \% within 500 m in Canada.\\
Manaugh and El-Geneidy (2011) & Population within 500 m walk \textbackslash{} distance of supermarkets\\
Daniels and Mulley (2013) & used 400, 800 and 1200 m thresholds \textbackslash{} for calculating walkability score\\
\addlinespace
 & the mean walking distance to bus service \textbackslash{} 461 m with 75th percentile at 566 m.\\
 & In the same study they found mean walking \textbackslash{} to rail around 805 m and the 75th percentile \textbackslash{} at 1,018 m. Also, it is clear that these \textbackslash{} distances are significantly beyond the 400 m \textbackslash{} for buses and 800 for rail.\\
\bottomrule
\end{longtable}
\endgroup{}
\end{landscape}

\hypertarget{impedance-functions}{%
\subsection{Impedance functions}\label{impedance-functions}}

Impedance function is used to describe the willingness of cyclists and pedestrians to travel to a destination as a function of cost (distance, time, etc.); it is a component of accessibility (Arranz-Lopez et al., 2019; Iacono et al., 2010; Yang \& Diez-Roux, 2012). The impedance function obtained by fitting to a real dataset provides a continuous description of cycling and walking probability at different costs. The spatial distribution of bicycle and pedestrian travel can be expressed using distance decay functions (Iacono et al., 2010) as the travel distance is a limiting factor for implementing use(Larsen \& El-Geneidy, 2011). Distance decay functions describe the effect of distance on spatial interactions and typically express distance as a function of travel impedance (time or cost). Rybarczyk and Wu (2010) identified the importance of the spatial patterns of bicycle facilities and local network connectivity when studying accessibility. Furthermore, increased connectivity within a network also allows for increased accessibility(Rybarczyk \& Wu, 2010).

Some researchers have argued that -- like maximum travel thresholds -- distance-decay rates should differ according to trip purpose and different population groups (Garcia-Palomares, Gutierrez, \& Cardozo, 2013; X. Wu, Lu, Lin, \& Yang, 2019). Similarly, researchers also argue that walking and cycling impedance functions should be calculated separately due to their differing travel speeds and maximum travel ranges (Cheng, Caset, De Vos, Derudder, \& Witlox, 2019). For example, the distance-decay curve for work trips shown in Figure \ref{fig:initiative} assumes that cyclists are half as likely to reach a work destination 20 minutes away than one 10 minutes away. Therefore, jobs 20 minutes away would be applied half the weight of jobs 10 minutes away. These cycling weights differ slightly from walking trips since fewer people are willing to walk longer to work. While adjusting the distance-decay functions by mode has the potential to improve accuracy, it can also be said that a consistent approach to measuring accessibility across modes is preferable due to the possibility of causing one mode to seem less accessible when applying different decay functions (State et al.~Initiative, 2021).

Regarding the types of impedance functions considered, a negative exponential curve is familiar (Saghapour et al., 2017; X. Wu et al., 2019) -- example shown in Figure \ref{fig:initiative}. However, some studies have also calculated study-specific distance-decay curves based on trip data rather than assuming a standard function. Wu, Lu et al.~(2019) calculated a distance-decay function using data from Shenzhen's dockless bicycle-sharing system. Their findings show that a lognormal distance decay best fits the distribution of bike-sharing trips, with the willingness to cycle increasing up to 500m and decreasing after that. García-Palomares, Gutiérrez et al.~(2013) took a similar approach to measure walking accessibility to metro stations and found a linear distance-decay trend that varied significantly by age (Garcia-Palomares et al., 2013; X. Wu et al., 2019).

\begin{figure}

{\centering \includegraphics[width=0.9\linewidth]{figure/ch02-Fig-02} 

}

\caption{Example travel time decay functions by mode for work vs non work trips (Source: State Smart Transportation Initiative - based on data from 2017 National Household Travel Survey)}\label{fig:initiative}
\end{figure}

\hypertarget{slope}{%
\subsection{Slope}\label{slope}}

One of the factors associated with the natural environment that affects bicycle and walking trips is the slope. Hence, pedestrians and cyclists will travel out of their way to bypass segments with steep slopes. For them, small positive increments in slope decrease travel speeds while increasing energy use and travel time. Besides, due to the differences in efforts to go up-slope versus down-slope, pedestrians and cyclists may not select the same way. It is referred to as anisotropic movement (Ebener, El Morjani, Ray, \& Black, 2005).

Few studies have considered slope in measures of active accessibility. However, it is acknowledged as an essential factor since people often avoid routes with significant elevation gain, and routes with steep slopes may significantly impact accessibility. Often, network analyst tools use the shortest path from the road network, which may not reflect actual cycling or walking behavior. Vale, Saraiva, et al.~(2015) concluded that slopes should always be included in the accessibility of bicycling and that they are also crucial for walking. However, it is mainly absent from walking accessibility measures (D. S. Vale et al., 2016). However, the greater availability of elevation data and advances in research in various disciplines offer opportunities to understand the behavior of individuals better when traveling in infrastructure-poor contexts and challenge assumptions surrounding the most significant costs to be minimized. Papa, Carpentieri, et al.~(2018) also highlighted a significant difference in catchment areas when including versus excluding the slope attribute (33\% \(km^2\) difference for adults over 75) (Papa, Carpentieri, \& Guida, 2018). Wood, Jones et al.~(2018) studied the sensitivity in distance calculation to variations in three travel-time modeling approaches, taking as reference a model that accounted for variations in land cover and directionality in slope (anisotropy). They found that an approach based on measuring Euclidean distances on a flat surface underestimated the distance traveled relative to the reference. The second approach, which calculated the distances constrained to a road network, also varied substantially from the reference, underestimating it in some areas and overestimating it in others. Finally, the third approach, which accounted for land cover and elevation but ignored the directionality of slopes, slightly underestimated travel times (Wood, Jones, Peters, \& Richards, 2018).

Lundberg (2012) examined the local cycling and walking networks through Geographic Information Systems (GIS) using accessibility. They extracted a percent slope raster layer from the DEM layer obtained from part of the National Elevation Dataset (NED). The percent slope of the DEM ranged from 0 to 360. In Arc Map, the rise equals the run when the slope angle equals 45 degrees. Expressed as a percentage, the slope of this angle is 100\%. As the slope angle approaches vertical (90o), the percentage slope approaches infinity. An X and Y coordinate was first calculated for the start point of each line segment. Next, an X and Y coordinate was calculated for the endpoint of each line segment. Arc Map's 3D Analyst extension was used to convert the street network into a 3D layer, at which point the percent slope could be calculated as the Z-value for each line segment in the network. A Z-value (elevation) was calculated at each line segment's start and end points. The equation \ref{eq:slope} was used to derive the percent slope for each line segment:

\begin{equation}
\frac{End_z - Start_z}{\sqrt{(End_x - Start_x)^2 + (End_y - Start_y)^2}} \times 100
\label{eq:slope}
\end{equation}

Slope values indicate uphill travel, while negative slope values indicate downhill travel. In this regard, they proposed different walking and cycling speeds based on the different slopes using Parkin and Rotheram's (2010) findings on the impact of slopes on bicycle travel speeds(Parkin \& Rotheram, 2010). Table \ref{tab:table_10} summarizes the various bicycle travel speeds used in the GIS modeling.

\newpage

\begingroup\fontsize{12}{14}\selectfont

\begin{longtable}[t]{>{\centering\arraybackslash}p{5cm}>{\centering\arraybackslash}p{5cm}}
\caption{\label{tab:table10}\label{tab:table_10}Bicycle travel speeds used in GIS modeling}\\
\toprule
\multicolumn{1}{>{\centering\arraybackslash}p{5cm}}{\textbf{Slope}} & \multicolumn{1}{>{\centering\arraybackslash}p{5cm}}{\textbf{Speed(mph)}}\\
\midrule
-10 & 18.8\\
-7 & 17.1\\
-5 & 16.1\\
-2 & 14.5\\
0 & 13.4\\
\addlinespace
2 & 11.7\\
5 & 8.9\\
7 & 7.2\\
10 & 4.5\\
\bottomrule
\end{longtable}
\endgroup{}

Pedestrian travel speeds were also calculated based on the effect of slope. Tobler's hiking function was used to identify the effect of slope on travel speed. The equation \ref{eq:modified-formula} represents the modified Tobler's formula adjusted for percent slope as follows:

\begin{equation}
V = 6 e^{-3.5 |s + 0.05|}
\label{eq:modified-formula}
\end{equation}

Where v is velocity, e is the base for natural logarithms, and s is the slope in percent. Table \ref{tab:table_11} summarizes a pedestrian's travel speed used in the modeling in GIS.

\begingroup\fontsize{12}{14}\selectfont

\begin{longtable}[t]{>{\centering\arraybackslash}p{5cm}>{\centering\arraybackslash}p{5cm}}
\caption{\label{tab:table11}\label{tab:table_11}Pedestrian travel speeds used in GIS modeling}\\
\toprule
\multicolumn{1}{>{\centering\arraybackslash}p{5cm}}{\textbf{Slope}} & \multicolumn{1}{>{\centering\arraybackslash}p{5cm}}{\textbf{Speed(mph)}}\\
\midrule
10.0 & 1.6\\
7.5 & 2.1\\
5.0 & 2.4\\
2.5 & 2.8\\
0.0 & 3.1\\
\addlinespace
-2.5 & 3.6\\
-5.0 & 3.1\\
-7.5 & 2.6\\
-10.0 & 2.3\\
\bottomrule
\end{longtable}
\endgroup{}

In another study, Paez et al.~(2020) calculated the slope from the vertical and horizontal displacements. The instantaneous slope \(m\) is given by the derivative of \(y = f(x)\) concerning \(x\). This is given by the equation \ref{eq:calculating-the-slope-from-the-vertical-and-horizontal-displacements} (Paez et al., 2020):

\begin{equation}
m = \frac{\Delta V}{\Delta h}
\label{eq:calculating-the-slope-from-the-vertical-and-horizontal-displacements}
\end{equation}

In a DEM layer, two physical aspects of the landscape related to resistance can be obtained directly from the grid: the vertical and horizontal displacement between nodes i and j. \(\Delta V\) and \(\Delta h\) are vertical and horizontal displacements, respectively. This slope is linked to speed via Tobler's formula for hiking travel using equation \ref{eq:slope-linked-to-speed-via-formula} (Tobler, 1993):

\begin{equation}
S = 100 e^{-3.5 |m + 0.5|}
\label{eq:slope-linked-to-speed-via-formula}
\end{equation}

Where the speed S is in \(m/min\), the amount of speed can be converted into travel time in minutes if it is divided the distance by speed using equation \ref{eq:amount-of-speed}:

\begin{equation}
t = \frac{d_i}{100 e^{-3.5 |m + 0.5|}} = \frac{1}{100} d_i e^{3.5 |m + 0.5|}
\label{eq:amount-of-speed}
\end{equation}

Where \(d_i\) can be the distance on the surface as discussed above or can be approximated by the horizontal distance \(\Delta h\), as seen in Figure \ref{fig:surface-distance}, travel time tends to increase as the slope increases. Figure \ref{fig:surface-distance} by (Paez et al., 2020) illustrates the calibrated results of a bespoke walking cost function that factors in varying terrain gradients and their impact on travel efficiency.

\begin{figure}

{\centering \includegraphics[width=0.9\linewidth]{figure/ch02-Fig-03} 

}

\caption{Relationship between surface distance/travel time and slope, adapted from Higgins, C. D. (2021). 'Hiking with Tobler: Tracking Movement and Calibrating a Cost Function for Personalized 3D Accessibility.' Findings, September.}\label{fig:surface-distance}
\end{figure}

\newpage

\hypertarget{weather}{%
\subsection{Weather}\label{weather}}

Other factors that are associated with the natural environment and have also been shown to affect cycling and walking trips are as follows: weather, temperature, shade, and aesthetics. The type of weather an individual has to travel through has been identified as a principal factor in the decision process for employing non-motorized travel modes. The pinnacle conditions that individuals consider using non-motorized travel include dry weather and pleasant temperatures (60 to 75) (Zacharias, 2001). High shade cover over a network and available aesthetics along a route increase the rates for non-motorized travel (Zahran, Brody, Maghelal, Prelog, \& Lacy, 2008).

\hypertarget{level-of-traffic-stress}{%
\subsection{Level of Traffic Stress}\label{level-of-traffic-stress}}

Several of the cycling accessibility approaches incorporate bicycle infrastructure using the level of traffic stress (LTS) (Imani, Miller, \& Saxe, 2019; Murphy \& Owen, 2019). Furth first proposed the LTS method, Mekuria et al.~(2016) to categorize street segments into four categories based on the number of lanes, the presence of a parking lane, the speed limit, the bike lane and parking lane width, and any bike lane blockage (Furth, Mekuria, \& Nixon, 2016). Faghih Imani et al.~(2019) and Murphy and Owen (2019) compare cycling accessibility measures using different LTS categories to calculate service areas. Both studies exclude highways and high-volume roads from the network and classify them into LTS categories using attribute information from the network dataset and the City of Toronto open data and OSM data, respectively(Imani et al., 2019; Murphy \& Owen, 2019).

The Canadian Bikeway Comfort and Safety (Can-BICS) Classification System aims to provide a standard nomenclature for bicycle facilities based on user safety and comfort. The Can-BICS classification categorizes bicycling facilities into high, medium, and low comfort infrastructure. There is some general alignment between the Can-BICS categories and LTS criteria. However, there are a few main differences:

Local street bikeways are classified as high comfort using Can-BICS, but either LTS 1 or 2, depending on the number of lanes.

Painted bike lanes may be assigned LTS 1 to 4 depending on the speed, width, and presence of parking lanes, whereas in Can-BICS, painted lanes are low-comfort facilities.

Trails and park walkways are LTS 1 but may be categorized as non-conforming Can-BICS facilities depending on the trail surface (e.g., gravel or dirt vs.~paved).

\begingroup\fontsize{12}{14}\selectfont

\begin{longtable}[t]{>{\centering\arraybackslash}p{5cm}>{\centering\arraybackslash}p{5cm}c}
\caption{\label{tab:table12}\label{tab:table_12}The Canadian Bikeway Comfort and Safety (Can-BICS) Classification System}\\
\toprule
\multicolumn{1}{>{\centering\arraybackslash}p{5cm}}{\textbf{Facility Type}} & \multicolumn{1}{>{\centering\arraybackslash}p{5cm}}{\textbf{Can-BICS Class}} & \multicolumn{1}{c}{\textbf{LTS Category}}\\
\midrule
Cycle tracks & High comfort & 1\\
Local street bikeway & High comfort & 1 or 2\\
Bike paths & High comfort & 1\\
Multi-use paths & Medium comfort & 1\\
Painted Bike lanes & Low comfort & 1 to 4\\
\addlinespace
Park trails and walkways & Non-conforming & 1\\
\bottomrule
\end{longtable}
\endgroup{}

\hypertarget{originsdestinations-applying-weights}{%
\subsection{Origins/destinations \& applying weights}\label{originsdestinations-applying-weights}}

How opportunities are measured depends on the type of opportunity and whether one or multiple opportunity types are considered. For example, the studies measuring job accessibility or number of people served by transit consider a total count, whereas for urban park access, Reyes et al.~(2014) and Cheng et al.~(2019) consider cell counts to account for park area(Cheng et al., 2019; Reyes, Paez, \& Morency, 2014). Among walkability and bikeability indices, weights are expected to be applied to the variables depending on the goals for analysis (Arellana, Alvarez, Oviedo, \& Guzman, 2021; L. D. Frank et al., 2010; D. S. Vale et al., 2016).

Most studies focus on origin-based accessibility (access to destinations); however, Vale et al.~(2015) argue that accessibility to destinations is also essential. For example, individuals may reside in highly accessible areas but work in low-accessibility areas. In this respect, topology-based measures may be preferred, or it may be helpful to consider accessibility regarding the population served around destinations of interest(D. S. Vale et al., 2016).

\hypertarget{summary}{%
\section{Summary}\label{summary}}

Overall, there remains considerable variation in the accessibility measures applied in walking and cycling. Among the four main types of active accessibility measures identified, most recent studies use an activity-based approach, either measuring cumulative opportunities (within a catchment area / weighted by distance from the origin) or measuring gravity models. The activity-based approaches mainly vary regarding the travel time, distance thresholds, and weighting impedance functions considered.

For instance, many of the walking approaches aim to incorporate a variable walking threshold (depending on age or location). At the same time, this is not seen within the context of cycling -- where prioritization of measures by route infrastructure is more apparent. Conversely, attention to infrastructure type, or comfort and safety, is not seen among pedestrian-focused studies.

When selecting an accessibility measure, there is a trade-off between complexity and measure interpretability. While adding more complexity or multiple indices for different population groups may increase accuracy, a simple and easy-to-implement measure may be more critical for widespread use.

\hypertarget{discussion}{%
\section{Discussion}\label{discussion}}

Eliciting user needs data is discovering requirements for a project by accessing available knowledge sources and communicating with stakeholders who directly or indirectly influence the requirements. Among the available requirements elicitation techniques, the most common and effective elicitation technique is interviews with stakeholders(Browne \& Rogich, 2001; Engelbrektsson, Yesil, \& Karlsson, 2000; Friedrich \& Van Der Poll, 2007; Hadar, Soffer, \& Kenzi, 2014). As a result, in this section, we discuss some of the questions required to extract data users' requirements.

\begin{figure}

{\centering \includegraphics[width=0.9\linewidth]{figure/ch02-Fig-04} 

}

\caption{Elicitation data needs in interview}\label{fig:Flowchart}
\end{figure}

Figure \ref{fig:Flowchart} presented is a structured approach for eliciting user needs data in the context of evaluating transportation projects or policies with a focus on accessibility, particularly walking and cycling modes. The flowchart is designed through a comprehensive set of questions tailored to gather detailed information from organizations on their current practices, preferences, and data needs related to accessibility analysis. The starting point questions whether the organization has previously used the accessibility method to evaluate transportation projects or policies focused on active travel modes like walking and cycling. This dichotomous branch leads to two separate pathways:
If `Yes':
The flowchart prompts the interviewer to inquire about the specific accessibility measurement tools the organization has used and their preferences among these.
Questions then delve into the modes of transportation, types of destinations analyzed, and whether analyses consider different traveler capabilities, reflecting inclusivity in their assessments.
The organization's perspective on the importance of individual traveler characteristics and the scale at which accessibility analysis is implemented are probed.
The flowchart encourages asking about the evaluation of accessibility by active travel and the data utilized, probing for additional relevant data sources that may be outside the organization's current use or awareness.
If `No':
The flowchart shifts focus to understanding current methods used by the organization for analyzing transportation and land use policy, awareness and perception of accessibility as an analysis method, and the respondent's own description of accessibility.
It explores the organization's potential interest in using accessibility analysis and the expected data needs for such analysis, particularly concerning active travel.
The flowchart ends by emphasizing the need for interviewers to understand any other relevant data sources the organization is aware of but not currently using, whether these are publicly available, and what their ideal data would be to support accessibility by active travel.

\hypertarget{math-sci}{%
\chapter{A Historical Analysis of the Evolution of Active Travel Behaviour in Canada}\label{math-sci}}

Placeholder

\hypertarget{abstract-1}{%
\section{Abstract}\label{abstract-1}}

\hypertarget{introduction-2}{%
\section{Introduction}\label{introduction-2}}

\hypertarget{background-1}{%
\section{Background}\label{background-1}}

\hypertarget{materials-and-methods}{%
\section{Materials and Methods}\label{materials-and-methods}}

\hypertarget{the-gss-survey}{%
\subsection{The GSS survey}\label{the-gss-survey}}

\hypertarget{estimating-impedance-function-parameters}{%
\subsection{Estimating impedance function parameters}\label{estimating-impedance-function-parameters}}

\hypertarget{results-and-discusion}{%
\section{Results and discusion}\label{results-and-discusion}}

\hypertarget{descriptive-analysis-of-walking-and-cycling-trips-from-1986-to-2015}{%
\subsection{Descriptive analysis of walking and cycling trips from 1986 to 2015}\label{descriptive-analysis-of-walking-and-cycling-trips-from-1986-to-2015}}

\hypertarget{impedance-function-analysis}{%
\subsection{Impedance function analysis}\label{impedance-function-analysis}}

\hypertarget{conclusion}{%
\chapter*{Conclusion}\label{conclusion}}
\addcontentsline{toc}{chapter}{Conclusion}

This study has embarked on a comprehensive journey through three decades of active travel behavior in Canada, from 1986 to 2015, leveraging data from the General Social Survey (GSS) to shed light on individual preferences and behaviors regarding walking and cycling. This exploration is particularly relevant as it spans a period of significant urban development, evolving societal attitudes towards health and the environment, and transformations in transportation infrastructure and policies.
Our results show that walking and cycling have consistently longer trip times, with shorter walking trips. Between 1986 and 2005, this gap shrunk, suggesting the intricate interactions between urban sprawl, a greater reliance on motorized transportation, and shifting preferences.git pus From 2005 to 2015, walking and cycling trip durations increased, likely influenced by heightened urbanization, health and sustainability concerns awareness, and urban planning shifts towards active transportation.

The study also uncovered trends in trip destinations, with cycling trips in 2015 primarily directed towards homes, schools, and workplaces, highlighting cycling's emergence as a sustainable commuting option. Walking trips predominantly originated from residential areas to workplaces and educational institutions, reflecting a shift towards walking for eco-friendly and health-conscious commuting.
However, the research encountered limitations, notably in handling weighted data for statistical tests and model diagnostics. The reliance on AIC, BIC, and log-likelihood values underscores the necessity for advanced tools that accommodate weighted data, addressing a methodological gap in statistical practices. The absence of cycling data further limits our analysis, presenting an avenue for future research to explore cycling trends and its potential as an alternative transportation mode for accessing cultural venues.

In light of these limitations, future research should prioritize the development of methodologies and diagnostic tools that effectively incorporate weighted data, enhancing the robustness and representativeness of findings in survey-based analyses. Investigating cycling data could also provide a fuller picture of active travel behaviors and their implications for urban planning and policy-making.
In conclusion, this thesis presents a detailed account of the evolution of active travel behaviors in Canada over three decades, highlighting the resurgence of walking and cycling as key modes of transportation. These insights are invaluable for urban planners, policymakers, and researchers striving to foster sustainable and active transportation systems. Future research can further enrich our understanding of active travel behaviors and support efforts to promote sustainable mobility options by addressing the identified methodological challenges and exploring underrepresented areas such as cycling trends.

\hypertarget{references}{%
\chapter*{References}\label{references}}
\addcontentsline{toc}{chapter}{References}

Placeholder

\hypertarget{refs}{}
\begin{CSLReferences}{1}{0}
\leavevmode\vadjust pre{\hypertarget{ref-achuthan2007measuring}{}}%
Achuthan, K., Titheridge, H., \& Mackett, R. (2007). Measuring pedestrian accessibility. In \emph{Proceedings of the geographical information science research UK (GISRUK) conference, national centre for geocomputation, national university of ireland} (pp. 264--269). Citeseer.

\leavevmode\vadjust pre{\hypertarget{ref-akgun2018cyclist}{}}%
Akgun, N., Dissanayake, D., Thorpe, N., \& Bell, M. C. (2018). Cyclist casualty severity at roundabouts--to what extent do the geometric characteristics of roundabouts play a part? \emph{Journal of Safety Research}, \emph{67}, 83--91.

\leavevmode\vadjust pre{\hypertarget{ref-angotti1996latin}{}}%
Angotti, T. (1996). Latin american urbanization and planning: Inequality and unsustainability in north and south. \emph{Latin American Perspectives}, \emph{23}(4), 12--34.

\leavevmode\vadjust pre{\hypertarget{ref-apparicio2008comparing}{}}%
Apparicio, P., Abdelmajid, M., Riva, M., \& Shearmur, R. (2008). Comparing alternative approaches to measuring the geographical accessibility of urban health services: Distance types and aggregation-error issues. \emph{International Journal of Health Geographics}, \emph{7}(1), 1--14.

\leavevmode\vadjust pre{\hypertarget{ref-arellana2021walk}{}}%
Arellana, J., Alvarez, V., Oviedo, D., \& Guzman, L. A. (2021). Walk this way: Pedestrian accessibility and equity in barranquilla and soledad, colombia. \emph{Research in Transportation Economics}, \emph{86}, 101024.

\leavevmode\vadjust pre{\hypertarget{ref-arranz2019}{}}%
Arranz-Lopez, A., Soria-Lara, J. A., Witlox, F., \& Paez, A. (2019). Measuring relative non-motorized accessibility to retail activities. \emph{International Journal of Sustainable Transportation}, \emph{13}(9), 639--651.

\leavevmode\vadjust pre{\hypertarget{ref-banister2005unsustainable}{}}%
Banister, D. (2005). \emph{Unsustainable transport: City transport in the new century}. Routledge.

\leavevmode\vadjust pre{\hypertarget{ref-banister2008sustainable}{}}%
Banister, D. (2008). The sustainable mobility paradigm. \emph{Transport Policy}, \emph{15}(2), 73--80.

\leavevmode\vadjust pre{\hypertarget{ref-ben1985discrete}{}}%
Ben-Akiva, M. E., Lerman, S. R., Lerman, S. R., et al. (1985). \emph{Discrete choice analysis: Theory and application to travel demand} (Vol. 9). MIT press.

\leavevmode\vadjust pre{\hypertarget{ref-ben2021disaggregate}{}}%
Ben-Akiva, M., \& Lerman, S. R. (2021). Disaggregate travel and mobility-choice models and measures of accessibility. In \emph{Behavioural travel modelling} (pp. 654--679). Routledge.

\leavevmode\vadjust pre{\hypertarget{ref-bhopal1995cycling}{}}%
Bhopal, R., \& Unwin, N. (1995). Cycling, physical exercise, and the millennium fund. \emph{BMJ}. British Medical Journal Publishing Group.

\leavevmode\vadjust pre{\hypertarget{ref-blevcic2015walkability}{}}%
Blevcic, I., Cecchini, A., Congiu, T., Fancello, F., Fancello, G., \& Trunfio, G. A. (2015). Walkability explorer: Application to a case-study. In \emph{International conference on computational science and its applications} (pp. 758--770). Springer.

\leavevmode\vadjust pre{\hypertarget{ref-bocarejo2012transport}{}}%
Bocarejo S, J. P., \& Oviedo H, D. R. (2012). Transport accessibility and social inequities: A tool for identification of mobility needs and evaluation of transport investments. \emph{Journal of Transport Geography}, \emph{24}, 142--154.

\leavevmode\vadjust pre{\hypertarget{ref-boisjoly2017}{}}%
Boisjoly, G., \& El-Geneidy, A. M. (2017). How to get there? A critical assessment of accessibility objectives and indicators in metropolitan transportation plans. \emph{Transport Policy}, \emph{55}, 38--50.

\leavevmode\vadjust pre{\hypertarget{ref-boschmann2008toward}{}}%
Boschmann, E. E., \& Kwan, M.-P. (2008). Toward socially sustainable urban transportation: Progress and potentials. \emph{International Journal of Sustainable Transportation}, \emph{2}(3), 138--157.

\leavevmode\vadjust pre{\hypertarget{ref-brezina2020}{}}%
Brezina, T., Leth, U., \& Lemmerer, H. (2020). Mental barriers in planning for cycling. In \emph{The politics of cycling infrastructure} (pp. 73--94). Policy Press.

\leavevmode\vadjust pre{\hypertarget{ref-brown2009}{}}%
Brown, J. R., Morris, E. A., \& Taylor, B. D. (2009b). Planning for cars in cities: Planners, engineers, and freeways in the 20th century. \emph{Journal of the American Planning Association}, \emph{75}(2), 161--177.

\leavevmode\vadjust pre{\hypertarget{ref-brown2009planning}{}}%
Brown, J. R., Morris, E. A., \& Taylor, B. D. (2009a). Planning for cars in cities: Planners, engineers, and freeways in the 20th century. \emph{Journal of the American Planning Association}, \emph{75}(2), 161--177.

\leavevmode\vadjust pre{\hypertarget{ref-browne2001empirical}{}}%
Browne, G. J., \& Rogich, M. B. (2001). An empirical investigation of user requirements elicitation: Comparing the effectiveness of prompting techniques. \emph{Journal of Management Information Systems}, \emph{17}(4), 223--249.

\leavevmode\vadjust pre{\hypertarget{ref-castiglione2006application}{}}%
Castiglione, J., Hiatt, R., Chang, T., \& Charlton, B. (2006). Application of travel demand microsimulation model for equity analysis. \emph{Transportation Research Record}, \emph{1977}(1), 35--42.

\leavevmode\vadjust pre{\hypertarget{ref-chapman2007transport}{}}%
Chapman, L. (2007). Transport and climate change: A review. \emph{Journal of Transport Geography}, \emph{15}(5), 354--367.

\leavevmode\vadjust pre{\hypertarget{ref-chaudhuri1997overview}{}}%
Chaudhuri, S., \& Dayal, U. (1997). An overview of data warehousing and OLAP technology. \emph{ACM Sigmod Record}, \emph{26}(1), 65--74.

\leavevmode\vadjust pre{\hypertarget{ref-chaudhuri1994including}{}}%
Chaudhuri, S., \& Shim, K. (1994). Including group-by in query optimization. In \emph{VLDB} (Vol. 94, pp. 12--15).

\leavevmode\vadjust pre{\hypertarget{ref-cheng2019investigating}{}}%
Cheng, L., Caset, F., De Vos, J., Derudder, B., \& Witlox, F. (2019). Investigating walking accessibility to recreational amenities for elderly people in nanjing, china. \emph{Transportation Research Part D: Transport and Environment}, \emph{76}, 85--99.

\leavevmode\vadjust pre{\hypertarget{ref-chi2013impact}{}}%
Chi, G., Porter, J. R., Cosby, A. G., \& Levinson, D. (2013). The impact of gasoline price changes on traffic safety: A time geography explanation. \emph{Journal of Transport Geography}, \emph{28}, 1--11.

\leavevmode\vadjust pre{\hypertarget{ref-colliat1996olap}{}}%
Colliat, G. (1996). OLAP, relational, and multidimensional database systems. \emph{ACM Sigmod Record}, \emph{25}(3), 64--69.

\leavevmode\vadjust pre{\hypertarget{ref-dai2011racial}{}}%
Dai, D. (2011). Racial/ethnic and socioeconomic disparities in urban green space accessibility: Where to intervene? \emph{Landscape and Urban Planning}, \emph{102}(4), 234--244.

\leavevmode\vadjust pre{\hypertarget{ref-devkota2012planning}{}}%
Devkota, B., Dudycha, D., \& Andrey, J. (2012). Planning for non-motorized travel in rural nepal: A role for geographic information systems. \emph{Journal of Transport Geography}, \emph{24}, 282--291.

\leavevmode\vadjust pre{\hypertarget{ref-dovey2020walkability}{}}%
Dovey, K., \& Pafka, E. (2020). What is walkability? The urban DMA. \emph{Urban Studies}, \emph{57}(1), 93--108.

\leavevmode\vadjust pre{\hypertarget{ref-duncan2011validation}{}}%
Duncan, D. T., Aldstadt, J., Whalen, J., Melly, S. J., \& Gortmaker, S. L. (2011). Validation of walk score for estimating neighborhood walkability: An analysis of four US metropolitan areas. \emph{International Journal of Environmental Research and Public Health}, \emph{8}(11), 4160--4179.

\leavevmode\vadjust pre{\hypertarget{ref-ebener2005physical}{}}%
Ebener, S., El Morjani, Z., Ray, N., \& Black, M. (2005). Physical accessibility to health care: From isotropy to anisotropy. \emph{GIS@ Development}, \emph{9}(6).

\leavevmode\vadjust pre{\hypertarget{ref-el2011place}{}}%
El-Geneidy, A., \& Levinson, D. (2011). Place rank: Valuing spatial interactions. \emph{Networks and Spatial Economics}, \emph{11}(4), 643--659.

\leavevmode\vadjust pre{\hypertarget{ref-emery2011wabsa}{}}%
Emery, J., \& Crump, C. (2011). The WABSA project: Assessing and improving your communitys walkability \& bikeability. 2003. \emph{North Carolina Dep. Heal. Behav. Heal. Educ. Sch. Public Heal. Univ. North Carolina Chapel Hill}.

\leavevmode\vadjust pre{\hypertarget{ref-engelbrektsson2000eliciting}{}}%
Engelbrektsson, P., Yesil, ozlem, \& Karlsson, I. (2000). Eliciting customer requirements in focus group interviews: Can efficiency be increased. In \emph{7th international product development management conference} (pp. 29--30). Citeseer.

\leavevmode\vadjust pre{\hypertarget{ref-espada2011application}{}}%
Espada, I., \& Luk, J. (2011). \emph{Application of accessibility measures}.

\leavevmode\vadjust pre{\hypertarget{ref-fan2012impact}{}}%
Fan, Y., Guthrie, A., \& Levinson, D. (2012). Impact of light-rail implementation on labor market accessibility: A transportation equity perspective. \emph{Journal of Transport and Land Use}, \emph{5}(3), 28--39.

\leavevmode\vadjust pre{\hypertarget{ref-fayyad1996data}{}}%
Fayyad, U., Piatetsky-Shapiro, G., \& Smyth, P. (1996). From data mining to knowledge discovery in databases. \emph{AI Magazine}, \emph{17}(3), 37--37.

\leavevmode\vadjust pre{\hypertarget{ref-frank2017international}{}}%
Frank, L. D., Fox, E. H., Ulmer, J. M., Chapman, J. E., Kershaw, S. E., Sallis, J. F., et al.others. (2017). International comparison of observation-specific spatial buffers.

\leavevmode\vadjust pre{\hypertarget{ref-frank2006many}{}}%
Frank, L. D., Sallis, J. F., Conway, T. L., Chapman, J. E., Saelens, B. E., \& Bachman, W. (2006). Many pathways from land use to health: Associations between neighborhood walkability and active transportation, body mass index, and air quality. \emph{Journal of the American Planning Association}, \emph{72}(1), 75--87.

\leavevmode\vadjust pre{\hypertarget{ref-frank2010development}{}}%
Frank, L. D., Sallis, J. F., Saelens, B. E., Leary, L., Cain, K., Conway, T. L., \& Hess, P. M. (2010). The development of a walkability index: Application to the neighborhood quality of life study. \emph{British Journal of Sports Medicine}, \emph{44}(13), 924--933.

\leavevmode\vadjust pre{\hypertarget{ref-frank2005linking}{}}%
Frank, L. D., Schmid, T. L., Sallis, J. F., Chapman, J., \& Saelens, B. E. (2005). Linking objectively measured physical activity with objectively measured urban form: Findings from SMARTRAQ. \emph{American Journal of Preventive Medicine}, \emph{28}(2), 117--125.

\leavevmode\vadjust pre{\hypertarget{ref-frank2003health}{}}%
Frank, L., Engelke, P., \& Schmid, T. (2003). \emph{Health and community design: The impact of the built environment on physical activity}. Island Press.

\leavevmode\vadjust pre{\hypertarget{ref-fransen2015identifying}{}}%
Fransen, K., Neutens, T., Farber, S., De Maeyer, P., Deruyter, G., \& Witlox, F. (2015). Identifying public transport gaps using time-dependent accessibility levels. \emph{Journal of Transport Geography}, \emph{48}, 176--187.

\leavevmode\vadjust pre{\hypertarget{ref-friedrich2007towards}{}}%
Friedrich, W. R., \& Van Der Poll, J. A. (2007). Towards a methodology to elicit tacit domain knowledge from users. \emph{Interdisciplinary Journal of Information, Knowledge, and Management}, \emph{2}, 179.

\leavevmode\vadjust pre{\hypertarget{ref-furth2016network}{}}%
Furth, P. G., Mekuria, M. C., \& Nixon, H. (2016). Network connectivity for low-stress bicycling. \emph{Transportation Research Record}, \emph{2587}(1), 41--49.

\leavevmode\vadjust pre{\hypertarget{ref-garcia2013walking}{}}%
Garcia-Palomares, J. C., Gutierrez, J., \& Cardozo, O. D. (2013). Walking accessibility to public transport: An analysis based on microdata and GIS. \emph{Environment and Planning B: Planning and Design}, \emph{40}(6), 1087--1102.

\leavevmode\vadjust pre{\hypertarget{ref-garling2014handbook}{}}%
Garling, T., Ettema, D., \& Friman, M. (2014). \emph{Handbook of sustainable travel}. Springer.

\leavevmode\vadjust pre{\hypertarget{ref-geurs2003evaluation}{}}%
Geurs, K. T., Eck, J. R. van, et al. (2003). Evaluation of accessibility impacts of land-use scenarios: The implications of job competition, land-use, and infrastructure developments for the netherlands. \emph{Environment and Planning B}, \emph{30}(1), 69--88.

\leavevmode\vadjust pre{\hypertarget{ref-geurs2001accessibility}{}}%
Geurs, K. T., \& Ritsema van Eck, J. R. (2001). Accessibility measures: Review and applications. Evaluation of accessibility impacts of land-use transportation scenarios, and related social and economic impact. \emph{RIVM Rapport 408505006}.

\leavevmode\vadjust pre{\hypertarget{ref-geurs2004}{}}%
Geurs, K. T., \& Van Wee, B. (2004). Accessibility evaluation of land-use and transport strategies: Review and research directions. \emph{Journal of Transport Geography}, \emph{12}(2), 127--140.

\leavevmode\vadjust pre{\hypertarget{ref-gossling2019social}{}}%
Gossling, S., Choi, A., Dekker, K., \& Metzler, D. (2019). The social cost of automobility, cycling and walking in the european union. \emph{Ecological Economics}, \emph{158}, 65--74.

\leavevmode\vadjust pre{\hypertarget{ref-grasser2013objectively}{}}%
Grasser, G., Van Dyck, D., Titze, S., \& Stronegger, W. (2013). Objectively measured walkability and active transport and weight-related outcomes in adults: A systematic review. \emph{International Journal of Public Health}, \emph{58}(4), 615--625.

\leavevmode\vadjust pre{\hypertarget{ref-greenberg2005does}{}}%
Greenberg, M. R., \& Renne, J. (2005). Where does walkability matter the most? An environmental justice interpretation of new jersey data. \emph{Journal of Urban Health}, \emph{82}(1), 90--100.

\leavevmode\vadjust pre{\hypertarget{ref-gupta1995aggregate}{}}%
Gupta, A., Harinarayan, V., \& Quass, D. (1995). Aggregate-query processing in data warehousing environments.

\leavevmode\vadjust pre{\hypertarget{ref-hadar2014role}{}}%
Hadar, I., Soffer, P., \& Kenzi, K. (2014). The role of domain knowledge in requirements elicitation via interviews: An exploratory study. \emph{Requirements Engineering}, \emph{19}, 143--159.

\leavevmode\vadjust pre{\hypertarget{ref-hakim1998effects}{}}%
Hakim, A. A., Petrovitch, H., Burchfiel, C. M., Ross, G. W., Rodriguez, B. L., White, L. R., \ldots{} Abbott, R. D. (1998). Effects of walking on mortality among nonsmoking retired men. \emph{New England Journal of Medicine}, \emph{338}(2), 94--99.

\leavevmode\vadjust pre{\hypertarget{ref-halden2000accessibility}{}}%
Halden, D., Mcguigan, D., Nisbet, A., \& Mckinnon, A. (2000). \emph{Accessibility: Review of measuring techniques and their application}. Great Britain, Scottish Executive, Central Research Unit Edinburgh, UK.

\leavevmode\vadjust pre{\hypertarget{ref-handy2020}{}}%
Handy, S. (2020). Is accessibility an idea whose time has finally come? \emph{Transportation Research Part D: Transport and Environment}, \emph{83}, 102319.

\leavevmode\vadjust pre{\hypertarget{ref-handy2002accessibility}{}}%
Handy, S. L. (2002). Accessibility-vs. Mobility-enhancing strategies for addressing automobile dependence in the US.

\leavevmode\vadjust pre{\hypertarget{ref-handy1997measuring}{}}%
Handy, S. L., \& Niemeier, D. A. (1997). Measuring accessibility: An exploration of issues and alternatives. \emph{Environment and Planning A}, \emph{29}(7), 1175--1194.

\leavevmode\vadjust pre{\hypertarget{ref-hansen1959accessibility}{}}%
Hansen, W. G. (1959). How accessibility shapes land use. \emph{Journal of the American Institute of Planners}, \emph{25}(2), 73--76.

\leavevmode\vadjust pre{\hypertarget{ref-harkey1998development}{}}%
Harkey, D. L., Reinfurt, D. W., \& Knuiman, M. (1998). Development of the bicycle compatibility index. \emph{Transportation Research Record}, \emph{1636}(1), 13--20.

\leavevmode\vadjust pre{\hypertarget{ref-harkey1998Development}{}}%
Harkey, D. L., Reinfurt, D. W., Knuiman, M., Stewart, J. R., Sorton, A., et al. (1998). \emph{Development of the bicycle compatibility index: A level of service concept, final report}. United States. Federal Highway Administration.

\leavevmode\vadjust pre{\hypertarget{ref-harris2001accessibility}{}}%
Harris, B. (2001). Accessibility: Concepts and applications. \emph{Journal of Transportation and Statistics}, \emph{4}(2/3), 15--30.

\leavevmode\vadjust pre{\hypertarget{ref-hochmair2015assessment}{}}%
Hochmair, H. H. (2015). Assessment of bicycle service areas around transit stations. \emph{International Journal of Sustainable Transportation}, \emph{9}(1), 15--29.

\leavevmode\vadjust pre{\hypertarget{ref-hoedl2010bikeability}{}}%
Hoedl, S., Titze, S., \& Oja, P. (2010). The bikeability and walkability evaluation table: Reliability and application. \emph{American Journal of Preventive Medicine}, \emph{39}(5), 457--459.

\leavevmode\vadjust pre{\hypertarget{ref-horacek2012sneakers}{}}%
Horacek, T. M., White, A. A., Greene, G. W., Reznar, M. M., Quick, V. M., Morrell, J. S., et al.others. (2012). Sneakers and spokes: An assessment of the walkability and bikeability of US postsecondary institutions. \emph{Journal of Environmental Health}, \emph{74}(7), 8--15.

\leavevmode\vadjust pre{\hypertarget{ref-hsu2014}{}}%
Hsu, C.-I., \& Tsai, Y.-C. (2014). An energy expenditure approach for estimating walking distance. \emph{Environment and Planning B: Planning and Design}, \emph{41}(2), 289--306.

\leavevmode\vadjust pre{\hypertarget{ref-hull2012accessibility}{}}%
Hull, A., Silva, C., \& Bertolini, L. (2012). \emph{Accessibility instruments for planning practice}. Cost Office Brussels.

\leavevmode\vadjust pre{\hypertarget{ref-hunt2007influences}{}}%
Hunt, J. D., \& Abraham, J. E. (2007). Influences on bicycle use. \emph{Transportation}, \emph{34}(4), 453--470.

\leavevmode\vadjust pre{\hypertarget{ref-iacono2010}{}}%
Iacono, M., Krizek, K. J., \& El-Geneidy, A. (2010). Measuring non-motorized accessibility: Issues, alternatives, and execution. \emph{Journal of Transport Geography}, \emph{18}(1), 133--140.

\leavevmode\vadjust pre{\hypertarget{ref-ilagcrstrand1970people}{}}%
Ilagcrstrand, T. (1970). What about people in regional science. \emph{Regional Science Association}, \emph{24}.

\leavevmode\vadjust pre{\hypertarget{ref-imani2019cycle}{}}%
Imani, A. F., Miller, E. J., \& Saxe, S. (2019). Cycle accessibility and level of traffic stress: A case study of toronto. \emph{Journal of Transport Geography}, \emph{80}, 102496.

\leavevmode\vadjust pre{\hypertarget{ref-ingram1971concept}{}}%
Ingram, D. R. (1971). The concept of accessibility: A search for an operational form. \emph{Regional Studies}, \emph{5}(2), 101--107.

\leavevmode\vadjust pre{\hypertarget{ref-iseki2014}{}}%
Iseki, H., \& Tingstrom, M. (2014). A new approach for bikeshed analysis with consideration of topography, street connectivity, and energy consumption. \emph{Computers, Environment and Urban Systems}, \emph{48}, 166--177.

\leavevmode\vadjust pre{\hypertarget{ref-jabbari2021accessibility}{}}%
Jabbari, M., Fonseca, F., \& Ramos, R. (2021). Accessibility and connectivity criteria for assessing walkability: An application in qazvin, iran. \emph{Sustainability}, \emph{13}(7), 3648.

\leavevmode\vadjust pre{\hypertarget{ref-kelobonye2019relative}{}}%
Kelobonye, K., McCarney, G., Xia, J. C., Swapan, M. S. H., Mao, F., \& Zhou, H. (2019). Relative accessibility analysis for key land uses: A spatial equity perspective. \emph{Journal of Transport Geography}, \emph{75}, 82--93.

\leavevmode\vadjust pre{\hypertarget{ref-khreis2016health}{}}%
Khreis, H., Warsow, K. M., Verlinghieri, E., Guzman, A., Pellecuer, L., Ferreira, A., et al.others. (2016). The health impacts of traffic-related exposures in urban areas: Understanding real effects, underlying driving forces and co-producing future directions. \emph{Journal of Transport \& Health}, \emph{3}(3), 249--267.

\leavevmode\vadjust pre{\hypertarget{ref-kimball2011data}{}}%
Kimball, R., \& Ross, M. (2011). \emph{The data warehouse toolkit: The complete guide to dimensional modeling}. John Wiley \& Sons.

\leavevmode\vadjust pre{\hypertarget{ref-koenig1980indicators}{}}%
Koenig, J.-G. (1980). Indicators of urban accessibility: Theory and application. \emph{Transportation}, \emph{9}(2), 145--172.

\leavevmode\vadjust pre{\hypertarget{ref-koglin2020}{}}%
Koglin, T. (2020). Spatial dimensions of the marginalisation of cycling--marginalization through rationalisation? In \emph{The politics of cycling infrastructure} (pp. 55--72). Policy Press.

\leavevmode\vadjust pre{\hypertarget{ref-koszowski2019active}{}}%
Koszowski, C., Gerike, R., Hubrich, S., Gotschi, T., Pohle, M., \& Wittwer, R. (2019). Active mobility: Bringing together transport planning, urban planning, and public health. In \emph{Towards user-centric transport in europe} (pp. 149--171). Springer.

\leavevmode\vadjust pre{\hypertarget{ref-krizek2009explaining}{}}%
Krizek, K. J., Handy, S. L., \& Forsyth, A. (2009). Explaining changes in walking and bicycling behavior: Challenges for transportation research. \emph{Environment and Planning B: Planning and Design}, \emph{36}(4), 725--740.

\leavevmode\vadjust pre{\hypertarget{ref-kwan1998}{}}%
Kwan, M.-P. (1998). Space-time and integral measures of individual accessibility: A comparative analysis using a point-based framework. \emph{Geographical Analysis}, \emph{30}(3), 191--216.

\leavevmode\vadjust pre{\hypertarget{ref-landis1997real}{}}%
Landis, B. W., Vattikuti, V. R., \& Brannick, M. T. (1997). Real-time human perceptions: Toward a bicycle level of service. \emph{Transportation Research Record}, \emph{1578}(1), 119--126.

\leavevmode\vadjust pre{\hypertarget{ref-landis2003intersection}{}}%
Landis, B. W., Vattikuti, V. R., Ottenberg, R. M., Petritsch, T. A., Guttenplan, M., \& Crider, L. B. (2003). Intersection level of service for the bicycle through movement. \emph{Transportation Research Record}, \emph{1828}(1), 101--106.

\leavevmode\vadjust pre{\hypertarget{ref-larsen2011travel}{}}%
Larsen, J., \& El-Geneidy, A. (2011). A travel behavior analysis of urban cycling facilities in montreal, canada. \emph{Transportation Research Part D: Transport and Environment}, \emph{16}(2), 172--177.

\leavevmode\vadjust pre{\hypertarget{ref-lavery2013driving}{}}%
Lavery, T., Paez, A., \& Kanaroglou, P. S. (2013). Driving out of choices: An investigation of transport modality in a university sample. \emph{Transportation Research Part A: Policy and Practice}, \emph{57}, 37--46.

\leavevmode\vadjust pre{\hypertarget{ref-lee2010reexamining}{}}%
Lee, B. H., Waddell, P., Wang, L., \& Pendyala, R. M. (2010). Reexamining the influence of work and nonwork accessibility on residential location choices with a microanalytic framework. \emph{Environment and Planning A}, \emph{42}(4), 913--930.

\leavevmode\vadjust pre{\hypertarget{ref-lee1998application}{}}%
Lee, M. S., \& McNally, M. G. (1998). Application of space-time prisms for the measurement of accessibility.

\leavevmode\vadjust pre{\hypertarget{ref-legrain2015and}{}}%
Legrain, A., Buliung, R., \& El-Geneidy, A. M. (2015). Who, what, when, and where: Revisiting the influences of transit mode share. \emph{Transportation Research Record}, \emph{2537}(1), 42--51.

\leavevmode\vadjust pre{\hypertarget{ref-legrain2016travelling}{}}%
Legrain, A., Buliung, R., \& El-Geneidy, A. M. (2016). Travelling fair: Targeting equitable transit by understanding job location, sectorial concentration, and transit use among low-wage workers. \emph{Journal of Transport Geography}, \emph{53}, 1--11.

\leavevmode\vadjust pre{\hypertarget{ref-leslie2007walkability}{}}%
Leslie, E., Coffee, N., Frank, L., Owen, N., Bauman, A., \& Hugo, G. (2007). Walkability of local communities: Using geographic information systems to objectively assess relevant environmental attributes. \emph{Health \& Place}, \emph{13}(1), 111--122.

\leavevmode\vadjust pre{\hypertarget{ref-leveille1999aging}{}}%
Leveille, S. G., Guralnik, J. M., Ferrucci, L., \& Langlois, J. A. (1999). Aging successfully until death in old age: Opportunities for increasing active life expectancy. \emph{American Journal of Epidemiology}, \emph{149}(7), 654--664.

\leavevmode\vadjust pre{\hypertarget{ref-levine2010zoned}{}}%
Levine, J. (2010). \emph{Zoned out: Regulation, markets, and choices in transportation and metropolitan land use}. RFF Press.

\leavevmode\vadjust pre{\hypertarget{ref-levinson2020towards}{}}%
Levinson, D., \& Wu, H. (2020). Towards a general theory of access. \emph{Journal of Transport and Land Use}, \emph{13}(1), 129--158.

\leavevmode\vadjust pre{\hypertarget{ref-li2020}{}}%
Li, A., Huang, Y., \& Axhausen, K. W. (2020). An approach to imputing destination activities for inclusion in measures of bicycle accessibility. \emph{Journal of Transport Geography}, \emph{82}, 102566.

\leavevmode\vadjust pre{\hypertarget{ref-lira2021}{}}%
Lira, B. M., \& Paez, A. (2021). Do drivers dream of walking? An investigation of travel mode dissonance from the perspective of affective values. \emph{Journal of Transport \& Health}, \emph{20}, 101015.

\leavevmode\vadjust pre{\hypertarget{ref-lomet1990hb}{}}%
Lomet, D. B., \& Salzberg, B. (1990). The hB-tree: A multiattribute indexing method with good guaranteed performance. \emph{ACM Transactions on Database Systems (TODS)}, \emph{15}(4), 625--658.

\leavevmode\vadjust pre{\hypertarget{ref-lowry2012assessment}{}}%
Lowry, M. B., Callister, D., Gresham, M., \& Moore, B. (2012). Assessment of communitywide bikeability with bicycle level of service. \emph{Transportation Research Record}, \emph{2314}(1), 41--48.

\leavevmode\vadjust pre{\hypertarget{ref-lucas2012transport}{}}%
Lucas, K. (2012). Transport and social exclusion: Where are we now? \emph{Transport Policy}, \emph{20}, 105--113.

\leavevmode\vadjust pre{\hypertarget{ref-lucas2019new}{}}%
Lucas, K. (2019). A new evolution for transport-related social exclusion research? \emph{Journal of Transport Geography}, \emph{81}, 102529.

\leavevmode\vadjust pre{\hypertarget{ref-lundberg2012accessibility}{}}%
Lundberg, B. (2012). \emph{Accessibility and university populations: Local effects on non-motorized transportation in the tuscaloosa-northport area} (PhD thesis). University of Alabama Libraries.

\leavevmode\vadjust pre{\hypertarget{ref-luo2003measures}{}}%
Luo, W., \& Wang, F. (2003). Measures of spatial accessibility to health care in a GIS environment: Synthesis and a case study in the chicago region. \emph{Environment and Planning B: Planning and Design}, \emph{30}(6), 865--884.

\leavevmode\vadjust pre{\hypertarget{ref-madsen2014developing}{}}%
Madsen, T., Schipperijn, J., Christiansen, L. B., Nielsen, T. S., \& Troelsen, J. (2014). Developing suitable buffers to capture transport cycling behavior. \emph{Frontiers in Public Health}, \emph{2}, 61.

\leavevmode\vadjust pre{\hypertarget{ref-maghelal2011}{}}%
Maghelal, P. K., \& Capp, C. J. (2011). Walkability: A review of existing pedestrian indices. \emph{Journal of the Urban \& Regional Information Systems Association}, \emph{23}(2).

\leavevmode\vadjust pre{\hypertarget{ref-martens2012justice}{}}%
Martens, K., Golub, A., \& Robinson, G. (2012). A justice-theoretic approach to the distribution of transportation benefits: Implications for transportation planning practice in the united states. \emph{Transportation Research Part A: Policy and Practice}, \emph{46}(4), 684--695.

\leavevmode\vadjust pre{\hypertarget{ref-milakis2015people}{}}%
Milakis, D., Cervero, R., Van Wee, B., \& Maat, K. (2015). Do people consider an acceptable travel time? Evidence from berkeley, CA. \emph{Journal of Transport Geography}, \emph{44}, 76--86.

\leavevmode\vadjust pre{\hypertarget{ref-miller1991modelling}{}}%
Miller, H. J. (1991). Modelling accessibility using space-time prism concepts within geographical information systems. \emph{International Journal of Geographical Information System}, \emph{5}(3), 287--301.

\leavevmode\vadjust pre{\hypertarget{ref-miller2004tobler}{}}%
Miller, H. J. (2004). Toblers first law and spatial analysis. \emph{Annals of the Association of American Geographers}, \emph{94}(2), 284--289.

\leavevmode\vadjust pre{\hypertarget{ref-miller2005place}{}}%
Miller, H. J. (2005). Place-based versus people-based accessibility. In \emph{Access to destinations} (pp. 63--89). Emerald Group Publishing Limited.

\leavevmode\vadjust pre{\hypertarget{ref-miller2009field}{}}%
Miller, H. J., \& Bridwell, S. A. (2009). A field-based theory for time geography. \emph{Annals of the Association of American Geographers}, \emph{99}(1), 49--75.

\leavevmode\vadjust pre{\hypertarget{ref-millera2011}{}}%
Millera, H. J. (2011). Collaborative mobility: Using geographic information science to cultivate cooperative transportation systems. \emph{Procedia-Social and Behavioral Sciences}, \emph{21}, 24--28.

\leavevmode\vadjust pre{\hypertarget{ref-millward2013active}{}}%
Millward, H., Spinney, J., \& Scott, D. (2013). Active-transport walking behavior: Destinations, durations, distances. \emph{Journal of Transport Geography}, \emph{28}, 101--110.

\leavevmode\vadjust pre{\hypertarget{ref-milne2012public}{}}%
Milne, E. M. (2012). A public health perspective on transport policy priorities. \emph{Journal of Transport Geography}, \emph{21}, 62--69.

\leavevmode\vadjust pre{\hypertarget{ref-morris1979accessibility}{}}%
Morris, J. M., Dumble, P. L., \& Wigan, M. R. (1979). Accessibility indicators for transport planning. \emph{Transportation Research Part A: General}, \emph{13}(2), 91--109.

\leavevmode\vadjust pre{\hypertarget{ref-mueller2015health}{}}%
Mueller, N., Rojas-Rueda, D., Cole-Hunter, T., De Nazelle, A., Dons, E., Gerike, R., \ldots{} Nieuwenhuijsen, M. (2015). Health impact assessment of active transportation: A systematic review. \emph{Preventive Medicine}, \emph{76}, 103--114.

\leavevmode\vadjust pre{\hypertarget{ref-murphy2019implementing}{}}%
Murphy, B., \& Owen, A. (2019). Implementing low-stress bicycle routing in national accessibility evaluation. \emph{Transportation Research Record}, \emph{2673}(5), 240--249.

\leavevmode\vadjust pre{\hypertarget{ref-neutens2015accessibility}{}}%
Neutens, T. (2015). Accessibility, equity and health care: Review and research directions for transport geographers. \emph{Journal of Transport Geography}, \emph{43}, 14--27.

\leavevmode\vadjust pre{\hypertarget{ref-ng2017cyclist}{}}%
Ng, A., Debnath, A. K., \& Heesch, K. C. (2017). Cyclist safety perceptions of cycling infrastructure at un-signalised intersections: Cross-sectional survey of queensland cyclists. \emph{Journal of Transport \& Health}, \emph{6}, 13--22.

\leavevmode\vadjust pre{\hypertarget{ref-nielsen2018bikeability}{}}%
Nielsen, T. A. S., \& Skov-Petersen, H. (2018). Bikeability--urban structures supporting cycling. Effects of local, urban and regional scale urban form factors on cycling from home and workplace locations in denmark. \emph{Journal of Transport Geography}, \emph{69}, 36--44.

\leavevmode\vadjust pre{\hypertarget{ref-oehl2019cyclists}{}}%
Oehl, M., Brandenburg, S., \& Huemer, A. K. (2019). Cyclists anger experiences in traffic: The cycling anger scale. \emph{Transportation Research Part F: Traffic Psychology and Behaviour}, \emph{62}, 564--574.

\leavevmode\vadjust pre{\hypertarget{ref-o2003aggregate}{}}%
OKelly, M. E., \& Horner, M. W. (2003). Aggregate accessibility to population at the county level: US 1940-2000. \emph{Journal of Geographical Systems}, \emph{5}(1), 5.

\leavevmode\vadjust pre{\hypertarget{ref-ortega2021evaluating}{}}%
Ortega, E., MartiN, B., LOPez-Lambas, M. E., \& Soria-Lara, J. A. (2021). Evaluating the impact of urban design scenarios on walking accessibility: The case of the madrid `centro'district. \emph{Sustainable Cities and Society}, \emph{74}, 103156.

\leavevmode\vadjust pre{\hypertarget{ref-otero2018health}{}}%
Otero, I., Nieuwenhuijsen, M., \& Rojas-Rueda, D. (2018). Health impacts of bike sharing systems in europe. \emph{Environment International}, \emph{115}, 387--394.

\leavevmode\vadjust pre{\hypertarget{ref-paez2020}{}}%
Paez, A., Anjum, Z., Dickson-Anderson, S. E., Schuster-Wallace, C. J., Ramos, B. M., \& Higgins, C. D. (2020). Comparing distance, time, and metabolic energy cost functions for walking accessibility in infrastructure-poor regions. \emph{Journal of Transport Geography}, \emph{82}, 102564.

\leavevmode\vadjust pre{\hypertarget{ref-paez2010accessibility}{}}%
Paez, A., Mercado, R. G., Farber, S., Morency, C., \& Roorda, M. (2010). Accessibility to health care facilities in montreal island: An application of relative accessibility indicators from the perspective of senior and non-senior residents. \emph{International Journal of Health Geographics}, \emph{9}(1), 1--15.

\leavevmode\vadjust pre{\hypertarget{ref-paez2012}{}}%
Paez, A., Scott, D. M., \& Morency, C. (2012a). Measuring accessibility: Positive and normative implementations of various accessibility indicators. \emph{Journal of Transport Geography}, \emph{25}, 141--153.

\leavevmode\vadjust pre{\hypertarget{ref-paez2012measuring}{}}%
Paez, A., Scott, D. M., \& Morency, C. (2012b). Measuring accessibility: Positive and normative implementations of various accessibility indicators. \emph{Journal of Transport Geography}, \emph{25}, 141--153.

\leavevmode\vadjust pre{\hypertarget{ref-papa2018measuring}{}}%
Papa, E., Carpentieri, G., \& Guida, C. (2018). Measuring walking accessibility to public transport for the elderly: The case of naples. \emph{TeMA-Journal of Land Use, Mobility and Environment}, 105--116.

\leavevmode\vadjust pre{\hypertarget{ref-parkin2010design}{}}%
Parkin, J., \& Rotheram, J. (2010). Design speeds and acceleration characteristics of bicycle traffic for use in planning, design and appraisal. \emph{Transport Policy}, \emph{17}(5), 335--341.

\leavevmode\vadjust pre{\hypertarget{ref-pearce2006neighbourhoods}{}}%
Pearce, J., Witten, K., \& Bartie, P. (2006). Neighbourhoods and health: A GIS approach to measuring community resource accessibility. \emph{Journal of Epidemiology \& Community Health}, \emph{60}(5), 389--395.

\leavevmode\vadjust pre{\hypertarget{ref-pereira2017distributive}{}}%
Pereira, R. H., Schwanen, T., \& Banister, D. (2017). Distributive justice and equity in transportation. \emph{Transport Reviews}, \emph{37}(2), 170--191.

\leavevmode\vadjust pre{\hypertarget{ref-pokorny2018conflicts}{}}%
Pokorny, P., Pritchard, R., \& Pitera, K. (2018). Conflicts between bikes and trucks in urban areas---a survey of norwegian cyclists. \emph{Case Studies on Transport Policy}, \emph{6}(1), 147--155.

\leavevmode\vadjust pre{\hypertarget{ref-pred1977choreography}{}}%
Pred, A. (1977). The choreography of existence: Comments on hagerstrands time-geography and its usefulness. \emph{Economic Geography}, \emph{53}(2), 207--221.

\leavevmode\vadjust pre{\hypertarget{ref-preston2007accessibility}{}}%
Preston, J., \& Raje, F. (2007). Accessibility, mobility and transport-related social exclusion. \emph{Journal of Transport Geography}, \emph{15}(3), 151--160.

\leavevmode\vadjust pre{\hypertarget{ref-prins2014many}{}}%
Prins, R. G., Pierik, F., Etman, A., Sterkenburg, R. P., Kamphuis, C. B., \& Van Lenthe, F. (2014). How many walking and cycling trips made by elderly are beyond commonly used buffer sizes: Results from a GPS study. \emph{Health \& Place}, \emph{27}, 127--133.

\leavevmode\vadjust pre{\hypertarget{ref-proffitt2019}{}}%
Proffitt, D. G., Bartholomew, K., Ewing, R., \& Miller, H. J. (2019). Accessibility planning in american metropolitan areas: Are we there yet? \emph{Urban Studies}, \emph{56}(1), 167--192.

\leavevmode\vadjust pre{\hypertarget{ref-pucher2010walking}{}}%
Pucher, J., Buehler, R., Bassett, D. R., \& Dannenberg, A. L. (2010). Walking and cycling to health: A comparative analysis of city, state, and international data. \emph{American Journal of Public Health}, \emph{100}(10), 1986--1992.

\leavevmode\vadjust pre{\hypertarget{ref-reggiani2011accessibility}{}}%
Reggiani, A., Bucci, P., \& Russo, G. (2011). Accessibility and impedance forms: Empirical applications to the german commuting network. \emph{International Regional Science Review}, \emph{34}(2), 230--252.

\leavevmode\vadjust pre{\hypertarget{ref-reyes2014walking}{}}%
Reyes, M., Paez, A., \& Morency, C. (2014). Walking accessibility to urban parks by children: A case study of montreal. \emph{Landscape and Urban Planning}, \emph{125}, 38--47.

\leavevmode\vadjust pre{\hypertarget{ref-rojas2011health}{}}%
Rojas-Rueda, D., De Nazelle, A., Tainio, M., \& Nieuwenhuijsen, M. J. (2011). The health risks and benefits of cycling in urban environments compared with car use: Health impact assessment study. \emph{Bmj}, \emph{343}.

\leavevmode\vadjust pre{\hypertarget{ref-rojas2012replacing}{}}%
Rojas-Rueda, D., Nazelle, A. de, Teixido, O., \& Nieuwenhuijsen, M. J. (2012). Replacing car trips by increasing bike and public transport in the greater barcelona metropolitan area: A health impact assessment study. \emph{Environment International}, \emph{49}, 100--109.

\leavevmode\vadjust pre{\hypertarget{ref-romero2010framework}{}}%
Romero, O., \& Abello, A. (2010). A framework for multidimensional design of data warehouses from ontologies. \emph{Data \& Knowledge Engineering}, \emph{69}(11), 1138--1157.

\leavevmode\vadjust pre{\hypertarget{ref-rosas2020}{}}%
Rosas-Satizabal, D., Guzman, L. A., \& Oviedo, D. (2020). Cycling diversity, accessibility, and equality: An analysis of cycling commuting in bogota. \emph{Transportation Research Part D: Transport and Environment}, \emph{88}, 102562.

\leavevmode\vadjust pre{\hypertarget{ref-ruffino2021}{}}%
Ruffino, P., \& Jarre, M. (2021). Appraisal of cycling and pedestrian projects. In \emph{Advances in transport policy and planning} (Vol. 7, pp. 165--203). Elsevier.

\leavevmode\vadjust pre{\hypertarget{ref-rybarczyk2014measuring}{}}%
Rybarczyk, G., \& Gallagher, L. (2014). Measuring the potential for bicycling and walking at a metropolitan commuter university. \emph{Journal of Transport Geography}, \emph{39}, 1--10.

\leavevmode\vadjust pre{\hypertarget{ref-rybarczyk2010bicycle}{}}%
Rybarczyk, G., \& Wu, C. (2010). Bicycle facility planning using GIS and multi-criteria decision analysis. \emph{Applied Geography}, \emph{30}(2), 282--293.

\leavevmode\vadjust pre{\hypertarget{ref-sadler2011application}{}}%
Sadler, R. C., Gilliland, J. A., \& Arku, G. (2011). An application of the edge effect in measuring accessibility to multiple food retailer types in southwestern ontario, canada. \emph{International Journal of Health Geographics}, \emph{10}(1), 1--15.

\leavevmode\vadjust pre{\hypertarget{ref-saelens2008built}{}}%
Saelens, B. E., \& Handy, S. L. (2008). Built environment correlates of walking: A review. \emph{Medicine and Science in Sports and Exercise}, \emph{40}(7 Suppl), S550.

\leavevmode\vadjust pre{\hypertarget{ref-saghapour2017measuring}{}}%
Saghapour, T., Moridpour, S., \& Thompson, R. G. (2017). Measuring cycling accessibility in metropolitan areas. \emph{International Journal of Sustainable Transportation}, \emph{11}(5), 381--394.

\leavevmode\vadjust pre{\hypertarget{ref-schoner2014missing}{}}%
Schoner, J. E., \& Levinson, D. M. (2014). The missing link: Bicycle infrastructure networks and ridership in 74 US cities. \emph{Transportation}, \emph{41}(6), 1187--1204.

\leavevmode\vadjust pre{\hypertarget{ref-sisson2006suitability}{}}%
Sisson, S. B., Lee, S. M., Burns, E. K., \& Tudor-Locke, C. (2006). Suitability of commuting by bicycle to arizona elementary schools. \emph{American Journal of Health Promotion}, \emph{20}(3), 210--213.

\leavevmode\vadjust pre{\hypertarget{ref-unit2003}{}}%
Social Exclusion Unit. (2003). Making the connections: Final report on transportation and social exclusion. \emph{Social Exclusion Unit. Retrieved September}, \emph{20}, 2015.

\leavevmode\vadjust pre{\hypertarget{ref-strawbridge1996successful}{}}%
Strawbridge, W. J., Cohen, R. D., Shema, S. J., \& Kaplan, G. A. (1996). Successful aging: Predictors and associated activities. \emph{American Journal of Epidemiology}, \emph{144}(2), 135--141.

\leavevmode\vadjust pre{\hypertarget{ref-talen2013}{}}%
Talen, E., \& Koschinsky, J. (2013). The walkable neighborhood: A literature review. \emph{International Journal of Sustainable Land Use and Urban Planning}, \emph{1}(1).

\leavevmode\vadjust pre{\hypertarget{ref-taylor2006}{}}%
Taylor, B. D. (2006). Putting a price on mobility: Cars and contradictions in planning. \emph{Journal of the American Planning Association}, \emph{72}(3), 279--284.

\leavevmode\vadjust pre{\hypertarget{ref-tilahun2007trails}{}}%
Tilahun, N. Y., Levinson, D. M., \& Krizek, K. J. (2007). Trails, lanes, or traffic: Valuing bicycle facilities with an adaptive stated preference survey. \emph{Transportation Research Part A: Policy and Practice}, \emph{41}(4), 287--301.

\leavevmode\vadjust pre{\hypertarget{ref-tinessa2021walkability}{}}%
Tinessa, F., Pagliara, F., Biggiero, L., \& Veneri, G. D. (2021). Walkability, accessibility to metro stations and retail location choice: Some evidence from the case study of naples. \emph{Research in Transportation Business \& Management}, \emph{40}, 100549.

\leavevmode\vadjust pre{\hypertarget{ref-tobler1993three}{}}%
Tobler, W. (1993). \emph{Three presentations on geographical analysis and modeling} (Vol. 93). Citeseer.

\leavevmode\vadjust pre{\hypertarget{ref-tong2015transportation}{}}%
Tong, L., Zhou, X., \& Miller, H. J. (2015). Transportation network design for maximizing space--time accessibility. \emph{Transportation Research Part B: Methodological}, \emph{81}, 555--576.

\leavevmode\vadjust pre{\hypertarget{ref-train2009discrete}{}}%
Train, K. E. (2009). \emph{Discrete choice methods with simulation}. Cambridge university press.

\leavevmode\vadjust pre{\hypertarget{ref-ulmer2003evaluating}{}}%
Ulmer, J. M., \& Hoel, L. A. (2003). \emph{Evaluating the accessibility of residential areas for bicycling and walking using GIS}. Citeseer.

\leavevmode\vadjust pre{\hypertarget{ref-useche2019healthy}{}}%
Useche, S. A., Montoro, L., Sanmartin, J., \& Alonso, F. (2019). Healthy but risky: A descriptive study on cyclists encouraging and discouraging factors for using bicycles, habits and safety outcomes. \emph{Transportation Research Part F: Traffic Psychology and Behaviour}, \emph{62}, 587--598.

\leavevmode\vadjust pre{\hypertarget{ref-vale2009sustainable}{}}%
Vale, D. de S. (2009). \emph{Sustainable urban form, accessibility and travel: The relationship between polycentric urban development and commuting in lisbon} (PhD thesis). Newcastle University.

\leavevmode\vadjust pre{\hypertarget{ref-vale2017influence}{}}%
Vale, D. S., \& Pereira, M. (2017). The influence of the impedance function on gravity-based pedestrian accessibility measures: A comparative analysis. \emph{Environment and Planning B: Urban Analytics and City Science}, \emph{44}(4), 740--763.

\leavevmode\vadjust pre{\hypertarget{ref-vale2016active}{}}%
Vale, D. S., Saraiva, M., \& Pereira, M. (2016). Active accessibility: A review of operational measures of walking and cycling accessibility. \emph{Journal of Transport and Land Use}, \emph{9}(1), 209--235.

\leavevmode\vadjust pre{\hypertarget{ref-vandenbulcke2009mapping}{}}%
Vandenbulcke, G., Steenberghen, T., \& Thomas, I. (2009). Mapping accessibility in belgium: A tool for land-use and transport planning? \emph{Journal of Transport Geography}, \emph{17}(1), 39--53.

\leavevmode\vadjust pre{\hypertarget{ref-vasconcelos2012evaluation}{}}%
Vasconcelos, A. S., \& Farias, T. L. (2012). Evaluation of urban accessibility indicators based on internal and external environmental costs. \emph{Transportation Research Part D: Transport and Environment}, \emph{17}(6), 433--441.

\leavevmode\vadjust pre{\hypertarget{ref-wachs1973physical}{}}%
Wachs, M., \& Kumagai, T. G. (1973). Physical accessibility as a social indicator. \emph{Socio-Economic Planning Sciences}, \emph{7}(5), 437--456.

\leavevmode\vadjust pre{\hypertarget{ref-wahlgren2012exploring}{}}%
Wahlgren, L., \& Schantz, P. (2012). Exploring bikeability in a metropolitan setting: Stimulating and hindering factors in commuting route environments. \emph{BMC Public Health}, \emph{12}(1), 1--16.

\leavevmode\vadjust pre{\hypertarget{ref-WalkScore2020}{}}%
Walk Score. (2020). Walk score. Retrieved from \url{https://www.walkscore.com/}

\leavevmode\vadjust pre{\hypertarget{ref-wang2015gis}{}}%
Wang, C.-H., \& Chen, N. (2015). A GIS-based spatial statistical approach to modeling job accessibility by transportation mode: Case study of columbus, ohio. \emph{Journal of Transport Geography}, \emph{45}, 1--11.

\leavevmode\vadjust pre{\hypertarget{ref-weiss2010promoting}{}}%
Weiss, R. L., Maantay, J. A., \& Fahs, M. (2010). Promoting active urban aging: A measurement approach to neighborhood walkability for older adults. \emph{Cities and the Environment}, \emph{3}(1), 12.

\leavevmode\vadjust pre{\hypertarget{ref-weuve2004physical}{}}%
Weuve, J., Kang, J. H., Manson, J. E., Breteler, M. M., Ware, J. H., \& Grodstein, F. (2004). Physical activity, including walking, and cognitive function in older women. \emph{Jama}, \emph{292}(12), 1454--1461.

\leavevmode\vadjust pre{\hypertarget{ref-wiersma2020}{}}%
Wiersma, J. et al. (2020). Commuting patterns and car dependency in urban regions. \emph{Journal of Transport Geography}, \emph{84}(C).

\leavevmode\vadjust pre{\hypertarget{ref-winters2013mapping}{}}%
Winters, M., Brauer, M., Setton, E. M., \& Teschke, K. (2013). Mapping bikeability: A spatial tool to support sustainable travel. \emph{Environment and Planning B: Planning and Design}, \emph{40}(5), 865--883.

\leavevmode\vadjust pre{\hypertarget{ref-wood2018pedestrian}{}}%
Wood, N., Jones, J., Peters, J., \& Richards, K. (2018). Pedestrian evacuation modeling to reduce vehicle use for distant tsunami evacuations in hawaii. \emph{International Journal of Disaster Risk Reduction}, \emph{28}, 271--283.

\leavevmode\vadjust pre{\hypertarget{ref-wu2019measuring}{}}%
Wu, X., Lu, Y., Lin, Y., \& Yang, Y. (2019). Measuring the destination accessibility of cycling transfer trips in metro station areas: A big data approach. \emph{International Journal of Environmental Research and Public Health}, \emph{16}(15), 2641.

\leavevmode\vadjust pre{\hypertarget{ref-wu2001computational}{}}%
Wu, Y.-H., \& Miller, H. J. (2001). Computational tools for measuring space-time accessibility within dynamic flow transportation networks. \emph{Journal of Transportation and Statistics}, \emph{4}(2/3), 1--14.

\leavevmode\vadjust pre{\hypertarget{ref-xing2018measuring}{}}%
Xing, L., Liu, Y., \& Liu, X. (2018). Measuring spatial disparity in accessibility with a multi-mode method based on park green spaces classification in wuhan, china. \emph{Applied Geography}, \emph{94}, 251--261.

\leavevmode\vadjust pre{\hypertarget{ref-yang2012walking}{}}%
Yang, Y., \& Diez-Roux, A. V. (2012). Walking distance by trip purpose and population subgroups. \emph{American Journal of Preventive Medicine}, \emph{43}(1), 11--19.

\leavevmode\vadjust pre{\hypertarget{ref-zacharias2001pedestrian}{}}%
Zacharias, J. (2001). Pedestrian behavior pedestrian behavior and perception in urban walking environments. \emph{Journal of Planning Literature}, \emph{16}(1), 3--18.

\leavevmode\vadjust pre{\hypertarget{ref-zahran2008cycling}{}}%
Zahran, S., Brody, S. D., Maghelal, P., Prelog, A., \& Lacy, M. (2008). Cycling and walking: Explaining the spatial distribution of healthy modes of transportation in the united states. \emph{Transportation Research Part D: Transport and Environment}, \emph{13}(7), 462--470.

\leavevmode\vadjust pre{\hypertarget{ref-zhuge1996strobe}{}}%
Zhuge, Y., Garcia-Molina, H., \& Wiener, J. L. (1996). The strobe algorithms for multi-source warehouse consistency. In \emph{Fourth international conference on parallel and distributed information systems} (pp. 146--157). IEEE.

\leavevmode\vadjust pre{\hypertarget{ref-zielstra2011comparative}{}}%
Zielstra, D., \& Hochmair, H. H. (2011). Comparative study of pedestrian accessibility to transit stations using free and proprietary network data. \emph{Transportation Research Record}, \emph{2217}(1), 145--152.

\end{CSLReferences}

\end{document}
